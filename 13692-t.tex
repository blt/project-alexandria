\documentclass[oneside]{book}
\usepackage[latin1]{inputenc}
\usepackage[reqno]{amsmath}
\usepackage{amssymb,graphicx,yfonts}
\usepackage{makeidx}
\makeindex
\renewcommand{\chaptername}{Article}
\DeclareMathOperator{\am}{am}
\DeclareMathOperator{\amh}{amh}
\DeclareMathOperator{\cg}{cg}
\DeclareMathOperator{\cn}{cn}
\DeclareMathOperator{\csch}{csch}
\DeclareMathOperator{\dn}{dn}
\DeclareMathOperator{\gd}{gd}
\DeclareMathOperator{\limdot}{lim.}
\DeclareMathOperator{\moddot}{mod.}
\DeclareMathOperator{\sech}{sech}
\DeclareMathOperator{\sg}{sg}
\DeclareMathOperator{\sn}{sn}
\DeclareMathOperator{\tg}{tg}

\begin{document}

\thispagestyle{empty}
\small
\begin{verbatim}

The Project Gutenberg EBook of Hyperbolic Functions, by James McMahon

This eBook is for the use of anyone anywhere at no cost and with
almost no restrictions whatsoever.  You may copy it, give it away or
re-use it under the terms of the Project Gutenberg License included
with this eBook or online at www.gutenberg.net


Title: Hyperbolic Functions

Author: James McMahon

Release Date: October 10, 2004 [EBook #13692]

Language: English

Character set encoding: TeX

*** START OF THIS PROJECT GUTENBERG EBOOK HYPERBOLIC FUNCTIONS ***




Produced by David Starner, Joshua Hutchinson, John Hagerson, 
and the Project Gutenberg On-line Distributed Proofreading Team.






\end{verbatim}
\normalsize
\newpage

\index{Equations!Differential|see{Differential Equation}}
\index{Function!anti-gudermanian|see{Anti-gu\-der\-man\-i\-an}}
\index{Function!anti-hyperbolic|see{Anti-hy\-per\-bo\-lic
functions}}
\index{Function!circular|see{Circular functions}}
\index{Function!elliptic|see{Elliptic functions}}
\index{Function!gudermanian|see{Gudermanian function}}
\index{Imaginary|see{Complex}}

\frontmatter

\begin{center}
\noindent \Large MATHEMATICAL MONOGRAPHS.

\bigskip \footnotesize{\textsc{edited by}} \\
\normalsize \textsc{MANSFIELD MERRIMAN and ROBERT S. WOODWARD.}

\bigskip\bigskip\huge
No. 4.

\bigskip
HYPERBOLIC FUNCTIONS.

\bigskip\bigskip\footnotesize\textsc{by} \\
\bigskip\large JAMES McMAHON, \\
\footnotesize\textsc{Professor of Mathematics in Cornell
University.}

\bigskip\bigskip\normalsize NEW YORK: \\
\medskip JOHN WILEY \& SONS. \\
\medskip \textsc{London: CHAPMAN \& HALL, Limited.} \\
\medskip 1906.

\bigskip\bigskip
\tiny \textsc{Copyright 1896} \\
\textsc{by} \\
\textsc{MANSFIELD MERRIMAN and ROBERT S. WOODWARD} \\
\textsc{under the title} \\
\textsc{HIGHER MATHEMATICS} \normalsize
\end{center}

\bigskip\bigskip
\scriptsize \noindent \textsc{Transcriber's Note:} \emph{I did my
best to recreate the index.} \normalsize

\newpage

\fbox{\parbox{11cm}{
\begin{center}
\textbf{MATHEMATICAL MONOGRAPHS.} \\
\small\textsc{edited by}\normalsize \\
\textbf{Mansfield Merriman and Robert S. Woodward.} \\
\smallskip \footnotesize \textbf{Octavo. Cloth.} \\
\end{center}
\begin{tabbing}
No. 99999\= \kill
\textbf{No. 1.}\>\textbf{History of Modern Mathematics.} \\
\>By \textsc{David Eugene Smith.} \$1.00 \emph{net}. \\
\smallskip
\textbf{No. 2.}\>\textbf{Synthetic Projective Geometry.} \\
\>By \textsc{George Bruce Halsted.} \$1.00 \emph{net}. \\
\smallskip
\textbf{No. 3.}\>\textbf{Determinants.} \\
\>By \textsc{Laenas Gifford Weld.} \$1.00 \emph{net}. \\
\smallskip
\textbf{No. 4.}\>\textbf{Hyperbolic Functions.} \\
\>By \textsc{James McMahon.} \$1.00 \emph{net}. \\
\smallskip
\textbf{No. 5.}\>\textbf{Harmonic Functions.} \\
\>By \textsc{William E. Byerly.} \$1.00 \emph{net}. \\
\smallskip
\textbf{No. 6.}\>\textbf{Grassmann's Space Analysis.} \\
\>By \textsc{Edward W. Hyde.} \$1.00 \emph{net}. \\
\smallskip
\textbf{No. 7.}\>\textbf{Probability and Theory of Errors.} \\
\>By \textsc{Robert S. Woodward.} \$1.00 \emph{net}. \\
\smallskip
\textbf{No. 8.}\>\textbf{Vector Analysis and Quaternions.} \\
\>By \textsc{Alexander Macfarlane.} \$1.00 \emph{net}. \\
\smallskip
\textbf{No. 9.}\>\textbf{Differential Equations.} \\
\>By \textsc{William Woolsey Johnson.} \$1.00 \emph{net}. \\
\smallskip
\textbf{No. 10.}\>\textbf{The Solution of Equations.} \\
\>By \textsc{Mansfield Merriman.} \$1.00 \emph{net}. \\
\smallskip
\textbf{No. 11.}\>\textbf{Functions of a Complex Variable.} \\
\>By \textsc{Thomas S. Fiske.} \$1.00 \emph{net}. \\
\smallskip
\textbf{No. 12.}\>\textbf{The Theory of Relativity.} \\
\>By \textsc{Robert D. Carmichael.} \$1.00 \emph{net}. \\
\smallskip
\textbf{No. 13.}\>\textbf{The Theory of Numbers.} \\
\>By \textsc{Robert D. Carmichael.} \$1.00 \emph{net}. \\
\smallskip
\textbf{No. 14.}\>\textbf{Algebraic Invariants.} \\
\>By \textsc{Leonard E. Dickson.} \$1.25 \emph{net}. \\
\end{tabbing}
\begin{center}
\smallskip \normalsize PUBLISHED BY \\
\smallskip \textbf{JOHN WILEY \& SONS, Inc., NEW YORK. \\
CHAPMAN \& HALL, Limited, LONDON.}
\end{center}}}

\chapter{Editors' Preface.}

The volume called Higher Mathematics, the first edition of which was
published in 1896, contained eleven chapters by eleven authors, each
chapter being independent of the others, but all supposing the
reader to have at least a mathematical training equivalent to that
given in classical and engineering colleges. The publication of that
volume is now discontinued and the chapters are issued in separate
form. In these reissues it will generally be found that the
monographs are enlarged by additional articles or appendices which
either amplify the former presentation or record recent advances.
This plan of publication has been arranged in order to meet the
demand of teachers and the convenience of classes, but it is also
thought that it may prove advantageous to readers in special lines
of mathematical literature.

It is the intention of the publishers and editors to add other
monographs to the series from time to time, if the call for the same
seems to warrant it. Among the topics which are under consideration
are those of elliptic functions, the theory of numbers, the group
theory, the calculus of variations, and non-Euclidean geometry;
possibly also monographs on branches of astronomy, mechanics, and
mathematical physics may be included. It is the hope of the editors
that this form of publication may tend to promote mathematical study
and research over a wider field than that which the former volume
has occupied.

\smallskip \footnotesize December, 1905. \normalsize

\chapter{Author's Preface.}

This compendium of hyperbolic trigonometry was first published as a
chapter in Merriman and Woodward's Higher Mathematics. There is
reason to believe that it supplies a need, being adapted to two or
three different types of readers. College students who have had
elementary courses in trigonometry, analytic geometry, and
differential and integral calculus, and who wish to know something
of the hyperbolic trigonometry on account of its important and
historic relations to each of those branches, will, it is hoped,
find these relations presented in a simple and comprehensive way in
the first half of the work. Readers who have some interest in
imaginaries are then introduced to the more general trigonometry of
the complex plane, where the circular and hyperbolic functions merge
into one class of transcendents, the singly periodic functions,
having either a real or a pure imaginary period. For those who also
wish to view the subject in some of its practical relations,
numerous applications have been selected so as to illustrate the
various parts of the theory, and to show its use to the physicist
and engineer, appropriate numerical tables being supplied for these
purposes.

With all these things in mind, much thought has been given to the
mode of approaching the subject, and to the presentation of
fundamental notions, and it is hoped that some improvements are
discernible. For instance, it has been customary to define the
hyperbolic functions in relation to a sector of the rectangular
hyperbola, and to take the initial radius of the sector coincident
with the principal radius of the curve; in the present work, these
and similar restrictions are discarded in the interest of analogy
and generality, with a gain in symmetry and simplicity, and the
functions are defined as certain characteristic ratios belonging to
any sector of any hyperbola. Such definitions, in connection with
the fruitful notion of correspondence of points on conics, lead to
simple and general proofs of the addition-theorems, from which
easily follow the conversion-formulas, the derivatives, the
Maclaurin expansions, and the exponential expressions. The proofs
are so arranged as to apply equally to the circular functions,
regarded as the characteristic ratios belonging to any elliptic
sector. For those, however, who may wish to start with the
exponential expressions as the definitions of the hyperbolic
functions, the appropriate order of procedure is indicated on
page~\pageref{def hyper as exp}, and a direct mode of bringing such
exponential definitions into geometrical relation with the
hyperbolic sector is shown in the Appendix.

\enlargethispage*{1000pt}
\smallskip \footnotesize December, 1905. \normalsize

\tableofcontents
\listoftables

%% ART. 1. CORRESPONDENCE OF POINTS ON CONICS ...Page 7
%%      2. AREAS OF CORRESPONDING TRIANGLES ...9
%%      3. AREAS OF CORRESPONDING SECTORS ...9
%%      4. CHARACTERISTIC RATIOS OF SECTORIAL MEASURES ...10
%%      5. RATIOS EXPRESSED AS TRIANGLE-MEASURES ...10
%%      6. FUNCTIONAL RELATIONS FOR ELLIPSE ...11
%%      7. FUNCTIONAL RELATIONS FOR HYPERBOLA ...11
%%      8. RELATIONS BETWEEN HYPERBOLIC FUNCTIONS ...12
%%      9. VARIATIONS OF THE HYPERBOLIC FUNCTIONS ...14
%%     10. ANTI HYPERBOLIC FUNCTIONS ...16
%%     11. FUNCTIONS OF SUMS AND DIFFERENCES ...16
%%     12. CONVERSION FORMULAS ...18
%%     13. LIMITING RATIOS ...19
%%     14. DERIVATIVES OF HYPERBOLIC FUNCTIONS ...20
%%     15. DERIVATIVES OF ANTI-HYPERBOLIC FUNCTIONS ...22
%%     16. EXPANSION OF HYPERBOLIC FUNCTIONS ...23
%%     17. EXPONENTIAL EXPRESSIONS ...24
%%     18. EXPANSION OF ANTI-FUNCTIONS ...25
%%     19. LOGARITHMIC EXPRESSION OF ANTI-FUNCTIONS ...27
%%     20. THE GUDERMANIAN FUNCTION ...28
%%     21. CIRCULAR FUNCTIONS OF GUDERMANIAN ...28
%%     22. GUDERMANIAN ANGLE ...29
%%     23. DERIVATIVES OF GUDERMANIAN AND INVERSE ...30
%%     24. SERIES FOR GUDERMANIAN AND ITS INVERSE ...31
%%     25. GRAPHS OF HYPERBOLIC FUNCTIONS ...32
%%     26. ELEMENTARY INTEGRALS ...35
%%     27. FUNCTIONS OF COMPLEX NUMBERS ..38
%%     28. ADDITION THEOREMS FOR COMPLEXES ...40
%%     29. FUNCTIONS OF PURE IMAGINARIES ...41
%%     30. FUNCTIONS OF \emph{x + iy} IN THE FORM \emph{X + iY} ...43
%%     31. THE CATENARY ...47
%%     32. THE CATENARY OF UNIFORM STRENGTH ...49
%%     33. THE ELASTIC CATENARY ...50
%%     34. THE TRACTORY ...51
%%     35. THE LOXODROME ...52
%%     36 COMBINED FLEXURE AND TENSION ...53
%%     37. ALTERNATING CURRENTS ...55
%%     38. MISCELLANEOUS APPLICATIONS ...60
%%     39. EXPLANATION OF TABLES ...62
%%
%% TABLE I. HYPERBOLIC FUNCTIONS ...64
%%      II. VALUES OF \textsc{cosh}(\emph{x+iy})
%%             AND \textsc{sinh}(\emph{x+iy}) ...66
%%     III. VALUES OF gd\emph{u} AND $0^\circ$ ...70
%%      IV. VALUES OF gd\emph{u}, \textsc{log sinh} \emph{u},
%%             \textsc{log cosh} \emph{u} ...70
%%
%% APPENDIX. HISTORICAL AND BIBLIOGRAPHICAL ...71
%%           EXPONENTIAL EXPRESSIONS AS DEFINITIONS ...72
%%
%% INDEX ...73

\mainmatter
\chapter{Correspondence of Points on Conics.}%
\index{Corresponding points!on conics}%
\index{Geometrical treatment of hyperbolic functions|(}%
\index{Hyperbola|(}

To prepare the way for a general treatment of the hyperbolic
functions a preliminary discussion is given on the relations,
between hyperbolic sectors. The method adopted is such as to apply
at the same time to sectors of the ellipse, including the circle;
and the analogy of the hyperbolic and circular functions will be
obvious at every step, since the same set of equations can be read
in connection with either the hyperbola or the ellipse.\footnote{
The hyperbolic functions are not so named on account of any analogy
with what are termed Elliptic Functions. ``The elliptic integrals,
and thence the elliptic functions, derive their name from the early
attempts of mathematicians at the rectification of the
ellipse.\,\ldots To a certain extent this is a disadvantage; \ldots\
because we employ the name hyperbolic function to denote $\cosh u,
\sinh u$, etc., by analogy with which the elliptic functions would
be merely the circular functions $\cos \phi, \sin \phi$,
etc.\,\ldots'' (Greenhill, Elliptic Functions, p.\
175.)\label{fnp7}}\index{Greenhill's!Elliptic Functions} It is
convenient to begin with the theory of correspondence of points on
two central conics of like species, i.e.
either both ellipses or both hyperbolas.%
\index{Circular functions}\index{Elliptic!functions}%
\index{Elliptic!integrals}\index{Elliptic!sectors}

\begin{center}
\includegraphics[width=80mm]{fig01.png}
\end{center}

To obtain a definition of corresponding points, let $O_1A_1, O_1B_1$
be conjugate radii of a central conic, and $O_2A_2, O_2B_2$
conjugate radii of any other central conic of the same species; let
$P_1, P_2$ be two points on the curves; and let their coordinates
referred to the respective pairs of conjugate directions be $(x_1,
y_1), (x_2, y_2)$; then, by analytic geometry,
\begin{equation}
\frac{x_1^2}{a_1^2} \pm \frac{y_1^2}{b_1^2} = 1,\qquad
\frac{x_2^2}{a_2^2} \pm \frac{y_2^2}{b_2^2} = 1. \tag{1}
\end{equation}
Now if the points $P_1, P_2$ be so situated that
\begin{equation}
\frac{x_1}{a_1} = \frac{x_2}{a_2},\qquad
\frac{y_1}{b_1} = \frac{y_2}{b_2}, \tag{2}
\end{equation}
the equalities referring to sign as well as magnitude, then $P_1,
P_2$ are called corresponding points in the two systems. If $Q_1,
Q_2$ be another pair of correspondents, then the sector and triangle
$P_1O_1Q_1$ are said to correspond respectively with the sector and
triangle $P_2O_2Q_2$. These definitions will apply also when the
conies coincide, the points $P_1, P_2$ being then referred to any
two pairs of conjugate diameters of the same conic.

In discussing the relations between corresponding areas it is
convenient to adopt the following use of the word ``measure'': The
measure of any area connected with a given central conic is the
ratio which it bears to the constant area of the triangle formed by
two conjugate diameters of the same conic.%
\index{Areas}\index{Measure!defined}

\index{Measure!of sector|(}For example, the measure of the sector
$A_1O_1P_1$ is the ratio
\begin{equation*}
\frac{\text{sector }A_1O_1P_1}{\text{triangle }A_1O_1B_1}
\end{equation*}
and is to be regarded as positive or negative according as
$A_1O_1P_1$ and $A_1O_1B_1$ are at the same or opposite sides of
their common initial line.\index{Hyperbola|)}

\chapter{Areas of Corresponding Triangles.}%
\index{Areas}

The areas of corresponding triangles have equal measures. For, let
the coordinates of $P_1, Q_1$ be $(x_1, y_1), (x'_1, y'_1)$, and let
those of their correspondents $P_2, Q_2$ be $(x_2, y_2), (x'_2,
y'_2)$; let the triangles $P_1O_1Q_1, P_2O_2Q_2$ be $T_1, T_2$, and
let the measuring triangles $A_1O_1B_1, A_2O_2B_2$ be $K_1, K_2$,
and their angles $\omega_1, \omega_1$; then, by analytic geometry,
taking account of both magnitude and direction of angles, areas, and
lines,
\begin{gather*}
\begin{aligned}
\frac{T_1}{K_1} &=
     \frac{\frac{1}{2}(x_1y'_1 - x'_1y_1)\sin\omega_1}
          {\frac{1}{2} a_1b_1\sin\omega_1} =
     \frac{x_1}{a_1}  \cdot \frac{y'_1}{b_1} -
     \frac{x'_1}{a_1} \cdot \frac{y_1}{b_1}; \\
\frac{T_2}{K_2} &=
     \frac{\frac{1}{2}(x_2y'_2 - x'_2y_2)\sin\omega_2}
          {\frac{1}{2} a_2b_2\sin\omega_2} =
     \frac{x_2}{a_2}  \cdot \frac{y'_2}{b_2} -
     \frac{x'_2}{a_2} \cdot \frac{y_2}{b_2}.
\end{aligned} \\
\intertext{Therefore, by (2),}
\frac{T_1}{K_1} = \frac{T_2}{K_2}. \tag{3}
\end{gather*}\index{Corresponding points!on sectors and triangles}

\chapter{Areas of Corresponding Sectors.}\index{Sectors of conics}

The areas of corresponding sectors have equal measures. For conceive
the sectors $S_1, S_2$ divided up into infinitesimal corresponding
sectors; then the respective infinitesimal corresponding triangles
have equal measures (Art.~2); but the given sectors are the limits
of the sums of these infinitesimal triangles, hence
\begin{equation*}
\frac{S_1}{K_1} = \frac{S_2}{K_2}. \tag{4}
\end{equation*}

In particular, the sectors $A_1O_1P_1, A_2O_2P_2$ have equal
measures; for the initial points $A_1, A_2$ are corresponding
points.

It may be proved conversely by an obvious reductio ad absurdum that
if the initial points of two equal-measured sectors correspond, then
their terminal points correspond.

Thus if any radii $O_1A_1, O_2A_2$ be the initial lines of two
equal-measured sectors whose terminal radii are $O_1P_1, O_2P_2$,
then $P_1, P_2$ are corresponding points referred respectively to
the pairs of conjugate directions $O_1A_1, O_1B_1$, and $O_2A_2,
O_2A_B$; that is,
\begin{equation*}
\frac{x_1}{a_1} = \frac{x_2}{a_2},\quad
\frac{y_1}{b_1} = \frac{y_2}{b_2}.
\end{equation*}

\small \begin{enumerate}
\item[Prob.~1.] Prove that the sector $P_1O_1Q_1$, is bisected by the
line joining $O_1$, to the mid-point of $P_1Q_1$. (Refer the points
$P_1, Q_1$, respectively, to the median as common axis of $x$, and
to the two opposite conjugate directions as axis of $y$, and show
that $P_1, Q_1$ are then corresponding points.)

\item[Prob.~2.] Prove that the measure of a circular sector is equal
to the radian measure of its angle.

\item[Prob.~3.] Find the measure of an elliptic quadrant, and of the
sector included by conjugate radii.
\end{enumerate} \normalsize

\chapter{Charactersitic Ratios of Sectorial Measures.}

Let $A_1O_1P_1 = S_1$, be any sector of a central conic; draw
$P_1M_1$ ordinate to $O_1A_1$, i.e.\ parallel to the tangent at
$A_1$; let $O_1M_1 = x_1, M_1P_1 = y_1, O_1A_1 = a_1$, and the
conjugate radius $O_1B_1 = b_1$; then the ratios $\dfrac{x_1}{a_1},
\dfrac{y_1}{b_1}$ are called the characteristic ratios of the given
sectorial measure $\dfrac{S_1}{K_1}$. These ratios are constant both
in magnitude and sign for all sectors of the same measure and
species wherever these may be situated (Art.~3). Hence there exists
a functional relation between the sectorial measure and each of its
characteristic ratios.\label{sectoral measures}\index{Characteristic
ratios}\index{Ratios!characteristic}

\chapter{Ratios Expressed as Triangle-measures.}

The triangle of a sector and its complementary triangle are measured
by the two characteristic ratios. For, let the triangle $A_1O_1P_1$
and its complementary triangle $P_1O_1B_1$ be denoted by $T_1,
T'_1$; then
\begin{equation}
\left.
\begin{aligned}
\frac{T_1}{K_1} &=\frac{\frac{1}{2} a_1y_1 \sin\omega_1}
                       {\frac{1}{2} a_1b_1 \sin\omega_1} =\frac{y_1}{b_1},
\\
\frac{T'_1}{K_1}&=\frac{\frac{1}{2} b_1x_1 \sin\omega_1}
                       {\frac{1}{2} a_1b_1 \sin\omega_1} =\frac{x_1}{a_1}.
\end{aligned}
\right\} \tag{5}
\end{equation}\index{Complementary triangles}\index{Geometrical
treatment of hyperbolic functions|)}

\chapter{Functional Relations for Ellipse.}

\begin{center}
\includegraphics[width=60mm]{fig02.png}
\end{center}

The functional relations that exist between the sectorial measure
and each of its characteristic ratios are the same for all elliptic,
including circular, sectors (Art.~4). Let $P_1, P_2$ be
corresponding points on an ellipse and a circle, referred to the
conjugate directions $O_1A_1, O_1B_1$ and $O_2A_2, O_2B_2$, the
latter pair being at right angles; let the angle $A_2O_2P_2 =
\theta$ in radian measure; then
\begin{gather*}
\frac{S_2}{K_2} = \frac{\frac{1}{2} a_2^2\theta}{\frac{1}{2} a_2^2}
                = \theta. \tag{6} \\
\therefore \frac{x_2}{a_2} = \cos \frac{S_2}{K_2},  \quad
  \frac{y_2}{b_2} = \sin \frac{S_2}{K_2}; \qquad [ a_2 = b_2 \\
\intertext{hence, in the ellipse, by Art.~3,}
\frac{x_1}{a_1} = \cos \frac{S_1}{K_1},\quad \frac{y_1}{b_1} =
  \sin \frac{S_1}{K_1}. \tag{7}
\end{gather*}\index{Circular functions}

\small \begin{enumerate}
\item[Prob.~4.] Given $x_1 = \tfrac{1}{2} a_1$; find the measure
of the elliptic sector $A_1O_1P_1$. Also find its area when $a_1 =
4, b_1 = 3, \omega = 60^\circ$.

\item[Prob.~5.] Find the characteristic ratios of an elliptic
sector whose measure is $\frac{1}{4}\pi$.

\item[Prob.~6.] Write down the relation between an elliptic
sector and its triangle. (See Art.~5.)\index{Measure!of sector|)}
\end{enumerate} \normalsize

\chapter{Functional Relations for Hyperbola.}%
\index{Function!hyperbolic, defined}%
\index{Hyperbolic functions!defined}%
\index{Hyperbolic functions!relations among}%
\index{Relations among functions}

The functional relations between a sectorial measure and its
characteristic ratios in the case of the hyperbola may be written in
the form
\begin{equation*}
\frac{x_1}{a_1} = \cosh \frac{S_1}{K_1},\quad
\frac{y_1}{b_1} = \sinh \frac{S_1}{K_1},
\end{equation*}
and these express that the ratio of the two lines on the left is a
certain definite function of the ratio of the two areas on the
right. These functions are called by analogy the hyperbolic cosine
and the hyperbolic sine. Thus, writing $u$ for $\dfrac{S_1}{K_1}$
the two equations
\begin{equation*}
\frac{x_1}{a_1} = \cosh u,\quad \frac{y_1}{b_1} = \sinh u \tag{8}
\end{equation*}
serve to define the hyperbolic cosine and sine of a given sectorial
measure $u$; and the hyperbolic tangent, cotangent, secant, and
cosecant are then defined as follows:
\begin{equation}
\left.
\begin{aligned}
\tanh u = \frac{\sinh u}{\cosh u}, &\quad
\coth u = \frac{\cosh u}{\sinh u},\\
\sech u = \frac{   1   }{\cosh u}, &\quad
\csch u = \frac{   1   }{\sinh u}.
\end{aligned}
\right\} \tag{9}
\end{equation}

The names of these functions may be read ``h-cosine,'' ''h-sine,''
``h-tangent,'' etc., or ``hyper-cosine,'' etc.

\chapter{Relations Among Hyperbolic Functions.}

Among the six functions there are five independent relations, so
that when the numerical value of one of the functions is given, the
values of the other five can be found. Four of these relations
consist of the four defining equations (9). The fifth is derived
from the equation of the hyperbola
\begin{gather*}
\frac{x_1^2}{a_1^2} - \frac{y_1^2}{b_1^2} = 1,\\
\intertext{giving}
\cosh^2 u - \sinh^2 u = 1. \tag{10}
\end{gather*}

By a combination of some of these equations other subsidiary
relations may be obtained; thus, dividing (10) successively by
$\cosh^2 u, \sinh^2 u$, and applying (9), give
\begin{equation}
\left.
\begin{aligned}
1 - \tanh^2 u &= \sech^2 u, \\
\coth^2 u - 1 &= \csch^2 u.
\end{aligned}
\right\} \tag{11}
\end{equation}

Equations (9), (10), (11) will readily serve to express the value of
any function in terms of any other. For example, when $\tanh u$ is
given,
\begin{gather*}
\coth u = \frac{1}{\tanh u}, \quad \sech u = \sqrt{1 - \tanh^2 u}, \\
\cosh u = \frac{   1   }{\sqrt{1-\tanh^2 u}}, \quad
\sinh u = \frac{\tanh u}{\sqrt{1-\tanh^2 u}}, \\
\csch u = \frac{\sqrt{1-\tanh^2 u}}{\tanh u}.
\end{gather*}

The ambiguity in the sign of the square root may usually be removed
by the following considerations:%
\index{Ambiguity of value}\index{Multiple values} The functions
$\cosh u, \sech u$ are always positive, because the primary
characteristic ratio $\dfrac{x_1}{a_1}$ is positive, since the
initial line $O_1A_1$ and the abscissa $O_1M_1$ are similarly
directed from $O_1$ on whichever branch of the hyperbola $P_1$ maybe
situated; but the functions $\sinh u, \tanh u, \coth u, \csch u$,
involve the other characteristic ratio $\dfrac{y_1}{b_1}$, which is
positive or negative according as $y_1$ and $b_1$ have the same or
opposite signs, i.e., as the measure $u$ is positive or negative;
hence these four functions are either all positive or all negative.
Thus when any one of the functions $\sinh u, \tanh u, \csch u, \coth
u$, is given in magnitude and sign, there is no ambiguity in the
value of any of the six hyperbolic functions; but when either $\cosh
u$ or $\sech u$ is given, there is ambiguity as to whether the other
four functions shall be all positive or all negative.

\begin{center}
\includegraphics[width=50mm]{fig03.png}
\end{center}

The hyperbolic tangent may be expressed as the ratio of two lines.
For draw the tangent line $AC = t$; then
\begin{align*}
\tanh u &= \frac{y}{b} : \frac{x}{a} = \frac{a}b \cdot \frac{y}x \\
        &= \frac{a}{b} \cdot \frac{t}{a} = \frac{t}{b}. \tag{12}
\end{align*}

The hyperbolic tangent is the measure of the triangle $OAC$. For
\begin{gather}
\frac{OAC}{OAB} = \frac{at}{ab} = \frac{t}{b} = \tanh u. \tag{13}
\end{gather}

Thus the sector $AOP$, and the triangles $AOP, POB, AOC$, are
proportional to $u, \sinh u, \cosh u, \tanh u$ (eqs.\ 5, 13); hence
\begin{equation}
\sinh u > u > \tanh u. \tag{14}
\end{equation}

\small \begin{enumerate}
\item[Prob.~7.] Express all the hyperbolic functions in terms of
$\sinh{u}$. Given $\cosh{u} = 2$, find the values of the other
functions.

\item[Prob.~8.] Prove from eqs.\ 10, 11, that
$\cosh{u} > \sinh{u}, \cosh{u} > 1, \tanh{u} < 1, \sech{u} < 1$.

\item[Prob.~9.] In the figure of Art.~1, let $OA = 2, OB = 1,
AOB = 60^{\circ}$, and area of sector $AOP = 3$; find the sectorial
measure, and the two characteristic ratios, in the elliptic sector,
and also in the hyperbolic sector; and find the area of the triangle
$AOP$. (Use tables of cos, sin, cosh, sinh.)%
\index{Areas}

\item[Prob.~10.] Show that $\coth{u}, \sech{u}, \csch{u}$ may each
be expressed as the ratio of two lines, as follows: Let the tangent
at $P$ make on the conjugate axes $OA, OB$, intercepts $OS = m, OT =
n$; let the tangent at $B$, to the conjugate hyperbola, meet $OP$ in
$R$, making $BR = l$; then
\begin{equation*}
\coth{u} = \frac{l}{a},\quad \sech{u} = \frac{m}{a},\quad
  \csch{u} = \frac{n}{b}.
\end{equation*}

\item[Prob.~11.] The measure of segment $AMP$ is $\sinh{u}\cosh{u} -
u$. Modify this for the ellipse. Modify also eqs.\ 10--14, and
probs.\ 8, 10.
\end{enumerate} \normalsize

\chapter{Variations of the Hyperbolic Functions.}%
\index{Variation of hyperbolic functions}

\begin{center}
\includegraphics[width=40mm]{fig04.png}\label{ch9fig}
\end{center}

Since the values of the hyperbolic functions depend only on the
sectorial measure, it is convenient, in tracing their variations, to
consider only sectors of one half of a rectangular hyperbola, whose
conjugate radii are equal, and to take the principal axis $OA$ as
the common initial line of all the sectors. The sectorial measure
$u$ assumes every value from $-\infty$, through $0$, to $+\infty$,
as the terminal point $P$ comes in from infinity on the lower
branch, and passes to infinity on the upper branch; that is, as the
terminal line $OP$ swings from the lower asymptotic position $y =
-x$, to the upper one, $y = x$. It is here assumed, but is proved in
Art.~17, that the sector $AOP$ becomes infinite as $P$ passes to
infinity.

Since the functions $\cosh{u}, \sinh{u}, \tanh{u}$, for any position
of $OP$, are equal to the ratios of $x, y, t$, to the principal
radius $a$, it is evident from the figure that
\begin{equation}
\cosh 0 = 1,\quad \sinh 0 = 0,\quad \tanh 0 = 0, \tag{15}
\end{equation}
and that as $u$ increases towards positive infinity, $\cosh u, \sinh
u$ are positive and become infinite, but $\tanh u$ approaches unity
as a limit; thus
\begin{equation}
\cosh \infty = \infty,\quad
\sinh \infty = \infty,\quad
\tanh \infty = 1. \tag{16}
\end{equation}

Again, as $u$ changes from zero towards the negative side, $\cosh u$
is positive and increases from unity to infinity, but $\sinh u$ is
negative and increases numerically from zero to a negative infinite,
and $\tanh u$ is also negative and increases numerically from zero
to negative unity; hence
\begin{equation}
\cosh (-\infty) = \infty,\quad
\sinh (-\infty) = -\infty,\quad
\tanh (-\infty) = -1. \tag{17}
\end{equation}

For intermediate values of $u$ the numerical values of these
functions can be found from the formulas of Arts.\ 16, 17, and are
tabulated at the end of this chapter. A general idea of their manner
of variation can be obtained from the curves in Art.~25, in which
the sectorial measure $u$ is represented by the abscissa, and the
values of the functions $\cosh u$, $\sinh u$, etc., are represented
by the ordinate.

The relations between the functions of $-u$ and of $u$ are evident
from the definitions, as indicated above, and in Art.~8. Thus
\begin{equation}
\left.
\begin{aligned}
\cosh (-u) &= +\cosh u, &\quad \sinh (-u) &= -\sinh u, \\
\sech (-u) &= +\sech u, &\quad \csch (-u) &= -\csch u, \\
\tanh (-u) &= -\tanh u, &\quad \coth (-u) &= -\coth u. \\
\end{aligned}
\right\} \tag{18}
\end{equation}

\small \begin{enumerate}
\item[Prob.~12.] Trace the changes in $\sech u, \coth u, \csch u$,
as $u$ passes from $-\infty$ to $+\infty$. Show that $\sinh u, \cosh
u$ are infinites of the same order when $u$ is infinite. (It will
appear in Art.~17 that $\sinh u, \cosh u$ are infinites of an order
infinitely higher than the order of $u$.)

\item[Prob.~13.] Applying eq.\ (12) to figure,
page~\pageref{ch9fig}, prove $\tanh u_1 = \tan AOP$.
\end{enumerate} \normalsize

\chapter{Anti-hyperbolic Functions.}%
\index{Anti-hyperbolic functions}

The equations $\dfrac{x}{a} = \cosh u, \dfrac{y}{b} = \sinh u,
\dfrac{t}{b} = \tanh u$, etc., may also be expressed by the inverse
notation $u = \cosh^{-1}\dfrac{x}{a}, u = \sinh^{-1}\dfrac{y}{b}, u
= \tanh^{-1}\dfrac{t}{b}$, etc., which may be read: ``$u$ is the
sectorial measure whose hyperbolic cosine is the ratio $x$ to $a$,''
etc.; or ``$u$ is the anti-h-cosine of $\dfrac{x}{a}$,'' etc.

Since there are two values of $u$, with opposite signs, that
correspond to a given value of $\cosh u$, it follows that if $u$ be
determined from the equation $\cosh u = m$, where $m$ is a given
number greater than unity, $u$ is a two-valued function of $m$. The
symbol $\cosh^{-1} m$ will be used to denote the positive value of
$u$ that satisfies the equation $\cosh u = m$. Similarly the symbol
$\sech^{-1} m$ in will stand for the positive value of $u$ that
satisfies the equation $\sech u = m$. The signs of the other
functions $\sinh^{-1}m, \tanh^{-1}m, \coth^{-1}m, \csch^{-1}m$, are
the same as the sign of $m$. Hence all of the anti-hyperbolic
functions of real numbers are one-valued.%
\index{Ambiguity of value}\index{Multiple values}

\small \begin{enumerate}
\item[Prob.~14.] Prove the following relations:
\begin{equation*}
\cosh^{-1}m =     \sinh^{-1}\sqrt{m^2-1},\quad
\sinh^{-1}m = \pm \cosh^{-1}\sqrt{m^2+1},
\end{equation*}
the upper or lower sign being used according as $m$ is positive or
negative. Modify these relations for $\sin^{-1}, \cos^{-1}$.

\item[Prob.~15.] In figure, Art.~1, let $OA = 2, OB = 1, AOB =
60^\circ$; find the area of the hyperbolic sector $AOP$, and of the
segment $AMP$, if the abscissa of $P$ is 3. (Find $\cosh^{-1}$ from
the tables for $\cosh$.)
\end{enumerate} \normalsize

\chapter{Functions of Sums and Differences.}%
\index{Addition-theorems}\index{Functions!of sum and difference}%
\index{Geometrical treatment of hyperbolic functions}

(a) To prove the difference-formulas
\begin{equation}
\left.
\begin{aligned}
\sinh(u-v) &= \sinh u \cosh v - \cosh u \sinh v,  \\
\cosh(u-v) &= \cosh u \cosh v - \sinh u \sinh v.
\end{aligned}
\right\} \tag{19}
\end{equation}\index{Difference formula}%
\index{Hyperbolic funcitons!addition-theorems for}

\begin{center}
\includegraphics[width=80mm]{fig05.png}
\end{center}

Let $OA$ be any radius of a hyperbola, and let the sectors $AOP,
AOQ$ have the measures $u, v$; then $u-v$ is the measure of the
sector $QOP$. Let $OB, OQ'$ be the radii conjugate to $OA, OQ$; and
let the co�rdinates of $P, Q, Q'$ be $(x_1, y_1)$, $(x, y)$, $(x',
y')$ with reference to the axes $OA, OB$; then
\begin{align*}
\sinh (u-v) &=
  \sinh\frac{\text{sector }QOP}{K} =
       \frac{\text{triangle }QOP}{K}\quad \tag*{[Art.~5.} \\
&= \frac{\frac{1}{2}(xy_1 - x_1y)\sin\omega}
        {\frac{1}{2}a_1b_1\sin\omega}
    = \frac{y_1}{b_1}\cdot\frac{x}{a_1} -
        \frac{y}{b_1}\cdot\frac{x_1}{a_1} \notag \\
&= \sinh u \cosh v - \cosh u \sinh v; \notag
\end{align*}
\begin{align*}
\cosh(u-v) &= \cosh\frac{\text{sector }QOP}{K} =
              \frac{\text{triangle }POQ'}{K}\tag*{[Art.~5.} \\
&= \frac{\frac{1}{2}(x_1y' - y_1x')\sin\omega}
        {\frac{1}{2} a_1b_1\sin\omega}
= \frac{y'}{b_1}\cdot\frac{x_1}{a_1}
  - \frac{y}{b_1}\cdot\frac{x'}{a_1}; \\
\intertext{but}
\frac{y'}{b_1} &= \frac{x}{a_1}, \qquad
  \frac{x'}{a_1} = \frac{y}{b_1}, \tag{20}
\end{align*}
since $Q, Q'$ are extremities of conjugate radii; hence
\begin{equation*}
\cosh(u-v) = \cosh u \cosh v - \sinh u \sinh v.
\end{equation*}

In the figures $u$ is positive and $v$ is positive or negative.
Other figures may be drawn with $u$ negative, and the language in
the text will apply to all. In the case of elliptic sectors, similar
figures may be drawn, and the same language will apply, except that
the second equation of (20) will be $\dfrac{x'}{a_1} =
\dfrac{-y}{b_1}$; therefore
\begin{align*}
\sin(u-v) &= \sin u \cos v - \cos u \sin v,\\
\cos(u-v) &= \cos u \cos v + \sin u \sin v.
\end{align*}\index{Circular functions}

(b) To prove the sum-formulas
\begin{equation}
\left.
\begin{aligned}
\sinh(u + v) &= \sinh u \cosh v + \cosh u \sinh v,\\
\cosh(u + v) &= \cosh u \cosh v + \sinh u \sinh v.
\end{aligned}
\right\} \tag{21}
\end{equation}

These equations follow from (19) by changing $v$ into $-v$, and then
for $\sinh (-v)$, $\cosh (-v)$, writing $-\sinh v$, $\cosh v$ (Art.\
9, eqs.\ (18)).

\medskip (c) To prove that
\begin{equation}
\tanh (u \pm v) = \frac{\tanh u \pm \tanh v}
                       {1 \pm \tanh u \tanh v}. \tag{22}
\end{equation}

Writing $\tanh (u \pm v) = \dfrac{\sinh(u \pm v)}{\cosh(u \pm v)}$,
expanding and dividing numerator and denominator by $\cosh u \cosh
v$, eq.\ (22) is obtained.

\small \begin{enumerate}

\item[Prob.~16.] Given $\cosh u = 2, \cosh v = 3$, find
$\cosh(u + v)$.

\item[Prob.~17.] Prove the following identities:
\begin{enumerate}
\item $\sinh 2u = 2 \sinh u \cosh u$.
\item $\cosh 2u = \cosh^2 u + \sinh^2 u = 1 + 2 \sinh^2 u
                = 2 \cosh^2 u - 1$.
\item $1 + \cosh u = 2 \cosh^2 \frac{1}{2}u,
       \cosh u - 1 = 2 \sinh^2 \frac{1}{2}u$.
\item $\tanh \frac{1}{2}u = \dfrac{\sinh u}{1 + \cosh u}
         = \dfrac{\cosh u - 1}{\sinh u}
         = \left( \dfrac{\cosh u - 1}{\cosh u + 1}\right)
               ^{\frac{1}{2}}$.
\item $\sinh 2u = \dfrac{2\tanh u}{1 - \tanh^2 u},\
       \cosh 2u = \dfrac{1 + \tanh^2 u}{1 - \tanh^2 u}$.
\item $\sinh 3u = 3 \sinh u + 4 \sinh^3 u,\
       \cosh 3u = 4 \cosh^3u - 3 \cosh u$.
\item $\cosh u + \sinh u = \dfrac{1 + \tanh\frac{1}{2}u}
                                 {1 - \tanh\frac{1}{2}u}$.
\item $(\cosh u + \sinh u)(\cosh v + \sinh v)
         = \cosh (u+v) + \sinh (u+v)$.
\item Generalize (h); and show also what it becomes when
$u = v = \ldots$
\item $\sinh^2 x \cos^2 y + \cosh^2 x \sin^2 y
        = \sinh^2 x + \sin^2 y$.
\item $\cosh^{-1}m \pm \cosh^{-1}n =
  \cosh^{-1}\left[ mn \pm\sqrt{(m^2-1)(n^2-1)} \right]$.
\item $\sinh^{-1}m \pm \sinh^{-1}n =
  \sinh^{-1}\left[m\sqrt{1+n^2}\pm n\sqrt{1+m^2}\right]$.
\end{enumerate}

\item[Prob.~18.] What modifications of signs are required in (21),
(22), in order to pass to circular functions?

\item[Prob.~19.] Modify the identities of Prob. 17 for the same
purpose.
\end{enumerate} \normalsize

\chapter{Conversion Formulas.}\index{Conversion-formulas}

To prove that
\begin{equation}
\left.
\begin{aligned}
\cosh u_1 + \cosh u_2 &= 2\cosh\tfrac{1}{2}(u_1+u_2)
    \cosh\tfrac{1}{2}(u_1-u_2), \\
\cosh u_1 - \cosh u_2 &= 2\sinh\tfrac{1}{2}(u_1+u_2)
    \sinh\tfrac{1}{2}(u_1-u_2), \\
\sinh u_1 + \sinh u_2 &= 2\sinh\tfrac{1}{2}(u_1+u_2)
    \cosh\tfrac{1}{2}(u_1-u_2), \\
\sinh u_1 - \sinh u_2 &= 2\cosh\tfrac{1}{2}(u_1+u_2)
    \sinh\tfrac{1}{2}(u_1-u_2).
\end{aligned}
\right\} \tag{23}
\end{equation}
From the addition formulas it follows that
\begin{align*}
\cosh (u+v) + \cosh (u-v) &= 2 \cosh u \cosh v, \\
\cosh (u+v) - \cosh (u-v) &= 2 \sinh u \sinh v, \\
\sinh (u+v) + \sinh (u-v) &= 2 \sinh u \cosh v, \\
\sinh (u+v) - \sinh (u-v) &= 2 \cosh u \sinh v,
\end{align*}
and then by writing $u + v = u_1$, $u-v = u_2$, $u = \frac{1}{2}(u_1
+ u_2)$, $v = \frac{1}{2}(u_1 - u_2)$, these equations take the form
required.

\small \begin{enumerate}
\item[Prob.~20.] In passing to circular functions, show that the
only modification to be made in the conversion formulas is in the
algebraic sign of the right-hand member of the second formula.

\item[Prob.~21.] Simplify $\dfrac{\cosh 2u + \cosh 4v}{\sinh 2u + \sinh
4v}$, $\dfrac{\cosh 2u + \cosh 4v}{\cosh 2u - \cosh 4v}$.

\item[Prob.~22.] Prove $\sinh^2 x - \sinh^2 y = \sinh (x+y) \sinh
(x-y)$.

\item[Prob.~23.] Simplify $\cosh^2 x \cosh^2 y \pm \sinh^2 x \sinh^2
y$.

\item[Prob.~24.] Simplify $\cosh^2 x \cos^2 y + \sinh^2 x \sin^2 y$.
\end{enumerate} \normalsize

\chapter{Limiting Ratios.}\index{Limiting ratios}%
\index{Ratios!limiting}

To find the limit, as $u$ approaches zero, of
\begin{equation*}
\frac{\sinh u}{u}, \frac{\tanh u}{u},
\end{equation*} which are then indeterminate in form.

By eq.\ (14), $\sinh u > u > \tanh u$; and if $\sinh u$ and $\tanh
u$ be successively divided by each term of these inequalities, it
follows that
\begin{gather*}
1 < \frac{\sinh u}{u} < \cosh u ,\\
\sech u < \frac{\tanh u}{u} <1,
\end{gather*}
but when $u \doteq 0$, $\cosh u \doteq 1$, $\sech u \doteq 1$, hence
\begin{equation}
\lim_{u\doteq 0} \frac{\sinh u}{u} = 1,
  \lim_{u\doteq 0} \frac{\tanh u}{u} = 1. \tag{24}
\end{equation}

\chapter{Derivatives of Hyperbolic Functions.}%
\index{Derived functions}\index{Hyperbolic functions!derivatives of}%
\index{Hyperbolic functions!variation of}

To prove that
\begin{equation}
\left.
\begin{aligned}
(\textit{a}) && \frac{d(\sinh u)}{du} &= \cosh u, \\
(\textit{b}) && \frac{d(\cosh u)}{du} &= \sinh u, \\
(\textit{c}) && \frac{d(\tanh u)}{du} &= \sech^2 u, \\
(\textit{d}) && \frac{d(\sech u)}{du} &= -\sech u\; \tanh u, \\
(\textit{e}) && \frac{d(\coth u)}{du} &= -\csch^2 u, \\
(\textit{f}) && \frac{d(\csch u)}{du} &= -\csch u\; \coth u, \\
\end{aligned} \right\} \tag{25}
\end{equation}

\begin{enumerate}
\item[(a)] Let
\begin{align*}
y &= \sinh u, \\
\Delta y &= \sinh \left( {u + \Delta u} \right) - \sinh u \\
  &= 2\cosh \frac{1}{2}\left( {2u + \Delta u} \right)
                                        \sinh \frac{1}{2}\Delta u, \\
\frac{\Delta y}{\Delta u} &=
  \cosh \left( {u + \frac{1}{2}\Delta u} \right)
  \frac{\sinh \frac{1}{2}\Delta u}{\frac{1}{2}\Delta u}. \\
\intertext{Take the limit of both sides, as $\Delta u \doteq 0$, and
put}
\limdot &\frac{\Delta y}{\Delta u} = \frac{dy}{du} =
   \frac{d\left( {\sinh u} \right)}{du}, \\
\limdot &\cosh \left( {u + \frac{1}{2}\Delta u} \right) = \cosh u, \\
\limdot &\frac{\sinh \frac{1}{2}\Delta u}
              {\frac{1}{2}\Delta u} = 1; \tag{see Art. 13} \\
\intertext{then }
&\frac{d\left( {\sinh u} \right)}{du} = \cosh u. \\
\end{align*}

\item[(b)] Similar to (a).

\item[(c)] \begin{align*}
 \frac{d\left( {\tanh u} \right)}{du} &= \frac{d}{du} \cdot
     \frac{\sinh u}{\cosh u} \\
 &= \frac{\cosh ^2 u - \sinh ^2 u}{\cosh ^2 u} =
    \frac{1}{\cosh ^2 u} = \sech^{2} u.
\end{align*}

\item[(d)] Similar to (c).

\item[(e)] \begin{equation*}
\frac{d(\sech u)}{du} = \frac{d}{du} \cdot \frac{1}{\cosh u}
  = -\frac{\sinh u}{\cosh^2 u} = -\sech u \tanh u.
\end{equation*}

\item[(f)] Similar to (e).
\end{enumerate}

It thus appears that the functions $\sinh u, \cosh u$ reproduce
themselves in two differentiations; and, similarly, that the
circular functions $\sin u, \cos u$ produce their opposites in two
differentiations. In this connection it may be noted that the
frequent appearance of the hyperbolic (and circular) functions in
the solution of physical problems is chiefly due to the fact that
they answer the question: What function has its second derivative
equal to a positive (or negative) constant multiple of the function
itself? (See Probs.\ 28--30.) An answer such as $y = \cosh mx$ is
not, however, to be understood as asserting that $mx$ is an actual
sectorial measure and $y$ its characteristic ratio; but only that
the relation between the numbers $mx$ and $y$ is the same as the
known relation between the measure of a hyperbolic sector and its
characteristic ratio; and that the numerical value of $y$ could be
found from a table of hyperbolic cosines.

\small \begin{enumerate}
\item[Prob.~25.] Show that for circular functions the only
modifications required are in the algebraic signs of (b), (d).

\item[Prob.~26.] Show from their derivatives which of the
hyperbolic and circular functions diminish as $u$ increases.

\item[Prob.~27.] Find the derivative of $\tanh u$ independently
of the derivatives of $\sinh u$, $\cosh u$.

\item[Prob.~28.] Eliminate the constants by differentiation from
the equation
\begin{equation*}
y = A \cosh mx + B \sinh mx,
\end{equation*} and prove that $\dfrac{d^2y}{dx^2} = m^2y.$

\item[Prob.~29.] Eliminate the constants from the equation
\begin{equation*}
y = A \cos mx + B \sin mx,
\end{equation*}
and prove that $\dfrac{d^2y}{dx^2} = -m^2y.$%
\index{Circular functions}\index{Elimination of constants}

\item[Prob.~30.] Write down the most general solutions of the
differential equations
\begin{equation*}
\frac{d^2y}{dx^2} = m^2y, \quad
\frac{d^2y}{dx^2} = -m^2y, \quad
\frac{d^4y}{dx^4} = m^4y.
\end{equation*}\index{Differential equation}
\end{enumerate} \normalsize

\chapter{Derivatives of Anti-hyperbolic Functions.}%
\index{Anti-hyperbolic functions}\index{Derived functions}

\begin{equation}
\left.
\begin{aligned}
(\textit{a}) && \frac{d(\sinh^{-1} x)}{dx} &=
                                           \frac{1}{\sqrt{x^2+1}}, \\
(\textit{b}) && \frac{d(\cosh^{-1} x)}{dx} &=
                                           \frac{1}{\sqrt{x^2-1}}, \\
(\textit{c}) && \frac{d(\tanh^{-1} x)}{dx} &=
                             \left. \frac{1}{1-x^2} \right]_{x<1}, \\
(\textit{d}) && \frac{d(\coth^{-1} x)}{dx} &=
                             \left. \frac{1}{1-x^2} \right]_{x>1}, \\
(\textit{e}) && \frac{d(\sech^{-1} x)}{dx} &=
                                         -\frac{1}{x\sqrt{1-x^2}}, \\
(\textit{f}) && \frac{d(\csch^{-1} x)}{dx} &=
                                         -\frac{1}{x\sqrt{x^2+1}}, \\
\end{aligned}
\right\} \tag{26}
\end{equation}

\begin{enumerate}
\item[(a)] Let $u = \sinh^{-1} x$, then $x = \sinh u$, $dx =
\cosh u\,du = \sqrt{1 + \sinh^2 u} = \sqrt{1 + x^2} du$, $du =
\dfrac{dx}{\sqrt{1 + x^2}}$.

\item[(b)] Similar to (a).

\item[(c)] Let $u = \tanh^{-1} x$, then $x = \tanh u$, $dx =
\sech^2 u\,du = (1 - \tanh^2 u)du = (1 - x^2)du$, $du = \dfrac{dx}{1
- x^2}$.

\item[(d)] Similar to (c).

\item[(e)]
\begin{equation*}
\frac{d(\sech^{-1} x)}{dx}
  = \frac{d}{dx}\left( \cosh^{-1} \frac{1}{x} \right)
  = \frac{\frac{-1}{x^2}}{\left( \frac{1}{x^2} - 1 \right)^{\frac{1}{2}}}
  = \frac{-1}{x\sqrt{1-x^2}}.
\end{equation*}

\item[(f)] Similar to (e).
\end{enumerate}

\small \begin{enumerate}
\item[Prob. 31.] Prove
\begin{align*}
  \frac{d(\sin^{-1} x)}{dx} &=  \frac{1}{\sqrt{1 - x^2}},
& \frac{d(\cos^{-1} x)}{dx} &= -\frac{1}{\sqrt{1 - x^2}}, \\
  \frac{d(\tan^{-1} x)}{dx} &=  \frac{1}{1 + x^2},
& \frac{d(\cot^{-1} x)}{dx} &= -\frac{1}{1 + x^2} .
\end{align*}

\item[Prob.~32.] Prove
\begin{align*}
d\sinh^{-1}\frac{x}a &= \frac{dx}{\sqrt{x^2+a^2}}, &
d\cosh^{-1}\frac{x}a &= \frac{dx}{\sqrt{x^2-a^2}}, \\
d\tanh^{-1}\frac{x}a &=  \left.\frac{adx}{a^2-x^2}\right]_{x<a}, &
d\coth^{-1}\frac{x}a &= -\left.\frac{adx}{x^2-a^2}\right]_{x>a}.
\end{align*}

\item[Prob.~33.] Find $d(\sech^{-1} x)$ independently of $\cosh^{-1}
x$.

\item[Prob.~34.] When $\tanh^{-1} x$ is real, prove that $\coth^{-1} x$
is imaginary, and conversely; except when $x = 1$.

\item[Prob.~35.] Evaluate $\dfrac{\sinh^{-1}x}{\log x}$,
$\dfrac{\cosh^{-1}x}{\log x}$ when $x = \infty$.
\end{enumerate} \normalsize

\chapter{Expansion of Hyperbolic Functions.}%
\index{Hyperbolic functions!expansions of}\index{Limiting ratios}%
\index{Series}

For this purpose take Maclaurin's Theorem,
\begin{gather*}
f(u) = f(0) + uf'(0) + \frac{1}{2!}u^2f''(0)
       + \frac{1}{3!}u^3f'''(0) + \ldots, \\
\intertext{and put}
f(u) = \sinh u,\quad f'(u) = \cosh u,\quad
  f''(u) = \sinh u, \ldots, \\
\intertext{then}
f(0) = \sinh 0 = 0,\quad f'(0) = \cosh 0 = 1, \ldots; \\
\intertext{hence}
\sinh u = u + \frac{1}{3!}u^3 + \frac{1}{5!}u^5 + \ldots; \tag{27}
\intertext{and similarly, or by differentiation,}
\cosh u = 1 + \frac{1}{2!}u^2 + \frac{1}{4!}u^4 + \ldots. \tag{28}
\end{gather*}\index{Expansion in series}

By means of these series the numerical values of $\sinh u, \cosh u$,
can be computed and tabulated for successive values of the
independent variable $u$. They are convergent for all values of $u$,
because the ratio of the $n$th term to the preceding is in the first
case $\dfrac{u^2}{(2n-1)(2n-2)}$, and in the second case
$\dfrac{u^2}{(2n-2)(2n-3)}$, both of which ratios can be made less
than unity by taking $n$ large enough, no matter what value $u$ has.
Lagrange's remainder shows equivalence of function and series.%
\index{Convergence}

From these series the following can be obtained by division:
\begin{equation}
\left.
\begin{aligned}
 \tanh u &= u - \frac{1}{3} u^3 + \frac{2}{ 15} u^5 +
                                    \frac{17}{  315} u^7 + \ldots, \\
 \sech u &= 1 - \frac{1}{2} u^2 + \frac{5}{ 24} u^4 -
                                    \frac{61}{  720} u^6 + \ldots, \\
u\coth u &= 1 + \frac{1}{3} u^2 - \frac{1}{ 45} u^4 +
                                    \frac{ 2}{  945} u^6 - \ldots, \\
u\csch u &= 1 - \frac{1}{6} u^2 + \frac{7}{360} u^4 -
                                    \frac{31}{15120} u^6 + \ldots.
\end{aligned}
\right\} \tag{29}
\end{equation}

These four developments are seldom used, as there is no observable
law in the coefficients, and as the functions $\tanh u, \sech u,
\coth u, \csch u$, can be found directly from the previously
computed values of $\cosh u, \sinh u$.

\small \begin{enumerate}
\item[Prob. 36.] Show that these six developments can be adapted to
the circular functions by changing the alternate signs.
\end{enumerate} \normalsize

\chapter{Exponential Expressions.}\index{Exponential expressions}%
\index{Hyperbolic functions!exponential functions for}

Adding and subtracting (27), (28) give the identities
\begin{gather*}
\begin{aligned}
\cosh u + \sinh u &= 1 + u + \frac{1}{2!} u^2 + \frac{1}{3!} u^3
                             + \frac{1}{4!} u^4 + \ldots = e^u,    \\
\cosh u - \sinh u &= 1 - u + \frac{1}{2!} u^2 - \frac{1}{3!} u^3
                             + \frac{1}{4!} u^4 - \ldots = e^{-u},
\end{aligned}
\intertext{hence}
\left.
\begin{aligned}
\cosh u &= \tfrac{1}{2}(e^u + e^{-u}), &
\sinh u &= \tfrac{1}{2}(e^u - e^{-u}), \\
\tanh u &= \frac{e^u - e^{-u}}{e^u + e^{-u}}, & \sech u &=
\frac{2}{e^u + e^{-u}},\quad\text{etc.}
\end{aligned}
\right\} \tag{30}
\end{gather*}

The analogous exponential expressions for $\sin u, \cos u$ are
\begin{equation*}
\cos u = \frac{1}{2} (e^{ui} + e^{-ui}),\quad
\sin u = \frac{1}{2i}(e^{u} - e^{-ui}),\quad (i=\sqrt{-1})
\end{equation*}
where the symbol $e^{ui}$ stands for the result of substituting $ui$
for $x$ in the exponential development
\begin{equation*}
e^x = 1 + x + \frac{1}{2!} x^2 + \frac{1}{3!} x^3 + \ldots
\end{equation*}\index{Circular functions}

This will be more fully explained in treating of complex numbers,
Arts.~28, 29.

\small \begin{enumerate}
\item[Prob.~37.] Show that the properties of the hyperbolic functions
could be placed on a purely algebraic basis by starting with
equations (30) as their definitions; for example, verify the
identities:\label{def hyper as exp}
\begin{gather*}
\sinh (-u) = -\sinh u,\quad \cosh (-u) = \cosh u,\\
\cosh^2 u - \sinh^2 u = 1,\quad
  \sinh (u+v) = \sinh u \cosh v + \cosh u \sinh v,\\
\frac{d^2(\cosh mu)}{du^2} = m^2 \cosh mu,\quad
  \frac{d^2(\sinh mu)}{du^2} = m^2 \sinh mu.
\end{gather*}

\item[Prob.~38.] Prove $(\cosh u + \sinh u)^n = \cosh nu + \sinh nu$.

\item[Prob.~39.] Assuming from Art.~14 that $\cosh u$, $\sinh u$
satisfy the differential equation $\dfrac{d^2y}{du^2} = y$, whose
general solution may be written $y = Ae^n + Be^{-n}$, where $A$, $B$
are arbitrary constants; show how to determine $A$, $B$ in order to
derive the expressions for $\cosh u$, $\sinh u$, respectively. [Use
eq.\ (15).]\index{Differential equation}

\item[Prob.~40.] Show how to construct a table of exponential functions
from a table of hyperbolic sines and cosines, and \emph{vice versa}.

\item[Prob.~41.] Prove $u = \log_e (\cosh u + \sinh u)$.

\item[Prob.~42.] Show that the area of any hyperbolic sector is
infinite when its terminal line is one of the asymptotes.

\item[Prob.~43.] From the relation $2 \cosh u = e^n + e^{-n}$ prove
\begin{equation*}
2^{n-1}(\cosh u)^n = \cosh nu + n \cosh (n-2)u +
  \tfrac{1}{2} n(n-1)\cosh (n-4)u + \ldots,
\end{equation*}
and examine the last term when $n$ is odd or even. Find also the
corresponding expression for $2^{n-1} (\sinh u)^n$.
\end{enumerate}\index{Exponential expressions} \normalsize

\chapter{Expansion of Anti-functions.}%
\index{Anti-hyperbolic functions}

Since
\begin{gather*}
\begin{aligned}
\frac{d(\sinh^{-1} x)}{dx} &= \frac{1}{\sqrt{1+x^2}}
 = (1+x^2)^{-\frac{1}{2}} \\
&= 1 - \frac{1}{2}\cdot x^2 + \frac{1}{2}\cdot\frac{3}{4}\cdot x^4
     - \frac{1}{2}\cdot\frac{3}{4}\cdot\frac{5}{6}\cdot x^6
                                                          + \ldots,\\
\end{aligned}
\intertext{hence, by integration,}
\sinh^{-1} x = x - \frac{1}{2}\cdot \frac{x^3}{3} +
   \frac{1}{2}\cdot\frac{3}{4}\cdot\frac{x^5}{5} -
   \frac{1}{2}\cdot\frac{3}{4}\cdot\frac{5}{6}\cdot\frac{x^7}{7}
                                                + \ldots, \tag{31}
\end{gather*}
the integration-constant being zero, since $\sinh^{-1} x$ vanishes
with $x$. This series is convergent, and can be used in computation,
only when $x < 1$.  Another series, convergent when $x > 1$, is
obtained by writing the above derivative in the form
\begin{align*}
\frac{d(\sinh^{-1} x)}{dx} &= (x^2+1)^{-\frac{1}{2}} =
  \frac{1}{x} \left(1 + \frac{1}{x^2}\right)^{-\frac{1}{2}} \\
&= \frac{1}{x} \left[ 1 - \frac{1}{2}\cdot\frac{1}{x^2} +
   \frac{1}{2}\cdot\frac{3}{4}\cdot\frac{1}{x^4} -
   \frac{1}{2}\cdot\frac{3}{4}\cdot\frac{5}{6}\cdot \frac{1}{x^6}
                                                 + \dotsb \right], \\
\therefore\; \sinh^{-1} &= C + \log x +
   \frac{1}{2}\cdot\frac{1}{2x^2} -
   \frac{1}{2}\cdot\frac{3}{4}\cdot\frac{1}{4x^4} +
   \frac{1}{2}\cdot\frac{3}{4}\cdot\frac{5}{6}\cdot\frac{1}{6x^6}
                                                   - \dotsb, \tag{32}
\end{align*}
where $C$ is the integration-constant, which will be shown in
Art.~19 to be equal to $\log_e 2$.\index{Convergence}%
\index{Expansion in series}

A development of similar form is obtained for $\cosh^{-1} x$; for
\begin{gather*}
\begin{aligned}
\frac{d(\cosh^{-1} x)}{dx} &= (x^2-1)^{-\frac{1}{2}} =
  \frac{1}{x}\left(1-\frac{1}{x^2}\right)^{-\frac{1}{2}} \\
&= \frac{1}{x} \left[ 1 + \frac{1}{2}\cdot\frac{1}{x^2} +
   \frac{1}{2}\cdot\frac{3}{4}\cdot\frac{1}{x^4} +
   \frac{1}{2}\cdot\frac{3}{4}\cdot\frac{5}{6}\cdot\frac{1}{x^6}
                                                    + \dotsb \right],
\end{aligned} \\
\intertext{hence}
\cosh^{-1} x = C + \log x - \frac{1}{2}\cdot \frac{1}{2x^2} -
   \frac{1}{2}\cdot\frac{3}{4}\cdot\frac{1}{4x^4} -
   \frac{1}{2}\cdot\frac{3}{4}\cdot\frac{5}{6}\cdot \frac{1}{6x^6}
                                                   - \dotsb, \tag{33}
\end{gather*}
in which $C$ is again equal to $\log_e 2$ [Art.~19, Prob.~46]. In
order that the function $\cosh^{-1} x$ may be real, $x$ must not be
less than unity; but when $x$ exceeds unity, this series is
convergent, hence it is always available for computation.

Again
\begin{align*}
\frac{d(\tanh^{-1} x)}{dx} &= \frac{1}{1-x^2}
   = 1 + x^2 + x^4 + x^6 + \dotsb, \\
\intertext{and hence}
\tanh^{-1} x &= x + \frac{1}{3} x^3 +
  \frac{1}{5} x^5 + \frac{1}{7} x^7 + \dotsb, \tag{34}
\end{align*}

From (32), (33), (34) are derived:
\begin{align*}
\sech^{-1} x &= \cosh^{-1} \frac{1}{x} \\
&= C - \log x - \frac{x^2}{2 \cdot 2} -
   \frac{1 \cdot 3 \cdot x^4}{2 \cdot 4\cdot 4}
   - \frac{1 \cdot 3\cdot 5 \cdot x^6}
          {2 \cdot 4\cdot 6\cdot 6 } - \dotsb; \tag{35} \\
\csch^{-1} x &= \sinh^{-1} \frac{1}{x} =
   \frac{1}{x} - \frac{1}{2}\cdot\frac{1}{3x^3} +
   \frac{1}{2}\cdot\frac{3}{4}\cdot\frac{1}{5x^5} -
   \frac{1}{2}\cdot\frac{3}{4}\cdot\frac{5}{6}\cdot\frac{1}{7x^7} +
   \dotsb, \\
&= C - \log x + \frac{x^2}{2 \cdot 2} -
   \frac{1 \cdot 3 \cdot x^4} {2 \cdot 4 \cdot 4} +
   \frac{1 \cdot 3 \cdot 5 \cdot x^6}
        {2 \cdot 4 \cdot 6 \cdot 6} - \dotsb; \tag{36} \\
\coth^{-1} x &= \tanh^{-1} \frac{1}{x} =
   \frac{1}{x} + \frac{1}{3x^3} + \frac{1}{5x^5} +
   \frac{1}{7x^7} + \dotsb. \tag{37}
\end{align*}

\small \begin{enumerate}
\item[Prob.~44.] Show that the series for $\tanh^{-1} x, \coth^{-1}
x, \sech^{-1} x$, are always available for computation.

\item[Prob.~45.] Show that one or other of the two developments of the
inverse hyperbolic cosecant is available.
\end{enumerate} \normalsize

\chapter{Logarithmic Expression of Anti-Functions.}%
\index{Logarithmic!expressions}

Let
\begin{align*}
x &= \cosh u, \\
\intertext{then}
\sqrt{x^2 - 1} &= \sinh u; \\
\intertext{therefore}
x + \sqrt{x^2 - 1} &= \cosh u + \sinh u = e^u, \\
\intertext{and}
u = \cosh^{-1} x &= \log{\left(x + \sqrt{x^2 - 1}\right)}. \tag{38} \\
\intertext{Similarly,}
\sinh^{-1} x &= \log{\left(x + \sqrt{x^2 + 1}\right)}. \tag{39} \\
\intertext{Also}
\sech^{-1} x &= \cosh^{-1}\frac{1}{x} =
  \log{\frac{1 + \sqrt{1 - x^2}}{x}}, \tag{40} \\
\csch^{-1} x &= \sinh^{-1}\frac{1}{x} =
  \log{\frac{1 + \sqrt{1 + x^2}}{x}}. \tag{41} \\
\intertext{Again, let}
x &= \tanh u = \frac{e^u - e^{-u}}{e^u + e^{-u}}, \\
\intertext{therefore}
\frac{1 + x}{1 - x} &= \frac{e^u}{e^{-u}} = e^{2u}, \\
2u &= \log{\frac{1 + x}{1 - x}}, \quad
  \tanh^{-1} = \tfrac{1}{2}\log{\frac{1 + x}{1 - x}}; \tag{42} \\
\intertext{and}
\coth^{-1} x &= \tanh^{-1} \frac{1}{x} =
  \tfrac{1}{2}\log{\frac{x + 1}{x - 1}}. \tag{43}
\end{align*}

\small \begin{enumerate}
\item[Prob.~46.] Show from (38), (39), that, when $x \doteq \infty$,
\begin{equation*}
\sinh^{-1} x - \log x \doteq \log 2,\qquad
\cosh^{-1} x - \log x \doteq \log 2,
\end{equation*}
and hence show that the integration-constants in (32), (33) are each
equal to $\log 2$.

\item[Prob.~47.] Derive from (42) the series for $\tanh^{-1}x$ given in
(34).

\item[Prob.~48.] Prove the identities:
\begin{align*}
& \log x = 2\tanh^{-1} \frac{x - 1}{x + 1}
= \tanh^{-1} \frac{x^2 - 1}{x^2 + 1}
= \sinh^{-1} \tfrac{1}{2} (x - x^{-1})
= \cosh^{-1} \tfrac{1}{2} (x + x^{-1}); \\
& \log\sec x = 2\tanh^{-1} \tfrac{1}{2} x;\
  \log\csc x = 2\tanh^{-1}
                \tan^2\left(\frac{1}{4}\pi + \frac{1}{2}x\right); \\
& \log\tan x = -\tanh^{-1} \cos 2x
= -\sinh^{-1} \cot 2x = \cosh^{-1} \csc 2x.
\end{align*}
\end{enumerate} \normalsize

\chapter{The Gudermanian Function.}\label{gudermanian}%
\label{Gudermanian!function}

The correspondence of sectors of the same species was discussed in
Arts.~1--4. It is now convenient to treat of the correspondence that
may exist between sectors of different species.

Two points $P_1, P_2$, on any hyperbola and ellipse, are said to
correspond with reference to two pairs of conjugates $O_1 A_1, O_1
B_1$, and $O_2 A_2, O_2 B_2$, respectively, when
\begin{equation}
\frac{x_1}{a_1} = \frac{a_2}{x_2} ,\tag{44}
\end{equation}
and when $y_1, y_2$ have the same sign. The sectors $A_1 O_1 P_1,
A_2 O_2 P_2$ are then also said to correspond. Thus corresponding
sectors of central conics of different species are of the same sign
and have their primary characteristic ratios reciprocal. Hence there
is a fixed functional relation between their respective measures.
The elliptic sectorial measure is called the gudermanian of the
corresponding hyperbolic sectorial measure, and the latter the
anti-gudermanian of the former. This relation is expressed by
\begin{gather*}
\frac{S_2}{K_2} = \gd \frac{S_1}{K_1}  \\
  \text{or } v = \gd u, \text{ and } u = \gd^{-1} v. \tag{45}
\end{gather*}%
\index{Anti-gudermanian}\index{Corresponding points!on conics}%
\index{Corresponding points!on sectors and triangles}%
\index{Sectors of conics}

\chapter{Circular Functions of Gudermanian.}%
\index{Circular functions!of gudermanian}%
\index{Hyperbolic functions!relations to circular functions}

The six hyperbolic functions of $u$ are expressible in terms of the
six circular functions of its gudermanian; for since
\begin{equation}
\frac{x_1}{a_1} = \cosh u, \quad \frac{x_2}{a_2} = \cosh v,
  \tag{see Arts.\ 6, 7}
\end{equation}
in which $u, v$ are the measures of corresponding hyperbolic and
elliptic sectors, hence
\begin{equation}
\left.
\begin{aligned}
  \cosh u &= \sec v, \qquad [\text{eq.\ (44)}] \\
  \sinh u &= \sqrt{\sec^2 v - 1} = \tan v,     \\
  \tanh u &= \frac{\tan v}{\sec v} = \sin v,   \\
  \coth u &= \csc v,   \\
  \sech u &= \cos v,   \\
  \csch u &= \cot v.   \\
\end{aligned}
\right\}\tag{46}
\end{equation}

The gudermanian is sometimes useful in computation; for instance, if
$\sinh u$ be given, $v$ can be found from a table of natural
tangents, and the other circular functions of $v$ will give the
remaining hyperbolic functions of $u$. Other uses of this function
are given in Arts.\ 22--26, 32--36.\index{Relations among functions}

\small \begin{enumerate}
\item[Prob.~49.] Prove that
\begin{align*}
\gd u &= \sec^{-1} (\cosh u) = \tan^{-1} (\sinh u) \\
      &= \cos^{-1} (\sech u) = \sin^{-1} (\tanh u).
\end{align*}%
\index{Anti-hyperbolic functions}\index{Circular functions}

\item[Prob.~50.] Prove
\begin{align*}
\gd^{-1} v &= \cosh^{-1} (\sec v) = \sinh^{-1} (\tan v) \\
           &= \sech^{-1} (\cos v) = \tanh^{-1} (\sin v).
\end{align*}

\item[Prob.~51.] Prove
\begin{align*}
\gd 0 &= 0, \;
  \gd   \infty  =  \tfrac{1}{2}\pi,\
  \gd (-\infty) = -\tfrac{1}{2}\pi, \\
\gd^{-1} 0 &= 0, \;
  \gd^{-1}\left( \tfrac{1}{2}\pi\right) =  \infty, \;
  \gd^{-1}\left(-\tfrac{1}{2}\pi\right) = -\infty.
\end{align*}

\item[Prob.~52.] Show that $\gd u$ and $\gd^{-1} v$ are odd
functions of $u, v$.

\item[Prob.~53.] From the first identity in 4, Prob.~17, derive the
relation $\tanh \frac{1}{2} u = \tan \frac{1}{2} v$.

\item[Prob.~54.] Prove $\tanh^{-1} (\tan u) = \tfrac{1}{2} \gd 2u,$
and $\tan^{-1} (\tanh x) = \tfrac{1}{2} \gd^{-1} 2x.$
\end{enumerate} \normalsize

\chapter{Gudermanian Angle}\index{Gudermanian!angle}%
\index{Hyperbolic functions!relations to gudermanian}

If a circle be used instead of the ellipse of Art.~20, the
gudermanian of the hyperbolic sectorial measure will be equal to the
radian measure of the angle of the corresponding circular sector
(see eq.~(6), and Art.~3, Prob.~2). This angle will be called the
gudermanian angle; but the gudermanian function $v$, as above
defined, is merely a number, or ratio; and this number is equal to
the radian measure of the gudermanian angle $\theta$, which is
itself usually tabulated in degree measure; thus
\begin{equation}
  \theta = \frac{180^\circ v}{\pi} \tag{47}
\end{equation}

\small \begin{enumerate}
\item[Prob.~55.] Show that the gudermanian angle of $u$ may be
constructed as follows:

\begin{center}
\includegraphics[width=35mm]{fig06.png}
\end{center}

Take the principal radius $OA$ of an equilateral hyperbola, as the
initial line, and $OP$ as the terminal line, of the sector whose
measure is $u$; from $M$, the foot of the ordinate of $P$, draw $MT$
tangent to the circle whose diameter is the transverse axis; then
$AOT$ is the angle required.%
\footnote{This angle was called by Gudermann the longitude of $u$,
and denoted by $lu$. His inverse symbol was $\textgoth{L}$; thus $u
= \textgoth{L}(lu)$. (Crelle's Journal, vol.~6, 1830.) Lambert, who
introduced the angle $\theta$, named it the transcendent angle.
(Hist.\ de l'acad.\ roy.\ de Berlin, 1761). Ho�el (Nouvelles
Annales, vol.~3, 1864) called it the hyperbolic amplitude of $u$,
and wrote it $\amh{u}$, in analogy with the amplitude of an elliptic
function, as shown in Prob.~62. Cayley (Elliptic Functions, 1876)
made the usage uniform by attaching to the angle the name of the
mathematician who had used it extensively in tabulation and in the
theory of elliptic functions of modulus unity.}%
\index{Cayley's Elliptic Functions}%
\index{Construction!for gudermanian}\index{Gudermann's notation}%
\index{Ho�el's notation, etc.}\index{Hyperbola}%
\index{Lambert's!notation}\index{Modulus}

\item[Prob.~56.] Show that the angle $\theta$ never exceeds
$90^{\circ}$.

\item[Prob.~57.] The bisector of angle $AOT$ bisects the sector $AOP$
(see Prob.~13, Art.~9, and Prob.~53, Art.~21), and the line $AP$.
(See Prob.~1, Art.~3.)

\item[Prob.~58.] This bisector is parallel to $TP$, and the points
$T$, $P$ are in line with the point diametrically opposite to $A$.

\item[Prob.~59.] The tangent at $P$ passes through the foot of the
ordinate of $T$, and intersects $TM$ on the tangent at $A$.

\item[Prob.~60.] The angle $APM$ is half the gudermanian angle.
\end{enumerate} \normalsize

\chapter{Derivatives of Gudermanian and Inverse.}%
\index{Derived functions}

Let
\begin{align*}
  v                   & = \gd{u}, \quad u = \gd^{-1}{v},  \\
\intertext{then}
  \sec{v}             & = \cosh{u},       \\
  \sec{v}\tan{v} \,dv & = \sinh{u} \,du,  \\
  \sec{v} \,dv        & = du,    \\
\intertext{therefore}
  d(\gd^{-1}{v})      & = \sec{v} \,dv.   \tag{48}  \\
\intertext{\qquad Again,}
  dv                  & = \cos{v} \,du = \sech{u} \,du,  \\
\intertext{therefore}
  d(\gd{u})           & = \sech{u} \,du.  \tag{49}
\end{align*}%
\index{Anti-gudermanian}

\small \begin{enumerate}
\item[Prob.~61.] Differentiate:
\begin{align*}
  y & = \sinh{u} - \gd{u},
& y & = \sin{v}  + \gd^{-1}{v},    \\
  y & = \tanh{u}\sech{u} + \gd{u},
& y & = \tan{v}\sec{v}   + \gd^{-1}{v}.
\end{align*}

\item[Prob.~62.] Writing the ``elliptic integral of the first
kind'' in the form
\begin{equation*}
u = \int_0^\phi \frac{d\phi}{\sqrt{1 - \kappa^2\sin^2\phi}},
\end{equation*}
$\kappa$ being called the modulus, and $\phi$ the amplitude; that is,
\begin{equation*}
\phi = \am u, (\moddot \kappa),
\end{equation*}
show that, in the special case when $\kappa = 1$,
\begin{align*}
u &= \gd^{-1} \phi, & \am u &= \gd u, & \sin \am u &= \tanh u, \\
\cos \am u &= \sech u, & \tan \am u &= \sinh u;
\end{align*}
and that thus the elliptic functions $\sin \am u$, etc., degenerate
into the hyperbolic functions, when the modulus is unity.%
\footnote{The relation $\gd u = \am u, (\moddot 1)$, led Ho�el to
name the function $\gd u$, the hyperbolic amplitude of $u$, and to
write it $\amh u$ (see note, Art.~22). In this connection Cayley
expressed the functions $\tanh u$, $\sech u$, $\sinh u$ in the form
$\sin \gd u$, $\cos \gd u$, $\tan \gd u$, and wrote them $\sg u$,
$\cg u$, $\tg u$, to correspond with the abbreviations $\sn u$, $\cn
u$, $\dn u$ for $\sin \am u$, $\cos \am u$, $\tan \am u$. Thus
$\tanh u = \sg u = \sn u, (\moddot 1)$; etc.\index{Modulus}

\indent It is well to note that neither the elliptic nor the
hyperbolic functions received their names on account of the relation
existing between them in a special case. (See foot-note,
p.~\ref{fnp7})}%
\index{Amplitude!hyperbolic}\index{Cayley's Elliptic Functions}%
\index{Elliptic functions}\index{Elliptic integrals}%
\index{Elliptic sectors}\index{Ho�el's notation, etc.}
\end{enumerate} \normalsize

\chapter{Series for Gudermanian and its Inverse.}%
\index{Gudermanian!function}\index{Series}

Substitute for $\sech u, \sec v$ in (49), (48) their expansions,
Art.~16, and integrate, then
\begin{align*}
\gd u &= u - \frac{1}{6} u^3 + \frac{1}{24} u^5 - \frac{61}{5040}
u^7 + \dotsb \tag{50} \\
\gd^{-1} v &= v + \frac{1}{6} v^3 + \frac{1}{24} v^5 -
\frac{61}{5040} v^7 + \dotsb \tag{51}
\end{align*}
No constants of integration appear, since $\gd u$ vanishes with $u$,
and $\gd^{-1} v$ with $v$. These series are seldom used in
computation, as $\gd u$ is best found and tabulated by means of
tables of natural tangents and hyperbolic sines, from the equation
\begin{equation*}
\gd u = \tan^{-1}(\sinh u),
\end{equation*}
and a table of the direct function can be used to furnish the
numerical values of the inverse function; or the latter can be
obtained from the equation,
\begin{equation*}
\gd^{-1} v = \sinh^{-1}(\tan v) = \cosh^{-1}(\sec v).
\end{equation*}\index{Expansion in series}

To obtain a logarithmic expression for $\gd^{-1} v$, let
\begin{align*}
\gd^{-1} v &= u,\quad v = \gd u, \\
\intertext{therefore}
\sec v &= \cosh u,\quad \tan v = \sinh u, \\
\sec v + \tan v &= \cosh u + \sinh u = e^u, \\
e^u = \frac{1 + \sin v}{\cos v} &=
  \frac{1-\cos(\frac{1}{2}\pi + v)}{\sin(\frac{1}{2}\pi + v)} =
  \tan\left(\frac{1}{4}\pi + \frac{1}{2}v\right), \\
u = \gd^{-1} v &=
       \log_e\tan\left(\frac{1}{4}\pi + \frac{1}{2}v\right). \tag{52}
\end{align*}

\small \begin{enumerate}
\item[Prob.~63.] Evaluate $\left.\dfrac{\gd u -
u}{u^3}\right]_{u\doteq 0}$, $\left.\dfrac{\gd^{-1} v -
v}{v^3}\right]_{v\doteq 0}$.\index{Limiting ratios}

\item[Prob.~64.] Prove that $\gd u - \sin u$ is an infinitesimal of the
fifth order, when $u \doteq 0$.\index{Logarithmic!expressions}

\item[Prob.~65.] Prove the relations $\frac{1}{4}\pi + \frac{1}{2} v
\tan^{-1}e^u$, $\frac{1}{4}\pi - \frac{1}{2} v = \tan^{-1}e^{-u}$.
\end{enumerate} \normalsize

\chapter{Graphs of Hyperbolic Functions.}%
\index{Construction!of graphs}\index{Graphs}%
\index{Hyperbolic functions!graphs of}

\begin{figure*}[p]
\begin{center}
\includegraphics[width=60mm]{fig07.png} \\
\includegraphics[width=50mm]{fig08.png}
\includegraphics[width=50mm]{fig09.png} \\
\includegraphics[width=100mm]{fig10.png}
\end{center}
\end{figure*}

Drawing two rectangular axes, and laying down a series of points
whose abscissas represent, on any convenient scale, successive
values of the sectorial measure, and whose ordinates represent,
preferably on the same scale, the corresponding values of the
function to be plotted, the locus traced out by this series of
points will be a graphical representation of the variation of the
function as the sectorial measure varies. The equations of the
curves in the ordinary cartesian notation are:

\medskip \begin{center}
\begin{tabular}{l l l}
\multicolumn{1}{c}{Fig.} &
       \multicolumn{1}{c}{Full Lines.}
                      & \multicolumn{1}{c}{Dotted Lines.} \\
A    & $y = \cosh x,$ & $y = \sech x;$  \\
B    & $y = \sinh x,$ & $y = \csch x;$  \\
C    & $y = \tanh x,$ & $y = \coth x;$  \\
D    & $y = \gd x.$   &
\end{tabular}
\end{center}

Here $x$ is written for the sectorial measure $u$, and $y$ for the
numerical value of $\cosh u$, etc. It is thus to be noted that the
variables $x$, $y$ are numbers, or ratios, and that the equation $y
= \cosh x$ merely expresses that the relation between the numbers
$x$ and $y$ is taken to be the same as the relation between a
sectorial measure and its characteristic ratio. The numerical values
of $\cosh u, \sinh u, \tanh u$ are given in the tables at the end of
this chapter for values of $u$ between $0$ and $4$. For greater
values they may be computed from the developments of Art.~16.

The curves exhibit graphically the relations:
\begin{gather*}
\sech u = \frac{1}{\cosh u}, \quad
  \csch u = \frac{1}{\sinh u}, \quad
  \coth u = \frac{1}{\tanh u}; \\
\cosh u \nless 1, \quad \sech u \ngtr 1, \quad
  \tanh u \ngtr 1, \quad \gd u < \tfrac{1}{2}\pi,
  \text{ etc.}; \\
\sinh(-u) = -\sinh u, \quad \cosh(-u) = \cosh u, \\
\tanh(-u) = -\tanh u, \quad \gd(-u) = -\gd u, \text{ etc.}; \\
\cosh 0 = 1, \quad \sinh 0 = 0, \quad
  \tanh 0 = 0, \quad \csch(0) = \infty, \text{ etc.}; \\
\cosh(\pm\infty) = \infty, \quad
  \sinh(\pm\infty) = \pm\infty, \quad
  \tanh(\pm\infty) = \pm 1, \text{ etc.}
\end{gather*}

The slope of the curve $y = \sinh x$ is given by the equation
$\dfrac{dy}{dx} = \cosh x$, showing that it is always positive, and
that the curve becomes more nearly vertical as $x$ becomes infinite.
Its direction of curvature is obtained from $\dfrac{d^2y}{dx^2} =
\sinh x$, proving that the curve is concave downward when $x$ is
negative, and upward when $x$ is positive. The point of inflexion is
at the origin, and the inflexional tangent bisects the angle between
the axes.

The direction of curvature of the locus $y = \sech x$ is given by
$\dfrac{d^2y}{dx^2}=$ $\sech x (2 \tanh^2 x - 1)$, and thus the
curve is concave downwards or upwards according as $2 \tanh^2 x - 1$
is negative or positive. The inflexions occur at the points $x = \pm
\tanh^{-1} .707, = \pm .881$, $y =.707$; and the slopes of the
inflexional tangents are $\mp\frac{1}{2}$.

The curve $y = \csch x$ is asymptotic to both axes, but approaches
the axis of $x$ more rapidly than it approaches the axis of $y$, for
when $x = 3$, $y$ is only $.1$, but it is not till $y = 10$ that $x$
is so small as $.1$. The curves $y = \csch x$, $y = \sinh x$ cross
at the points $x = \pm .881$, $y = \pm 1$.

\small \begin{enumerate}
\item[Prob.~66.] Find the direction of curvature, the inflexional
tangent, and the asymptotes of the curves $y = \gd x$, $y = \tanh
x$.\index{Gudermanian!function}

\item[Prob.~67.] Show that there is no inflexion-point on the curves
$y = \cosh x$, $y = \coth x$.

\item[Prob.~68.] Show that any line $y = mx + n$ meets the curve $y =
\tanh x$ in either three real points or one. Hence prove that the
equation $\tanh x = mx + n$ has either three real roots or one. From
the figure give an approximate solution of the equation $\tanh x = x
- 1$.

\item[Prob.~69.] Solve the equations: $\cosh x = x + 2$; $\sinh x =
\frac{3}{2} x$; $\gd x = x - \frac{1}{2}\pi$.

\item[Prob.~70.] Show which of the graphs represent even functions,
and which of them represent odd ones.
\end{enumerate} \normalsize

\chapter{Elementary Integrals.}%
\index{Anti-hyperbolic functions}\index{Circular functions}%
\index{Equations!Numerical}%
\index{Hyperbolic functions!integrals involving}\index{Integrals}

The following useful indefinite integrals follow from Arts. 14, 15,
23:

\newcommand{\dint}{\displaystyle\int}
\medskip\begin{tabular}{rll}
& \multicolumn{1}{c}{Hyperbolic.} & \multicolumn{1}{c}{Circular.} \\
1. & $\dint \sinh u\: du = \cosh u,$
   & $\dint \sin u\: du = -\cos u,$ \\
2. & $\dint \cosh u\: du = \sinh u,$
   & $\dint \cos u\: du = \sin u,$ \\
3. & $\dint \tanh u\: du = \log \cosh u,$
   & $\dint \tan u\: du = -\log \cos u,$ \\
4. & $\dint \coth u\: du = \log \sinh u,$
   & $\dint \cot u\: du = \log \sin u,$ \\
5. & $\dint \csch u\: du = \log \tanh \frac{u}{2},$
   & $\dint \csc u\: du = \log \tan\dfrac{u}2,$ \\
   & $\qquad = -\sinh^{-1}(\csch u),$
   & $\qquad = -\cosh^{-1}(\csch u),$ \\
6. & $\dint \sech u\: du = \gd u,$
   & $\dint \sec u\: du = \gd^{-1} u,$ \\
7. & $\dint \frac{dx}{\sqrt{x^2+a^2}} =
            \sinh^{-1}\frac{x}{a},$\footnotemark
   & $\dint \frac{dx}{\sqrt{a^2-x^2}} =
            \sin^{-1}\frac{x}{a},$ \\
8. & $\dint \frac{dx}{\sqrt{x^2-a^2}} =
            \cosh^{-1}\frac{x}{a},$
   & $\dint \frac{-dx}{\sqrt{a^2-x^2}} =
            \cos^{-1}\frac{x}{a},$ \\
9. & $\dint \left.\frac{dx}{a^2-x^2}\right]_{x<a} =
            \frac{1}{a}\tanh^{-1}\frac{x}{a},$
   & $\dint \frac{dx}{a^2+x^2} = \frac{1}{a}\tan^{-1}\frac{x}{a},$
\\
10. & $\dint \left.\frac{-dx}{x^2-a^2}\right]_{x>a} =
             \frac{1}{a}\coth^{-1} \frac{x}{a},$
    & $\dint \frac{-dx}{a^2+x^2} =
             \frac{1}{a}\cot^{-1}\frac{x}{a},$\\
11. & $\dint \frac{-dx}{x\sqrt{a^2-x^2}} =
             \frac{1}{a}\sech^{-1}\frac{x}a,$
    & $\dint \frac{dx}{x\sqrt{x^2-a^2}} =
             \frac{1}{a}\sec^{-1}\frac{x}{a},$ \\
12. & $\dint \frac{-dx}{x\sqrt{a^2+x^2}} =
             \frac{1}{a}\csch^{-1} \frac{x}{a},$
    & $\dint \frac{-dx}{x\sqrt{x^2-a^2}} =
             \frac{1}{a}\csc^{-1}\frac{x}{a}.$
\end{tabular}
\footnotetext{Forms 7--12 are preferable to the respective
logarithmic expressions (Art.~19), on account of the close analogy
with the circular forms, and also because they involve functions
that are directly tabulated. This advantage appears more clearly in
13--20.}

From these fundamental integrals the following may be
derived:
\begin{equation*}
\begin{aligned}
13.\quad \int\frac{dx}{\sqrt{ax^2+2bx+c}} &=
\frac{1}{\sqrt{a}}\sinh^{-1} \frac{ax+b}{\sqrt{ac-b^2}},\
a \text{ positive, } ac > b^2;\\
&= \frac{1}{\sqrt{a}}\cosh^{-1} \frac{ax+b}{\sqrt{b^2-ac}},\
a \text{ positive, } ac < b^2;\\
&= \frac{1}{\sqrt{-a}}\cos^{-1} \frac{ax+b}{\sqrt{b^2-ac}},\ a
\text{ negative}.
\end{aligned}
\end{equation*}

\begin{equation*}
\begin{aligned}
14.\quad \int\frac{dx}{ax^2+2bx+c} &=
   \frac{1}{\sqrt{ac-b^2}}\tan^{-1}\frac{ax+b}{\sqrt{ac-b^2}},\
   ac > b^2; \\
&= \frac{-1}{\sqrt{b^2-ac}}\tanh^{-1} \frac{ax+b}{\sqrt{b^2-ac}},\
   ac < b^2, ax+b < \sqrt{b^2-ac};\\
&= \frac{-1}{\sqrt{b^2-ac}}\coth^{-1} \frac{ax+b}{\sqrt{b^2-ac}},\
   ac < b^2, ax+b > \sqrt{b^2-ac};
\end{aligned}
\end{equation*}

Thus,
\begin{align*}
\int_4^5 \frac{dx}{x^2-4x+3}
&= \left.-\coth^{-1}(x-2)\right]_4^5 = \coth^{-1}2 -\coth^{-1}3 \\
&= \tanh^{-1}(.5) - \tanh^{-1}(.3333) = .5494 - .3466
= .2028.\footnotemark \\
\int_2^{2.5} \frac{dx}{x^2-4x+3}
&= \left.-\tanh^{-1}(x-2)\right]_2^{2.5} = \tanh^{-1}0 -
\tanh^{-1}(0.5) =-.5494.
\end{align*}
\footnotetext{For $\tanh^{-1}(.5)$ interpolate between $\tanh (.54)
= .4930$, $\tanh (.56) =.5080$ (see tables, pp.~\pageref{Table1p1},
\pageref{Table1p2}); and similarly for $\tanh^{-1}(.3333)$.}%
\index{Interpolation}

(By interpreting these two integrals as areas, show graphically
that the first is positive, and the second negative.)%
\index{Areas}

\begin{align*}
15.\quad \int \frac{dx}{(a-x)\sqrt{x-b}} &=
   \frac{2}{\sqrt{a-b}}\tanh^{-1}\sqrt{\frac{x-b}{a-b}}, \\
&\text{or } \frac{-2}{\sqrt{b-a}}\tan^{-1}\sqrt{\frac{x-b}{b-a}}, \\
&\text{or } \frac{2}{\sqrt{a-b}}\coth^{-1}\sqrt{\frac{x-b}{a-b}};
\end{align*}
the real form to be taken. (Put $x - b = z^2$, and apply 9, 10.)

\begin{align*}
16.\quad \int\frac{dx}{(a-x)\sqrt{b-x}} &=
   \frac{2}{\sqrt{b-a}}\tanh^{-1}\sqrt{\frac{b-x}{b-a}}, \\
&\text{or }\frac{2}{\sqrt{b-a}}\coth^{-1}\sqrt{\frac{b-x}{b-a}}, \\
&\text{or } \frac{-2}{\sqrt{a-b}}\tan^{-1}\sqrt{\frac{b-x}{a-b}};
\end{align*}
the real form to be taken.

\begin{equation*}
17.\quad \int(x^2-a^2)^{\frac{1}{2}}dx =
  \frac{1}{2}x(x^2-a^2)^{\frac{1}{2}} -
  \frac{1}{2}a^2\cosh^{-1}\frac{x}{a}.
\end{equation*}

By means of a reduction-formula this integral is easily made to
depend on 8. It may also be obtained by transforming the expression
into hyperbolic functions by the assumption $x = a\cosh u$, when the
integral takes the form
\begin{align*}
a^2\int\sinh^2u\,du = \frac{a^2}{2}\int(\cosh 2u-1)du
&= \frac{1}{4}a^2(\sinh 2u-2u) \\
&= \frac{1}{2}a^2(\sinh u\cosh u-u),
\end{align*}
which gives 17 on replacing $a\cosh u$ by $x$, and $a\sinh u$ by
$(x^2-a^2)^{\frac{1}{2}}$. The geometrical interpretation of the
result is evident, as it expresses that the area of a
rectangular-hyperbolic segment $AMP$ is the difference between a
triangle $OMP$ and a sector $OAP$.%
\index{Areas}\index{Geometrical interpretation}\index{Hyperbola}%
\index{Reduction formula}

\begin{align*}
18.\quad \int(a^2-x^2)^{\frac{1}{2}}dx &=
  \frac{1}{2}x(a^2-x^2)^{\frac{1}{2}} +
  \frac{1}{2}a^2\sin^{-1}\frac{x}{a}.\\
19.\quad \int(x^2+a^2)^{\frac{1}{2}} dx &=
  \frac{1}{2}x(x^2-a^2)^{\frac{1}{2}} +
  \frac{1}{2}a^2\sinh^{-1}\frac{x}{a}.\\
20.\quad \int\sec^3\phi\,d\phi &=
  \int(1+\tan^2\phi)^{\frac{1}{2}}d\tan\phi\\
&= \frac{1}{2}\tan\phi(1+\tan^2\phi)^{\frac{1}{2}}
  +\frac{1}{2}\sinh^{-1}(\tan\phi) \\
&= \frac{1}{2}\sec\phi\tan\phi + \frac{1}{2}\gd^{-1}\phi. \\
21.\quad \int\sech^3u\,du &=
   \frac{1}{2}\sech u\tanh u + \frac{1}{2}\gd u.
\end{align*}

\small \begin{enumerate}
\item[Prob.~71.] What is the geometrical interpretation of 18, 19?

\item[Prob.~72.] Show that $\int(ax^2+2bx+c)^{\frac{1}{2}}dx$ reduces
to 17, 18, 19, respectively: when $a$ is positive, with $ac < b^2$;
when $a$ is negative; and when $a$ is positive, with $ac > b^2$.

\item[Prob.~73.] Prove
\begin{align*}
\int\sinh u\tanh u\,du &= \sinh u - \gd u,\\
\int\cosh u\coth u\,du &= \cosh u + \log\tanh\frac{u}{2}.
\end{align*}

\item[Prob.~74.] Integrate $(x^2+2x+5)^{-\frac{1}{2}}dx$,
$(x^2+2x+5)^{-1}dx$, $(x^2+2x+5)^{\frac{1}{2}}dx$.

\item[Prob.~75.] In the parabola $y^2 = 4px$, if $s$ be the length of
arc measured from the vertex, and $\phi$ the angle which the tangent
line makes with the vertical tangent, prove that the intrinsic
equation of the curve is $\dfrac{ds}{d\phi} = 2p\sec^3\phi$, $s =
p\sec\phi\tan\phi + p\gd^{-1}\phi$.\index{Intrinsic equation}%
\index{Parabola}

\item[Prob.~76.] The polar equation of a parabola being
$r = a\sec^2\theta$, referred to its focus as pole, express $s$ in
terms of $\theta$.

\item[Prob.~77.] Find the intrinsic equation of the curve
$\dfrac{y}{a} = \cosh \dfrac{x}{a}$, and of the curve $\dfrac{y}{a}
= \log\sec\dfrac{x}{a}$.

\item[Prob.~78.] Investigate a formula of reduction for
$\dint\cosh^nx\,dx$; also integrate by parts \\
$\cosh^{-1}x\,dx$, $\tanh^{-1}x\,dx$, $(\sinh^{-1}x)^2dx$; and show
that the ordinary methods of reduction for $\dint\cos^mx\sin^nx\,dx$
can be applied to $\dint\cosh^mx\sinh^nx\,dx$.\index{Reduction
formula}
\end{enumerate}\normalsize

\chapter{Functions of Complex Numbers.}%
\index{Complex numbers|(}\index{Function!of complex numbers}%
\index{Hyperbolic functions!of complex numbers|(}%
\index{Numbers, complex|(}

As vector quantities are of frequent occurrence in Mathematical
Physics; and as the numerical measure of a vector in terms of a
standard vector is a complex number of the form $x+iy$, in which $x,
y$ are real, and $i$ stands for $\sqrt{-1}$; it becomes necessary in
treating of any class of functional operations to consider the
meaning of these operations when performed on such generalized
numbers.\footnote{%
The use of vectors in electrical theory is shown in Bedell and
Crehore's Alternating Currents, Chaps, XIV--XX (first published in
1892). The advantage of introducing the complex measures of such
vectors into the differential equations is shown by Steinmetz, Proc.
Elec. Congress, 1893; while the additional convenience of expressing
the solution in hyperbolic functions of these complex numbers is
exemplified by Kennelly, Proc. American Institute Electrical
Engineers, April 1895. (See below, Art.~37.)}%
\index{Alternating currents}%
\index{Bedel and Crehore's alternating currents}%
\index{Kennelly on alternating currents}%
\index{Steinmetz on alternating currents}\index{Vectors} The
geometrical definitions of $\cosh u$, $\sinh u$, given in Art.~7,
being then no longer applicable, it is necessary to assign to each
of the symbols $\cosh(x+iy)$, $\sinh(x+iy)$, a suitable algebraic
meaning, which should be consistent with the known algebraic values
of $\cosh x$, $\sinh x$, and include these values as a particular
case when $y = 0$. The meanings assigned should also, if possible,
be such as to permit the addition-formulas of Art.~11 to be made
general, with all the consequences that flow from them.

Such definitions are furnished by the algebraic developments in
Art.~16, which are convergent for all values of $u$, real or
complex. Thus the definitions of $\cosh(x+iy)$, $\sinh(x+iy)$ are to
be
\begin{equation*}
  \left.
    \begin{aligned}
      \cosh(x+iy) &=  1 + \frac{1}{2!}(x+iy)^2 + \frac{1}{4!}(x+iy)^4
      + \dots,\\
      \sinh(x+iy) & = (x+iy) + \frac{1}{3!}(x+iy)^3 + \dots
    \end{aligned}
  \right\}\tag{52}
\end{equation*}

From these series the numerical values of $\cosh(x+iy)$,
$\sinh(x+iy)$ could be computed to any degree of approximation, when
$x$ and $y$ are given. In general the results will come out in the
complex form\footnote{%
It is to be borne in mind that the symbols cosh, sinh, here stand
for algebraic operators which convert one number into another; or
which, in the language of vector-analysis, change one vector into
another, by stretching and turning.}
\begin{align*}
  \cosh(x+iy) &= a+ib,\\
  \sinh(x+iy) &= c+id.
\end{align*}
The other functions are defined as in Art.~7, eq.~(9).

\small \begin{enumerate}
\item[Prob.~79.] Prove from these definitions that, whatever $u$ may
be,
\begin{align*}
  \cosh(-u) &= \cosh u, &
  \sinh(-u) &= -\sinh u, \\
  \frac{d}{du} \cosh u &= \sinh u, &
  \frac{d}{du} \sinh u &= \cosh u,\\
  \frac{d^2}{du^2} \cosh mu &= m^2\cosh mu, &
  \frac{d^2}{du^2} \sinh mu &= m^2\sinh mu.\footnotemark
\end{align*}
\footnotetext{The generalized hyperbolic functions usually present
themselves in Mathematical Physics as the solution of the
differential equation $\dfrac{d^2\phi}{du^2} = m^2\phi$, where
$\phi$, $m$, $u$ are complex numbers, the measures of vector
quantities. (See Art.~37.)}\index{Operators, generalized}%
\index{Permanence of equivalence}
\end{enumerate}\index{Circular functions}\normalsize

\chapter{Addition-Theorems for Complexes.}%
\index{Addition-theorems}

The addition-theorems for $\cosh(u+v)$, etc., where $u$, $v$ are
complex numbers, may be derived as follows. First take $u$, $v$ as
real numbers, then, by Art.~11,
\begin{align*}
\cosh(u+v) &= \cosh u \cosh v + \sinh u\ \sinh v; \\
\intertext{hence}
1 + \frac{1}{2!}(u+v)^2+ \dots
  &= \left(1+\frac{1}{2!}u^2+\dots\right)
     \left(1+\frac{1}{2!}v^2+\dots\right) \\
  &\quad + \left(u+\frac{1}{3!}u^3+\dots\right)
     \left(v+\frac{1}{3!}v^3+\dots\right)
\end{align*}

This equation is true when $u$, $v$ are any real numbers. It must,
then, be an algebraic identity. For, compare the terms of the $r$th
degree in the letters $u, v$ on each side. Those on the left are
$\dfrac{1}{r!}(u+v)^r$; and those on the right, when collected, form
an $r$th-degree function which is numerically equal to the former
for more than $r$ values of $u$ when $v$ is constant, and for more
than $r$ values of $v$ when $u$ is constant. Hence the terms of the
$r$th degree on each side are algebraically identical functions of
$u$ and $v$.\footnote{%
``If two $r$th-degree functions of a single variable be equal for
more than $r$ values of the variable, then they are equal for all
values of the variable, and are algebraically identical.''}
Similarly for the terms of any other degree. Thus the equation above
written is an algebraic identity, and is true for all values of $u$,
$v$, whether real or complex. Then writing for each side its symbol,
it follows that
\begin{align*}
  \cosh(u+v) &= \cosh u \cosh v + \sinh u \sinh v;\tag{53} \\
\intertext{and by changing $v$ into $-v$,}
  \cosh(u-v) &= \cosh u \cosh v - \sinh u \sinh v.\tag{54}
\end{align*}

In a similar manner is found
\begin{equation}
  \sinh(u\pm v) = \sinh u \cosh v \pm \cosh u \sinh v.\tag{55}
\end{equation}

In particular, for a complex argument,
\begin{equation}
  \left.
    \begin{aligned}
      \cosh(x\pm iy) &= \cosh x \cosh iy \pm \sinh x \sinh iy,\\
      \sinh(x\pm iy) &= \sinh x \cosh iy \pm \cosh x \sinh iy.
    \end{aligned}
  \right\}\tag{56}
\end{equation}

\small \begin{enumerate}
\item[Prob.~79.] Show, by a similar process of generalization,%
\footnote{This method of generalization is sometimes called the
principle of the ``permanence of equivalence of forms.'' It is not,
however, strictly speaking, a ``principle,'' but a method; for, the
validity of the generalization has to be demonstrated, for any
particular form, by means of the principle of the algebraic identity
of polynomials enunciated in the preceding foot-note. (See Annals of
Mathematics, Vol.~6, p.~81.)} that if $\sin u$, $\cos u$, $\exp u$%
\footnote{The symbol $\exp u$ stands for ``exponential function of
u,'' which is identical with $e^u$ when $u$ is real.} be defined by
their developments in powers of $u$, then, whatever $u$ may be,
\begin{align*}
  \sin (u+v) &= \sin u \cos v + \cos u \sin v,   \\
  \cos (u+v) &= \cos u \cos v - \sin u \sin v,   \\
  \exp (u+v) &= \exp u \exp v.
\end{align*}\index{Generalization}

\item[Prob. 80.] Prove that the following are identities:
\begin{align*}
  \cosh^2 u - \sinh^2 u &= 1,   \\
  \cosh   u + \sinh   u &= \exp u,   \\
  \cosh u - \sinh u &= \exp (-u),   \\
  \cosh u &= \tfrac{1}{2} [\exp u + \exp (-u)],   \\
  \sinh u &= \tfrac{1}{2} [\exp u - \exp (-u)].
\end{align*}
\end{enumerate} \normalsize

\chapter{Functions of Pure Imaginaries.}%
\index{Functions!of pure imaginaries}\index{Pure imaginary}

In the defining identities%
\index{Algebraic identity}
\begin{align*}
  \cosh u &= 1 + \frac{1}{2!} u^2 + \frac{1}{4!} u^4 + \dotsb,  \\
  \sinh u &= 1 + \frac{1}{3!} u^3 + \frac{1}{5!} u^5 + \dotsb,
\end{align*}
put for $u$ the pure imaginary $iy$, then
\begin{align*}
\cosh iy &= 1 - \frac{1}{2!} y^2 + \frac{1}{4!} y^4 - \dotsb
          = \cos y, \tag{57} \\
\sinh iy &= iy + \frac{1}{3!} (iy)^3 + \frac{1}{5!} (iy)^5
            + \dotsb \\
&= i\left[ y - \frac{1}{3!} y^3 + \frac{1}{5!} y^5 - \dotsb \right]
= i\sin y, \tag{58}  \\
\intertext{and, by division,}
\tanh iy &= i\tan y. \tag{59}
\end{align*}

These formulas serve to interchange hyperbolic and circular
functions. The hyperbolic cosine of a pure imaginary is real, and
the hyperbolic sine and tangent are pure imaginaries.

The following table exhibits the variation of $\sinh u$, $\cosh u$.
$\tanh u$, $\exp u$, as $u$ takes a succession of pure imaginary
values.\index{Tables}

\begin{minipage}{10cm}{
\begin{center}
\begin{tabular}{|c|c|c|c|c|}
\hline \rule[-5pt]{0pt}{16pt}
  $u$ & $\sinh u$ & $\cosh u$ & $\tanh u$ & $\exp u$ \\
\hline \rule[-5pt]{0pt}{16pt}
  $0$ & $0$ & $1$ & $0$ & $1$ \\
\hline \rule[-5pt]{0pt}{16pt}
  $\frac{1}{4}i\pi$ & $.7i$ & $.7\footnotemark$ & $i$ & $.7(1+i)$ \\
\hline \rule[-5pt]{0pt}{16pt}
  $\frac{1}{2}i\pi$ & $i$ & $0$ & $\infty i$ & $i$ \\
\hline \rule[-5pt]{0pt}{16pt}
  $\frac{3}{4}i\pi$ & $.7i$ & $-.7$ & $-i$ & $.7(1-i)$ \\
\hline \rule[-5pt]{0pt}{16pt}
  $i\pi$ & $0$ & $-1$ & $0$ & $-1$ \\
\hline \rule[-5pt]{0pt}{16pt}
  $\frac{5}{4}i\pi$ & $-.7i$ & $-.7$ & $i$ & $-.7(1+i)$ \\
\hline \rule[-5pt]{0pt}{16pt}
  $\frac{3}{2}i\pi$ & $-i$ & $0$ & $\infty i$ & $-i$ \\
\hline \rule[-5pt]{0pt}{16pt}
  $\frac{7}{4}i\pi$ & $-.7i$ & $.7$ & $-i$ & $-.7(1-i)$ \\
\hline \rule[-5pt]{0pt}{16pt}
  $2i\pi$ & $0$ & $1$ & $0$ & $1$ \\
\hline
\end{tabular}
\footnotetext{In this table $.7$ is written for
$\frac{1}{2}\sqrt{2}, = .707\dotsc$.}
\end{center}
}\end{minipage}

\small \begin{enumerate}
\item[Prob.~81.] Prove the following identities:
\begin{align*}
\cos y = \cosh iy \phantom{\frac{1}{i}}
      &= \frac{1}{2}  \left[\exp iy + \exp (-iy)\right], \\
\sin y = \frac{1}{i}  \sinh iy
      &= \frac{1}{2i} \left[\exp iy - \exp (-iy)\right], \\
\cos y + i\sin y &= \cosh iy + \sinh iy = \exp iy,    \\
\cos y - i\sin y &= \cosh iy - \sinh iy = \exp\, (-iy), \\
\cos iy &= \cosh y, \quad \sin iy = i\sinh y.
\end{align*}\index{Circular functions}%
\index{Hyperbolic functions!relations to circular functions}%
\index{Interchange of hyperbolic and circular functions}%
\index{Relations among functions}

\item[Prob.~82.] Equating the respective real and imaginary parts
on each side of the equation $\cos ny + i\sin ny = (\cos y + i\sin
y)^n$, express $\cos ny$ in powers of $\cos y$, $\sin y$; and hence
derive the corresponding expression for $\cosh ny$.\index{Circular
functions!of complex numbers}

\item[Prob.~83.] Show that, in the identities (57) and (58),
$y$ may be replaced by a general complex, and hence that
\begin{align*}
\sinh (x \pm iy) &= \pm i \sin  (y \mp ix),\\
\cosh (x \pm iy) &=       \cos  (y \mp ix),\\
\sin  (x \pm iy) &= \pm i \sinh (y \mp ix),\\
\cos  (x \pm iy) &=       \cosh (y \mp ix).
\end{align*}

\item[Prob.~84.] From the product-series for $\sin x$ derive
that for $\sinh x$:
\begin{align*}
\sin  x &= x\left(1 - \frac{x^2}{   \pi^2}\right)
            \left(1 - \frac{x^2}{2^2\pi^2}\right)
            \left(1 - \frac{x^2}{3^2\pi^2}\right) \ldots,\\
\sinh x &= x\left(1 + \frac{x^2}{   \pi^2}\right)
            \left(1 + \frac{x^2}{2^2\pi^2}\right)
            \left(1 + \frac{x^2}{3^2\pi^2}\right) \ldots.
\end{align*}\index{Product-series}
\end{enumerate} \normalsize

\chapter{Functions of $x+iy$ in the Form $X+iY$.}

By the addition-formulas,
\begin{gather*}
\begin{aligned}
\cosh (x + iy) &= \cosh x \cosh iy + \sinh x \sinh iy,\\
\sinh (x + iy) &= \sinh x \cosh iy + \cosh x \sinh iy,
\end{aligned}
\intertext{but}
\cosh iy = \cos y,\quad  \sinh iy = i \sin y, \\
\intertext{hence}
\left.
\begin{aligned}
\cosh (x + iy) &= \cosh x \cos y + i \sinh x \sin y,\\
\sinh (x + iy) &= \sinh x \cos y + i \cosh x \sin y.
\end{aligned}
\right\}\tag{60}
\end{gather*}

Thus if $\cosh (x + iy) = a+ib$, $\sinh (x + iy) = c + id$, then
\begin{equation}
\left.
\begin{aligned}
a &= \cosh x \cos y, &\quad b &= \sinh x \sin y,\\
c &= \sinh x \cos y, &\quad d &= \cosh x \sin y.
\end{aligned}
\right\}\tag{61}
\end{equation}

From these expressions the complex tables at the end of this chapter
have been computed.

Writing $\cosh z = Z$, where $z = x + iy$, $Z = X + iY$; let the
complex numbers $z, Z$ be represented on Argand diagrams,%
\index{Argand diagram}\index{Construction!of charts} in the usual
way, by the points whose coordinates are $(x, y)$, $(X, Y)$; and let
the point $z$ move parallel to the $y$-axis, on a given line $x =
m$, then the point $Z$ will describe an ellipse whose equation,
obtained by eliminating $y$ between the equations $X= \cosh m \cos
y$, $Y= \sinh m \sin y$, is
\begin{equation*}
\frac{X^2}{(\cosh m)^2} + \frac{Y^2}{(\sinh m)^2} = 1,
\end{equation*}
and which, as the parameter $m$ varies, represents a series of
confocal ellipses, the distance between whose foci is unity.
Similarly, if the point $z$ move parallel to the $x$-axis, on a
given line $y=n$, the point $Z$ will describe an hyperbola whose
equation, obtained by eliminating the variable $x$ from the
equations. $X = \cosh x \cos n$, $Y = \sinh x \sin n$, is
\begin{equation*}
\frac{X^2}{(\cos n)^2} - \frac{Y^2}{(\sin n)^2} = 1,
\end{equation*}
and which, as the parameter $n$ varies, represents a series of
hyperbolas confocal with the former series of
ellipses.\index{Ellipses, chart of confocal}\index{Hyperbola}

These two systems of curves, when accurately drawn at close
intervals on the $Z$ plane, constitute a chart of the hyperbolic
cosine; and the numerical value of $\cosh (m + in)$ can be read off
at the intersection of the ellipse whose parameter is $m$ with the
hyperbola whose parameter is $n$.\footnote{%
Such a chart is given by Kennelly, Proc.~A.~I.~E.~E., April 1895,
and is used by him to obtain the numerical values of $\cosh (x +
iy)$, $\sinh (x+iy)$, which present themselves as the measures of
certain vector quantities in the theory of alternating currents.
(See Art.~37.) The chart is constructed for values of $x$ and of $y$
between 0 and 1.2; but it is available for all values of $y$, on
account of the periodicity of the functions.}%
\index{Alternating currents}\index{Chart!of hyperbolic functions}%
\index{Kennelly's chart} A similar chart can be drawn for $\sinh
(x+iy)$, as indicated in Prob.~85.

\medskip Periodicity of Hyperbolic
Functions.\label{periodicity of hyperbolic functions}---The
functions $\sinh u$ and $\cosh u$ have the pure imaginary period
$2i\pi$. For
\begin{align*}
\sinh(u+2i\pi) &= \sinh u \cos 2\pi + i \cosh u \sin 2\pi = \sinh u,\\
\cosh(u+2i\pi) &= \cosh u \cos 2\pi + i \sinh u \sin 2\pi = \cosh u.
\end{align*}\index{Function!periodic}\index{Periodicity}

The functions $\sinh u$ and $\cosh u$ each change sign when the
argument $u$ is increased by the half period $i\pi$. For
\begin{align*}
\sinh (u+i\pi) &= \sinh u \cos \pi + i \cosh u \sin \pi = -\sinh u,\\
\cosh (u+i\pi) &= \cosh u \cos \pi + i \sinh u \sin \pi = -\cosh u.
\end{align*}

The function $\tanh u$ has the period $i\pi$. For, it follows from
the last two identities, by dividing member by member, that
\begin{equation*}
\tanh (u+i\pi) = \tanh u.
\end{equation*}

By a similar use of the addition formulas it is shown that
\begin{equation*}
\sinh (u + \frac{1}{2} i\pi) = i \cosh u,\quad
\cosh (u + \frac{1}{2} i\pi) = i \sinh u.
\end{equation*}

By means of these periodic, half-periodic, and quarter-periodic
relations, the hyperbolic functions of $x + iy$ are easily
expressible in terms of functions of $x+iy'$, in which $y'$ is less
than $\frac{1}{2} \pi$.

The hyperbolic functions are classed in the modern function-theory
of a complex variable as functions that are singly periodic with a
pure imaginary period, just as the circular functions are singly
periodic with a real period, and the elliptic functions are doubly
periodic with both a real and a pure imaginary period.

\medskip Multiple Values of Inverse Hyperbolic Functions.---It follows
from the periodicity of the direct functions that the inverse
functions $\sinh^{-1} m$ and $\cosh^{-1} m$ have each an indefinite
number of values arranged in a series at intervals of $2i\pi$. That
particular value of $\sinh^{-1} m$ which has the coefficient of $i$
not greater than $\frac{1}{2}\pi$ nor less than $-\frac{1}{2}\pi$ is
called the principal value of $\sinh^{-1} m$; and that particular
value of $\cosh^{-1} m$ which has the coefficient of $i$ not greater
than $\pi$ nor less than zero is called the principal value of
$\cosh^{-1} m$. When it is necessary to distinguish between the
general value and the principal value the symbol of the former will
be capitalized; thus
\begin{gather*}
\text{Sinh}^{-1} m = \sinh^{-1} m + 2ir\pi,\quad
\text{Cosh}^{-1} m = \cosh^{-1} m + 2ir\pi,\\
\text{Tanh}^{-1} m = \tanh^{-1} m +  ir\pi,
\end{gather*}
in which $r$ is any integer, positive or negative.%
\index{Ambiguity of value}\index{Anti-hyperbolic functions}%
\index{Multiple values}

\medskip Complex Roots of Cubic Equations.---It is well known that when
the roots of a cubic equation are all real they are expressible in
terms of circular functions. Analogous hyperbolic expressions are
easily found when two of the roots are complex. Let the cubic, with
second term removed, be written
\begin{equation*}
x^3 \pm 3bx = 2c.
\end{equation*}

Consider first the case in which $b$ has the positive sign. Let
$x = r \sinh u$, substitute, and divide by $r^3$, then
\begin{equation*}
\sinh^3 u + \frac{3b}{r^2} \sinh u = \frac{2c}{r^3}.
\end{equation*}

Comparison with the formula $\sinh^3 u + \frac{3}{4} \sinh u =
\frac{1}{4} \sinh 3u$ gives
\begin{gather*}
\frac{3b}{r^2} = \frac{3}{4},\quad
\frac{2c}{r^3} = \frac{\sinh 3u}{4},\\
\intertext{whence}
r = 2b^{\frac{1}{2}},\quad
\sinh 3u = \frac{c}{b^{\frac{3}{2}}},\quad
u = \frac{1}{3} \sinh^{-1} \frac{c}{b^{\frac{3}{2}}}; \\
\intertext{therefore}
x = 2b^{\frac{1}{2}}
   \sinh \left(\frac{1}{3}\sinh^{-1}\frac{c}{b^{\frac{3}{2}}}
         \right),
\end{gather*}
in which the sign of $b^{\frac{1}{2}}$ is to be taken the same as
the sign of $c$. Now let the principal value of
$\sinh^{-1}\dfrac{c}{b^{\frac{3}{2}}}$, found from the tables, be
$n$; then two of the imaginary values are $n\pm 2i\pi$, hence the
three values of $x$ are $2b^{\frac{1}{2}} \sinh\dfrac{n}3$ and
$2b^{\frac{1}{2}} \sinh\left(\dfrac{n}{3} \pm \dfrac{2i\pi}{3}
\right)$. The last two reduce to $-b^{\frac{1}{2}}
\sinh\left(\dfrac{n}{3} \pm i\sqrt{3}\cosh\dfrac{n}{3} \right)$.

Next, let the coefficient of $x$ be negative and equal to $-3b$. It
may then be shown similarly that the substitution $x = r \sin
\theta$ leads to the three solutions
\begin{equation*}
-2b^{\frac{1}{2}} \sin\frac{n}{3},\quad
b^{\frac{1}{2}} \left(\sin\frac{n}{3} \pm
                      \sqrt{3}\cos\frac{n}{3}\right),\quad
\text{ where } n = \sin^{-1}\frac{c}{b^{\frac{3}{2}}}.
\end{equation*}
These roots are all real when $c \ngtr b^{\frac{3}{2}}$. If $c
> b^{\frac{3}{2}}$, the substitution $x = r\cosh u$ leads to the
solution
\begin{equation*}
x = 2b^{\frac{1}{2}} \cosh\left(\frac{1}{3}
                     \cosh^{-1}\frac{c}{b^{\frac{3}{2}}} \right),
\end{equation*}
which gives the three roots
\begin{equation*}
2b^{\frac{1}{2}} \cosh\frac{n}{3},\quad
-b^{\frac{1}{2}} \left( \cosh\frac{n}{3}
                        \pm i\sqrt{3}\sinh\frac{n}3 \right),\quad
\text{ wherein } n = \cosh^{-1}\frac{c}{b^{\frac{3}{2}}},
\end{equation*}
in which the sign of $b^{\frac{1}{2}}$ is to be taken the same as
the sign of $c$.

\index{Hyperbolic functions!applictions of|(}
\small \begin{enumerate}
\item[Prob.~85.] Show that the chart of $\cosh (x + iy)$ can be adapted
to $\sinh (x + iy)$, by turning through a right angle; also to $\sin
(x + iy)$.

\item[Prob.~86.] Prove the identity
\begin{equation*}
\tanh (x + iy) = \frac{\sinh 2x + i \sin 2y}{\cosh 2x + \cos 2y}.
\end{equation*}

\item[Prob.~87.] If $\cosh (x + iy) = a + ib$, be written in the
``modulus and amplitude'' form as $r(\cos\theta + i\sin \theta), = r
\exp i\theta$, then
\begin{align*}
r^2 = a^2 + b^2 &= \cosh^2 x = \sin^2 y = \cos^2 y - \sinh^2 x,\\
\tan \theta = \frac{b}{a} &= \tanh x \tan y.
\end{align*}%
\index{Amplitude!of complex number}\index{Modulus}

\index{Applications|(}
\item[Prob.~88.] Find the modulus and amplitude of $\sinh (x + iy)$.

\item[Prob.~89.] Show that the period of $\exp \dfrac{2\pi u}{a}$ is $ia$.

\item[Prob.~90.] When $m$ is real and $> 1$, $\cos^{-1} m = i
\cosh^{-1} m$, $\sin^{-1} m = \frac{\pi}2 - i \cosh^{-1} m$.

When $m$ is real and $< 1$, $\cosh^{-1} m = i \cos^{-1} m$.
\end{enumerate}\index{Complex numbers|)}%
\index{Hyperbolic functions!of complex numbers|)}%
\index{Numbers, complex|)} \normalsize

\chapter{The Catenary}\index{Catenary}\index{Physical problems|(}

A flexible inextensible string is suspended from two fixed points,
and takes up a position of equilibrium under the action of gravity.
It is required to find the equation of the curve in which it hangs.

Let $w$ be the weight of unit length, and $s$ the length of arc $AP$
measured from the lowest point $A$; then $ws$ is the weight of the
portion $AP$. This is balanced by the terminal tensions, $T$ acting
in the tangent line at $P$, and $H$ in the horizontal tangent.
Resolving horizontally and vertically gives
\begin{equation*}
T\cos\phi = H,\quad T\sin\phi = ws,
\end{equation*}
in which $\phi$ is the inclination of the tangent at $P$; hence
\begin{equation*}
\tan\phi = \frac{ws}{H} = \frac{s}{c},
\end{equation*}
where $c$ is written for $\dfrac{H}{w}$, the length whose weight is
the constant horizontal tension; therefore
\begin{gather*}
\frac{dy}{dx}=\frac{s}{c},\quad
\frac{ds}{dx}=\sqrt{1+\frac{s^2}{c^2}},\quad
\frac{dx}{c}=\frac{ds}{\sqrt{s^2+c^2}}, \\
\frac{x}{c}=\sinh^{-1}\frac{s}{c},\quad
\sinh\frac{x}{c}=\frac{s}{c}=\frac{dy}{dx},\quad
\frac{y}{c}=\cosh\frac{x}{c},
\end{gather*}
which is the required equation of the catenary, referred to an axis
of $x$ drawn at a distance $c$ below $A$.

The following trigonometric method illustrates the use of the
gudermanian: The ``intrinsic equation,'' $s = c\tan\phi$, gives $ds
= c\sec^2\phi\, d\phi$; hence $dx = ds\cos\phi = c\sec\phi\, d\phi$;
$dy = ds\sin\phi = c\sec\phi\tan\phi\, d\phi$; thus $x =
c\gd^{-1}\phi, y = c\sec\phi$; whence $\frac{y}{c} = \sec\phi =
\sec\gd\frac{x}{c} = \cosh \frac{x}{c}$; and $\frac{s}{c} = \tan
\gd\frac{x}{c} = \sinh\frac{x}{c}$.%
\index{Anti-gudermanian}\index{Differential equation}%
\index{Gudermanian!function}\index{Intrinsic equation}

\medskip Numerical Exercise.---A chain whose length is 30 feet is
suspended from two points 20 feet apart in the same horizontal; find
the parameter $c$, and the depth of the lowest point.

The equation $\frac{s}{c} = \sinh\frac{x}{c}$ gives $\frac{15}{c} =
\sinh\frac{10}{c}$, which, by putting $\frac{10}{c} = z$, may be
written $1.5 z = \sinh z$. By examining the intersection of the
graphs of $y = \sinh z$, $y = 1.5 z$, it appears that the root of
this equation is $z = 1.6$, nearly. To find a closer approximation
to the root, write the equation in the form $f(z) = \sinh z - 1.5 z
= 0$, then, by the tables,
\begin{align*}
  f(1.60) &= 2.3756 - 2.4000 = -.0244,  \\
  f(1.62) &= 2.4276 - 2.4300 = -.0024,  \\
  f(1.64) &= 2.4806 - 2.4600 = +.0206;
\end{align*}
whence, by interpolation, it is found that $f(1.6221) = 0$, and $z =
1.6221$, $c = \frac{10}{z} = 6.1649$. The ordinate of either of the
fixed points is given by the equation
\begin{equation*}
  \frac{y}{c} = \cosh\frac{x}{c} = \cosh\frac{10}{c} =
                \cosh 1.6221 = 2.6306,
\end{equation*}
from tables; hence $y = 16.2174$, and required depth of the vertex $
= y - c = 10.0525$ feet.\footnote{See a similar problem in Chap.~1,
Art.~7.}\index{Interpolation}

\small \begin{enumerate}
\item[Prob.~91.] In the above numerical problem, find the inclination
of the terminal tangent to the horizon.\index{Equations!Numerical}

\item[Prob.~92.] If a perpendicular $MN$ be drawn from the foot of
the ordinate to the tangent at $P$, prove that $MN$ is equal to the
constant $c$, and that $NP$ is equal to the arc $AP$. Hence show
that the locus of $N$ is the involute of the catenary, and has the
property that the length of the tangent, from the point of contact
to the axis of $x$, is constant. (This is the characteristic
property of the tractory).\index{Involute!of catenary}%
\index{Tractory}

\item[Prob.~93.] The tension $T$ at any point is equal to the weight
of a portion of the string whose length is equal to the ordinate $y$
of that point.

\item[Prob.~94.] An arch in the form of an inverted catenary\footnote{
For the theory of this form of arch, see ``Arch'' in the
Encyclop\ae{}dia Britannica.} is $30$ feet wide and $10$ feet high;
show that the length of the arch can be obtained from the equations
$\cosh z - \frac{2}{3}z = i$, $2s = \dfrac{30}{z} \sinh z$.%
\index{Arch}
\end{enumerate} \normalsize

\chapter{Catenary of Uniform Strength.}\index{Catenary!of uniform
strength}

If the area of the normal section at any point be made proportional
to the tension at that point, there will then be a constant tension
per unit of area, and the tendency to break will be the same at all
points. To find the equation of the curve of equilibrium under
gravity, consider the equilibrium of an element $PP'$ whose length
is $ds$, and whose weight is $g\rho\omega\, ds$, where $\omega$ is
the section at $P$, and $\rho$ the uniform density. This weight is
balanced by the difference of the vertical components of the
tensions at $P$ and $P'$, hence
\begin{align*}
d(T\sin\phi) &= g\rho\omega\, ds,\\
d(T\cos\phi) &= 0;
\end{align*}
therefore $T\cos\phi = H$, the tension at the lowest point, and $T =
H \sec \phi$. Again, if $\omega_0$ be the section at the lowest
point, then by hypothesis $\frac{\omega}{\omega_0} = \frac{T}{H} =
\sec \phi$, and the first equation becomes
\begin{gather*}
Hd(\sec\phi\sin\phi) = g\rho\omega_0\sec\phi\, ds, \\
\intertext{or}
cd\tan\phi = \sec\phi\, ds,
\end{gather*}
where $c$ stands for the constant $\dfrac{H}{g\rho\omega_0}$, the
length of string (of section $\omega_0$) whose weight is equal to
the tension at the lowest point; hence,
\begin{equation*}
ds = c \sec\phi\, d\phi,\quad \frac{s}{c} = \gd^{-1}\phi,
\end{equation*}
the intrinsic equation of the catenary of uniform strength.%
\index{Intrinsic equation}

Also
\begin{gather*}
dx = ds\cos\phi = c d\phi,\quad
dy = ds\sin\phi = c\tan\phi\, d\phi; \\
\intertext{hence}
x = c\phi,\quad y = c \log\sec\phi,
\intertext{and thus the Cartesian equation is}
\frac{y}{c} = \log \sec\frac{x}{c},
\end{gather*}
in which the axis of $x$ is the tangent at the lowest
point.\index{Differential equation}

\small \begin{enumerate}
\item[Prob.~95.] Using the same data as in Art.~31, find the parameter
$c$ and the depth of the lowest point. (The equation $\dfrac{x}{c} =
\gd\dfrac{s}{c}$ gives $\dfrac{10}{c} = \gd\dfrac{15}{c}$, which, by
putting $\dfrac{15}{c} = z$, becomes $\gd{z}= \dfrac{2}{3}z$. From
the graph it is seen that $z$ is nearly $1.8$. If $f(z) =
\gd{z}-\dfrac{2}{3}z$, then, from the tables of the gudermanian at
the end of this chapter,
\begin{align*}
f(1.80) & = 1.2432 - 1.2000 = +.0432,\\
f(1.90) & = 1.2739 - 1.2667 = +.0072,\\
f(1.95) & = 1.2881 - 1.3000 = -.0119,
\end{align*}
whence, by interpolation, $z = 1.9189$ and $c = 7.8170$. Again,
$\dfrac{y}{c} = \log_e{\sec{\dfrac{x}{c}}}$; but $\dfrac{x}{c} -
\dfrac{10}{c} = 1.2793$; and $1.2793 \text{ radians } =
73^{\circ}\,17'\,55''$; hence $y = 7.8170 \times .41914 \times
2.3026 = 7.5443$, the required depth.)\index{Interpolation}

\item[Prob.~96.] Find the inclination of the terminal tangent.

\item[Prob.~97.] Show that the curve has two vertical asymptotes.

\item[Prob.~98.] Prove that the law of the tension $T$, and of the
section $\omega$, at a distance $s$, measured from the lowest point
along the curve, is
\begin{equation*}
\frac{T}{H} = \frac{\omega}{\omega_0} = \cosh{\frac{c}{h}};
\end{equation*}
and show that in the above numerical example the terminal section is
$3.48$ times the minimum section.\index{Equations!Numerical}

\item[Prob.~99.] Prove that the radius of curvature is given by $\rho =
c \cosh{\dfrac{s}{f}}$. Also that the weight of the arc $s$ is given
by $W = H \sinh{\dfrac{s}{f}}$, in which $s$ is measured from the
vertex.
\end{enumerate} \normalsize

\chapter{The Elastic Catenary.}%
\index{Catenary!Elastic}\index{Curvature}

An elastic string of uniform section and density in its natural
state is suspended from two points. Find its equation of
equilibrium.

Let the element $d\sigma$ stretch into $ds$; then, by Hooke's law,
$ds = d\sigma(1 + \lambda T)$, where $\lambda$ is the elastic
constant of the string; hence the weight of the stretched element
$ds = g\rho\omega\, d\sigma = \dfrac{g\rho\omega\, ds}{(1 + \lambda
T)}$. Accordingly, as before,
\begin{align*}
d(T\sin{\phi}) & = \frac{g\rho\omega\, ds}{(1 + \lambda T)},\\
\intertext{and}
T\cos{\phi} & = H = g\rho\omega c,\\
\intertext{hence}
cd(\tan{\phi}) & = \frac{ds}{(1 + \mu\sec{\phi})},
\intertext{in which $\mu$ stands for $\lambda H$, the extension at
the lowest point; therefore}
ds &= c(\sec^2\phi + \mu\sec^3\phi)d\phi, \\
\frac{s}{c} &= \tan\phi + \frac{1}{2}\mu(\sec\phi\tan\phi
                       + \gd^{-1}\phi), \tag*{[prob.~20, p.~37}
\end{align*}
which is the intrinsic equation of the curve, and reduces to that of
the common catenary when $\mu = 0$. The coordinates $x$, $y$ may be
expressed in terms of the single parameter $\phi$ by putting
\begin{align*}
dx &= ds\cos\phi = c(\sec\phi + \mu\sec^2\phi)d\phi, \\
dy &= ds\sin\phi = c(\sec^2\phi + \mu\sec^3\phi)\sin\phi\, d\phi. \\
\intertext{Whence}
\frac{x}{c} &= \gd^{-1}\phi + \mu\tan\phi,\quad
\frac{y}{c} = \sec\phi + \frac{1}{2}\mu\tan^2\phi.
\end{align*}\index{Intrinsic equation}

These equations are more convenient than the result of eliminating
$\phi$, which is somewhat complicated.

\chapter{The Tractory.}%
\index{Arch}\index{Tractory}

[Note.\footnote{This curve is used in Schiele's anti-friction pivot
(Minchin's Statics, Vol.~I, p.~242); and in the theory of the skew
circular arch, the horizontal projection of the joints being a
tractory. (See ``Arch,'' Encyclop�dia Britannica.) The equation
$\phi=\gd\frac{t}{c}$ furnishes a convenient method of plotting the
curve.}]

To find the equation of the curve which possesses the property that
the length of the tangent from the point of contact to the axis of
$x$ is constant.

\begin{center}
\includegraphics[width=40mm]{fig11.png}
\end{center}

Let $PT$, $P'T'$ be two consecutive tangents such that $PT = P'T' =
c$, and let $OT = t$; draw $TS$ perpendicular to $P'T'$; then if
$PP' = ds$, it is evident that $ST'$ differs from $ds$ by an
infinitesimal of a higher order. Let $PT$ make an angle $\phi$ with
$OA$, the axis of $y$; then (to the first order of infinitesimals)
$PT d\phi = TS = TT'\cos\phi$; that is,
\begin{gather*}
c\,d\phi = \cos\phi\, dt,\quad t = c\,\gd^{-1}\phi, \\
x = t-c\,\sin\phi = c(\gd^{-1}\phi-\sin\phi),\quad y = c\,\cos\phi.
\end{gather*}%
\index{Anti-gudermanian}\index{Differential equation}

This is a convenient single-parameter form, which gives all values
of $x$, $y$ as $\phi$ increases from $0$ to $\frac{1}{2}\pi$. The
value of $s$, expressed in the same form, is found from the relation
\begin{equation*}
ds = ST' = dt\,\sin\phi = c\tan\,\phi\,d\phi,\quad
 s = c\,\log_e\sec\phi.
\end{equation*}

At the point $A$, $\phi=0$, $x=0$, $s=0$, $t=0$, $y=c$.  The
Cartesian equation, obtained by eliminating $\phi$, is
\begin{equation*}
\frac{x}{c}= \gd^{-1}\left(\cos^{-1}\frac{y}{c}\right) -
             \sin\left(\cos^{-1}\frac{y}{c}\right) =
             \cosh^{-1}\frac{c}{y} - \sqrt{1-\frac{y^2}{c^2}}.
\end{equation*}

If $u$ be put for $\dfrac{t}{c}$, and be taken as independent
variable, $\phi=\gd u$, $\dfrac{x}{c} = u - \tanh u$, $\dfrac{y}{c}
= \sech u$, $\dfrac{s}{c} = \log\cosh u.$

\small \begin{enumerate}
\item[Prob.~100.] Given $t = 2c$, show that $\phi = 74^\circ\, 35'$,
$s = 1.3249c$, $y = .2658c$, $x = 1.0360c.$ At what point is $t =
c$?

\item[Prob.~101.] Show that the evolute of the tractory is the
catenary. (See Prob.~92.)\index{Evolute of tractory}

\item[Prob.~102.] Find the radius of curvature of the tractory in
terms of $\phi$; and derive the intrinsic equation of the involute.%
\index{Intrinsic equation}\index{Involute!of tractory}
\end{enumerate} \normalsize

\chapter{The Loxodrome.}\index{Curvature}\index{Loxodrome}

On the surface of a sphere a curve starts from the equator in a
given direction and cuts all the meridians at the same angle. To
find its equation in latitude-and-longitude coordinates:

\begin{center}
\includegraphics[width=45mm]{fig12.png}
\end{center}

Let the loxodrome cross two consecutive meridians $AM$, $AN$ in the
points $P$, $Q$; let $P\!R$ be a parallel of latitude; let $O\!M =
x$, $M\!P = y$, $M\!N' = dx$, $RQ = dy$, all in radian measure; and
let the angle $M\!O\!P = RPQ = \alpha$; then
\begin{equation*}
\tan\alpha = \frac{RQ}{P\!R}\text{, but } P\!R = M\!N\cos
M\!P,\footnotemark
\end{equation*}
\footnotetext{Jones, Trigonometry (Ithaca, 1890), p.~185.}%
\index{Jones' Trigonometry} hence $dx\,\tan\alpha = dy\,\sec y$, and
$x\tan\alpha = \gd^{-1}y$, there being no integration-constant since
$y$ vanishes with $x$; thus the required equation is
\begin{equation*}
y = \gd(x\,\tan\alpha).
\end{equation*}%
\index{Anti-gudermanian}\index{Differential equation}

To find the length of the arc $OP$: Integrate the equation
\begin{equation*}
ds = dy\,\csc\alpha, \text{ whence } s = y\,\csc\alpha.
\end{equation*}

To illustrate numerically, suppose a ship sails northeast, from a
point on the equator, until her difference of longitude is
$45^\circ$, find her latitude and distance:

Here $\tan\alpha = 1$, and $y = \gd x = \gd\frac{1}{4}\pi = \gd
(.7854) = .7152$ radians; $s = y\sqrt{2} = 1.0114$ radii. The
latitude in degrees is $40.980$.

If the ship set out from latitude $y_1$, the formula must be
modified as follows: Integrating the above differential equation
between the limits $(x_1, y_1)$ and $(x_2, y_2)$ gives
\begin{equation*}
(x_2 - x_1)\tan\alpha = \gd^{-1}y_2 - \gd^{-1}y_1;
\end{equation*}
hence $\gd^{-1}y_2 = \gd^{-1}y_1 + (x_2 - x_1)\tan\alpha$, from
which the final latitude can be found when the initial latitude and
the difference of longitude are given. The distance sailed is equal
to $(y_2 - y_1)\csc\alpha$ radii, a radius being $60 \times
\frac{180}{\pi}$ nautical miles.\index{Gudermanian!function}

\medskip Mercator's Chart.---In this projection the meridians are
parallel straight lines, and the loxodrome becomes the straight line
$y' = x\tan\alpha$, hence the relations between the coordinates of
corresponding points on the plane and sphere are $x' = x$, $y' =
\gd^{-1}y$. Thus the latitude $y$ is magnified into $\gd^{-1}y$,
which is tabulated under the name of ``meridional part for latitude
$y$''; the values of $y$ and of $y'$ being given in minutes. A chart
constructed accurately from the tables can be used to furnish
graphical solutions of problems like the one proposed above.%
\index{Chart!Mercator's}\index{Mercator's Chart}

\small \begin{enumerate}
\item[Prob.~103.] Find the distance on a rhumb line between the
points ($30^\circ$ N, $20^\circ$ E) and ($30^\circ$ S, $40^\circ$
E).\index{Rhumb line}
\end{enumerate} \normalsize

\chapter{Combined Flexure and Tension.}%
\index{Beams, flexure of}\index{Flexure of Beams}

A beam that is built-in at one end carries a load $P$ at the other,
and is also subjected to a horizontal tensile force $Q$ applied at
the same point; to find the equation of the curve assumed by its
neutral surface: Let $x, y$ be any point of the elastic curve,
referred to the free end as origin, then the bending moment for this
point is $Qy - Px$. Hence, with the usual notation of the theory of
flexure,\footnote{Merriman, Mechanics of Materials (New York, 1895),
pp.~70--77, 267--269.}
\begin{gather*}
EI\frac{d^2y}{dx^2} = Qy - Px,\quad
  \frac{d^2y}{dx^2} = n^2(y - mx),
  \tag*{$\left[ m = \dfrac{P}{Q}\right.,\ n^2=\dfrac{Q}{EI}$.} \\
\intertext{which, on putting $y - mx = u$, and $\dfrac{d^2y}{dx^2} =
\dfrac{d^2u}{dx^2}$, becomes}
\frac{d^2u}{dx^2} = n^2u, \\
\intertext{whence}
u = A \cosh nx + B \sinh nx, \tag*{[probs.~28, 30} \\
\intertext{that is,}
y = mx + A \cosh nx + B \sinh nx.
\end{gather*}

The arbitrary constants $A$, $B$ are to be determined by the
terminal conditions.\index{Terminal conditions} At the free end $x =
0$, $y = 0$; hence $A$ must be zero, and
\begin{align*}
y &= mx + B \sinh nx, \\
\frac{dy}{dx} &= m+nB \cosh nx; \\
\intertext{but at the fixed end, $x = l$, and $\dfrac{dy}{dx} = 0$,
hence}
B &= -\frac{m}{n} \cosh nl, \\
\intertext{and accordingly}
y &= mx - \frac{m \sinh nx}{n \cosh nl}.
\end{align*}

To obtain the deflection of the loaded end, find the ordinate of the
fixed end by putting $x = l$, giving
\begin{equation*}
\text{deflection} = m(l - \frac{1}{n}\tanh nl),
\end{equation*}\index{Deflection of beams}

\small \begin{enumerate}
\item[Prob.~104.] Compute the deflection of a cast-iron beam,
$2 \times 2$ inches section, and $6$ feet span, built-in at one end
and carrying a load of $100$ pounds at the other end, the beam being
subjected to a horizontal tension of $8000$ pounds. [In this case $I
= \frac{4}{3}, E = 15 \times 10^6, Q = 8000, P = 100$; hence $n =
\frac{1}{50}, m = \frac{1}{80}$, deflection $= \frac{1}{80}(72 - 50
\tanh 1.44) = \frac{1}{80}(72 - 44.69) = .341$ inches.]

\item[Prob.~105.] If the load be uniformly distributed over the beam,
say $w$ per linear unit, prove that the differential equation is
\begin{equation*}
  EI \frac{d^2 y}{dx^2} = Qy - \tfrac{1}{2}wx^2, \text{ or }
     \frac{d^2 y}{dx^2} = n^2(y - mx^2),
\end{equation*}
and that the solution is $y = A \cosh nx + B \sinh nx + mx^2 +
\dfrac{2m}{n^2}$. Show also how to determine the arbitrary
constants.
\end{enumerate}\index{Distributed load} \normalsize

\chapter{Alternating Currents.}%
\index{Alternating currents}%
\index{Complex numbers!Applications of|(}%
\index{Currents, alternating}

[Note.\footnote{See references in foot-note Art.~27.}]

In the general problem treated the cable or wire is regarded as
having resistance, distributed capacity, self-induction, and
leakage; although some of these may be zero in special cases.%
\index{Self-induction of conductor} The line will also be considered
to feed into a receiver circuit of any description; and the general
solution will include the particular cases in which the receiving
end is either grounded or insulated. The electromotive force may,
without loss of generality, be taken as a simple harmonic function
of the time, because any periodic
function can be expressed in a Fourier series of simple harmonics.%
\footnote{Chapter V, Art.~8.} The E.M.F.\ and the current, which may
differ in phase by any angle, will be supposed to have given values
at the terminals of the receiver circuit; and the problem then is to
determine the E.M.F.\ and current that must be kept up at the
generator terminals; and also to express the values of these
quantities at any intermediate point, distant $x$ from the receiving
end; the four line-constants being supposed known, viz.:
\begin{verse}
  $R$ = resistance, in ohms per mile,  \\
  $L$ = coefficient of self-induction, in henrys per mile,  \\
  $C$ = capacity, in farads per mile,  \\
  $G$ = coefficient of leakage, in mhos per mile.%
\footnote{Kennelly denotes these constants by $r$, $l$, $c$, $g$.
Steinmetz writes $s$ for $\omega L$, $\kappa$ for $\omega C$,
$\theta$ for $G$, and he uses $C$ for current.}
\end{verse}%
\index{Capacity of conductor}\index{Electromotive force}%
\index{Fourier series}\index{Phase angle}%
\index{Resistance of conductor}

It is shown in standard works%
\footnote{Thomson and Tait, Natural Philosophy, Vol.~I. p.~40;
Rayleigh, Theory of Sound, Vol.~I. p.~20; Bedell and Crehore,
Alternating Currents, p.~214.}%
\index{Bedel and Crehore's alternating currents}%
\index{Rayleigh's Theory of Sound} that if any simple harmonic
function $a \sin(\omega t + \theta)$ be represented by a vector of
length $a$ and angle $\theta$, then two simple harmonics of the same
period $\dfrac{2\pi}{\omega}$, but having different values of the
phase-angle $\theta$, can be combined by adding their representative
vectors.\index{Vectors} Now the E.M.F. and the current at any point
of the circuit, distant $x$ from the receiving end, are of the form
\begin{equation}
e = e_1\sin{(\omega t + \theta)},\quad
i = i_1\sin{(\omega t + \theta')}, \tag{64}
\end{equation}
in which the maximum values $e_1$, $i_1$, and the phase-angles
$\theta$, $\theta'$, are all functions of $x$. These simple
harmonics will be represented by the vectors
$e_1\underline{/\theta}$, $i_1\underline{/\theta'}$; whose numerical
measures are the complexes $e_1(\cos\theta + j\sin\theta)$\footnote{
In electrical theory the symbol $j$ is used, instead of $i$, for
$\sqrt{-1}$.}, $i_1(\cos{\theta'} + j\sin{\theta'})$, which will be
denoted by $\bar{e}$, $\bar{\imath}$. The relations between
$\bar{e}$ and $\bar{\imath}$ may be obtained from the ordinary
equations\footnote{Bedell and Crehore, Alternating Currents, p.~181.
The sign of $dx$ is changed, because $x$ is measured from the
receiving end. The coefficient of leakage, $G$, is usually taken
zero, but is here retained for generality and symmetry.}
\begin{equation}
\frac{di}{dx} = Ge + C\frac{de}{dt},\quad
\frac{de}{dx} = Ri + L\frac{di}{dt}; \tag{65}
\end{equation}
for, since $\dfrac{de}{dt} = \omega e_1\cos{(\omega t + \theta)} =
\omega e_1 \sin{(\omega t + \theta + \frac{1}{2}\pi)}$, then
$\dfrac{de}{dt}$ will be represented by the vector $\omega
e_1\underline{/\theta + \frac{1}{2}\pi}$; and $\dfrac{di}{dx}$ by
the sum of the two vectors $Ge_1\underline{/\theta}, C\omega
e_1\underline{/\theta + \frac{1}{2}\pi}$; whose numerical measures
are the complexes $G\bar{e}$, $j\omega C\bar{e}$; and similarly for
$\dfrac{de}{dx}$ in the second equation; thus the relations between
the complexes $\bar{e}$, $\bar{\imath}$ are
\begin{equation}
\frac{d\bar{\imath}}{dx} = (G + j\omega C)\bar{e},\quad
\frac{d\bar{e}}{dx} = (R + j\omega L)\bar{\imath}.
  \tag*{(66)\footnotemark}
\end{equation}
\footnotetext{These relations have the advantage of not involving
the time. Steinmetz derives them from first principles without using
the variable $t$. For instance, he regards $R+j\omega L$ as a
generalized resistance-coefficient, which, when applied to $i$,
gives an E.M.F., part of which is in phase with $i$, and part in
quadrature with $i$. Kennelly calls $R + j\omega L$ the conductor
impedance; and $G + j\omega C$ the dielectric admittance; the
reciprocal of which is the dielectric impedance.}%
\index{Admittance of dielectric}\index{Impedance}%
\index{Kennelly's chart}\index{Operators, generalized}%
\index{Steinmetz on alternating currents}

Differentiating and substituting give
\begin{equation}
\left. \begin{aligned}
\frac{d^2\bar{e}}{dx^2}      &=
                          (R + j\omega L)(G + j\omega C)\bar{e}, \\
\frac{d^2\bar{\imath}}{dx^2} &=
                          (R + j\omega L)(G + j\omega C)\bar{\imath},
\end{aligned}
\right\} \tag{67}
\end{equation}
and thus $\bar{e}, \bar{\imath}$ are similar functions of $x$, to be
distinguished only by their terminal values.

It is now convenient to define two constants $m$, $m_1$ by the
equations\footnote{The complex constants $m$, $m_1$ are written $z,
y$ by Kennelly; and the variable length $x$ is written $L_2$.
Steinmetz writes $v$ for $m$.}%
\index{Kennelly on alternating currents}
\begin{equation}
m^2 = (R+j\omega L)(G + j\omega C),\quad
m_1 = \frac{m}{(G + j\omega C)}; \tag{68}
\end{equation}
and the differential equations may then be written
\begin{equation}
\frac{d^2\bar{e}}{dx^2}      = m^2\bar{e},\quad
\frac{d^2\bar{\imath}}{dx^2} = m^2\bar{\imath}, \tag{69}
\end{equation}
the solutions of which are\footnote{See Art.~14, Probs.~28--30; and
Art.~27, foot-note.}
\begin{equation*}
\bar{e} = A \cosh mx + B\ \sinh mx,\quad
\bar{\imath} = A' \cosh mx + B' \sinh mx,
\end{equation*}
wherein only two of the four constants are arbitrary; for
substituting in either of the equations (66), and equating
coefficients, give
\begin{gather*}
(G + j\omega C)A = mB',\quad (G + j\omega C)B = mA', \\
\intertext{whence}
B' = \frac{A}{m_1},\quad A' = \frac{B}{m_1}.
\end{gather*}\index{Differential equation}

Next let the assigned terminal values of $\bar{e}$, $\bar{\imath}$,
at the receiver, be denoted by $\bar{E}, \bar{I}$; then putting $x =
0$ gives $\bar{E} = A, \bar{I} = A'$, whence $B = m_1\bar{I}, B' =
\dfrac{\bar{E}}{m_1}$; and thus the general solution is
\begin{equation}
\left. \begin{aligned}
\bar{e}      &= \bar{E}\cosh mx + m_1\bar{I}\sinh mx,\\
\bar{\imath} &= \bar{I}\cosh mx + \frac{I}{m_1}\bar{E}\sinh mx.
\end{aligned}
\right\} \tag{70}
\end{equation}

If desired, these expressions could be thrown into the ordinary
complex form $X + jY, X' + jY'$, by putting for the letters their
complex values, and applying the addition-theorems for the
hyperbolic sine and cosine. The quantities $X, Y, X', Y'$ would then
be expressed as functions of $x$; and the representative vectors of
$e, i$, would be $e_1\underline{/\theta}, i_1\underline{/\theta'}$,
where ${e_{1}}^{2} = X^2 + Y^2, i^2 = {X'}^2 + {Y'}^2, \tan{\theta}
= \dfrac{Y}{X}, \tan{\theta'} = \dfrac{Y'}{X'}$.%
\index{Argand diagram}

For purposes of numerical computation, however, the formulas ($70$)
are the most convenient, when either a chart,\footnote{Art.~30,
footnote.} or a table,\footnote{See Table II.} of $\cosh{u}$,
$\sinh{u}$, is available, for complex values of $u$.%
\index{Chart!of hyperbolic functions}

\small \begin{enumerate}
\item[Prob.~106.\footnotemark]\footnotetext{The data for this example
are taken from Kennelly's article (l.~c., p.~38).}%
\index{Conductor resistance and impedance} Given the four
line-constants: $R$ = 2 ohms per mile, $L$ = 20 millihenrys per
mile, $C$ = $\frac{1}{2}$ microfarad per mile, $G$ = 0; and given
$\omega$, the angular velocity of E.M.F. to be 2000 radians per
second;\index{Electromotive force} then
\begin{align*}
\omega L &= 40 \text{ ohms, conductor reactance per mile};\\
R + j\omega L &= 2 + 40j \text{ ohms, conductor impedance per mile};\\
\omega C &= .001 \text{ mho, dielectric susceptance per mile};\\
G + j\omega C &= .001j \text{ mho, dielectric admittance per mile};\\
(G + j\omega C)^{-1} &= -1000j \text{ ohms, dielectric impedance per
  mile};\\
m^2 &= (R + j\omega L)(G + j\omega C) = .04 + .002j, \\
&\qquad \text{which is the measure of }
  .04005\underline{/177^{\circ}8'}; \\
\intertext{therefore}
m &= \text{ measure of } .2001\underline{/88^{\circ}34'}
   = .0050 + .2000j, \\
&\qquad \text{an abstract coefficient per mile, of
      dimensions } [\mathrm{length}]^{-1}, \\
m_1 &= \dfrac{m}{(G + j\omega C)} = 200 - 5j \text{ ohms}.
\end{align*}\index{Reactance of conductor}%
\index{Suceptance of dielectric}

\indent Next let the assigned terminal conditions at the receiver
be: $I = 0$ (line insulated)\index{Terminal conditions}; and $E =
1000$ volts, whose phase may be taken as the standard (or zero)
phase; then at any distance $x$, by (70),
\begin{align*}
\bar{e} &= E\cosh{mx}, & \bar{\imath} &= \frac{E}{m_1}\sinh{mx},
\end{align*}
in which $mx$ is an abstract complex.

\indent Suppose it is required to find the E.M.F. and current that
must be kept up at a generator $100$ miles away; then
\begin{gather*}
\bar{e} = 1000 \cosh(.5 + 20 j),\quad
\bar{\imath} = 200 (40 - j)^{-1} \sinh(.5 + 20j), \\
\intertext{but, by Prob.~89,}
\begin{aligned}
\cosh(.5 + 20 j) &= \cosh(.5 + 20 j - 6\pi j) \\
                 &= \cosh(.5 + 1.15 j) = .4600 + .4750j
\end{aligned}
\end{gather*}
obtained from Table II, by interpolation between $\cosh (.5 +
1.1j)$ and $\cosh (.5 + 1.2j)$; hence
\begin{equation*}
\bar{e} = 460 + 475j = e_1 (\cos \theta + j \sin\theta),
\end{equation*}
where $\log \tan \theta = \log 475 - \log 460 = .0139$, $\theta =
45^\circ\: 55'$, and $e_1 = 460 \sec \theta = 661.2$ volts, the
required E.M.F.\index{Interpolation}

\smallskip \indent Similarly $\sinh (.5 + 20j) =
\sinh (.5 + 1.15j) = .2126 + 1.0280j$, and hence
\begin{align*}
\bar{\imath} = \frac{200}{1601}(40+j)(.2126+1.028j)
            &= \frac{1}{1601}(1495+8266j) \\
            &= i_1(\cos\theta' + j\sin\theta'),
\end{align*}
where $\log \tan \theta' = 10.7427$, $\theta' = 79^\circ\, 45'$,
$i_1 = 1495 \sec \dfrac{\theta'}{1601} = 5.25$ amperes, the phase
and magnitude of required current.\index{Phase angle}

\indent Next let it be required to find $e$ at $x = 8$; then
\begin{equation*}
\bar{e}= 1000 \cosh (.04 + 1.6j) = 1000j \sinh (.04 + .03j),
\end{equation*}
by subtracting $\frac{1}{2}\pi j$, and applying
page~\pageref{periodicity of hyperbolic functions}. Interpolation
between $\sinh (0 + 0j)$ and $\sinh (0 +.1j)$ gives
\begin{align*}
\sinh ( 0 + .03j) &= .00000 + .02995j. \\
\intertext{Similarly}
\sinh (.1 + .03j) &= .10004 + .03004j. \\
\intertext{Interpolation between the last two gives}
\sinh (.04 + .03j) &= .04002 + .02999j.
\end{align*}
Hence $\bar{e} = j(40.02 + 29.99j) = -29.99 + 40.02j = e_1
(\cos\theta + j\sin\theta)$, where $\log \tan \theta = .12530$,
$\theta = 126^\circ\, 51'$, $e_1 = -29.99 \sec 126^\circ\, 51' =
50.01$ volts.\index{Interpolation}

\indent Again, let it be required to find $e$ at $x = 16$; here
\begin{gather*}
\bar{e} = 1000 \cosh (.08 + 3.2j) = -1000 \cosh (.08 + .06j), \\
\intertext{but}
\cosh (0 + .06j) = .9970 + 0j,\
  \cosh (.1 + .06j) = 1.0020 + .006j; \\
\intertext{hence}
\cosh (.08 +.06j) = 1.0010 + .0048j, \\
\intertext{and}
\bar{e} = -1001 + 4.8j = e_1(\cos\theta + j\sin\theta),
\end{gather*}
where $\theta = 180^\circ\, 17'$, $e_1 = 1001$ volts. Thus at a
distance of about 16 miles the E.M.F. is the same as at the
receiver, but in opposite phase. Since $\bar{e}$ is proportional to
$\cosh (.005 + .2j)x$, the value of $x$ for which the phase is
exactly $180^\circ$ is $\frac{\pi}{.2} = 15.7$. Similarly the phase
of the E.M.F.\ at $x = 7.85$ is $90^\circ$. There is agreement in
phase at any two points whose distance apart is $31.4$ miles.

\indent In conclusion take the more general terminal conditions in
which the line feeds into a receiver circuit, and suppose the
current is to be kept at $50$ amperes, in a phase $40^\circ$ in
advance of the electromotive force; then $\bar{I} 50(\cos 40^\circ +
\sin 40^\circ) = 38.30 + 32.14j$, and substituting the constants in
(70) gives
\begin{align*}
\bar{c}
&= 1000 \cosh (.005 + .2j)x + (7821 + 6236j) \sinh (.005 + .2j)x  \\
&= 460 + 475j - 4748 + 9366j = -4288 + 9841j
 = e_1(\cos\theta + j\sin\theta),
\end{align*}
where $\theta = 113^\circ\: 33'$, $e_1 = 10730$ volts, the E.M.F.\
at sending end. This is 17 times what was required when the other
end was insulated.\index{Terminal conditions}

\item[Prob.~107.] If $L = 0$, $G = 0$, $I = 0$; then $m = (1 + j)n$,
$m_1 = (1 + j)n_1$ where $n^2 = \dfrac{\omega RC}{2}$, $n_1^2 =
\dfrac{R}{2\omega C}$; and the solution is
\begin{align*}
  e_1 &= \frac{1}{\sqrt{2}} E\sqrt{\cosh 2nx + \cos 2nx},
  &\tan \theta  &= \tan nx \tanh nx,
\\
  i_1 &= \frac{1}{2n_1}     E\sqrt{\cosh 2nx - \cos 2nx},
  &\tan \theta' &= \tan nx \coth nx .
\end{align*}

\item[Prob.~108.] If self-induction and capacity be zero, and the
receiving end be insulated, show that the graph of the electromotive
force is a catenary if $G \neq 0$, a line if $G = 0$.

\item[Prob.~109.] Neglecting leakage and capacity, prove that the
solution of equations (66) is $\bar{\imath} = \bar{I}$, $\bar{c} =
\bar{E} + (R + j\omega L)\bar{I}x$.

\item[Prob.~110.] If $x$ be measured from the sending end, show how
equations (65), (66) are to be modified; and prove that
\begin{equation*}
  \bar{e} = \bar{E}_0 \cosh mx - m_1\bar{I}_0 \sinh mx,\
  \bar{\imath} = \bar{I}_0 \cosh mx - \frac{1}{m_1}\bar{E}_0 \sinh mx,
\end{equation*}
where $\bar{E}_0$, $\bar{I}_0$ refer to the sending end.
\end{enumerate}\index{Complex numbers!Applications of|)} \normalsize

\chapter{Miscellaneous Applications.}

\begin{enumerate}
\item[1.] The length of the arc of the logarithmic curve $y = a^x$
is $s = M(\cosh u + \log\tanh\frac{1}{2} u)$, in which $M =
\dfrac{1}{\log a}$, $\sinh u = \dfrac{y}{M}$.%
\index{Curvature}\index{Logarithmic!curve}

\item[2.] The length of arc of the spiral of Archimedes
$r = a\theta$ is $s = \frac{1}{4} a(\sinh 2u + 2u)$, where $\sinh u
= \theta$.\index{Spiral of Archimedes}

\item[3.] In the hyperbola $\frac{x^2}{a^2} - \frac{y^2}{b^2} = 1$
the radius of curvature is \\
$\rho = \dfrac{(a^2 \sinh^2 u + b^2 \cosh^2 u)^{\frac{3}{2}}}{ab}$;
in which $u$ is the measure of the sector $AOP$, i.e.\ $\cosh u =
\dfrac{x}{a}$, $\sinh u = \dfrac{y}{b}$.%
\index{Areas}\index{Hyperbola}

\item[4.] In an oblate spheroid, the superficial area of the zone
between the equator and a parallel plane at a distance $y$ is
$S=\dfrac{\pi b^2(\sinh 2u+2u)}{2e}$, wherein $b$ is the axial
radius, $e$ eccentricity, $\sinh u = \dfrac{ey}{p}$, and $p$
parameter of generating ellipse.\index{Spheroid, area of oblate}

\item[5.] The length of the arc of the parabola $y^2 = 2px$,
measured from the vertex of the curve, is $l = \frac{1}{4}p(\sinh 2u
+ 2u)$, in which $\sinh u = \dfrac{y}{p} = \tan\phi$, where $\phi$
is the inclination of the terminal tangent to the initial one.%
\index{Parabola}

\item[6.] The centre of gravity of this arc is given by
\begin{equation*}
3l\bar{x} = p^2(\cosh^3u-1),\quad 64l\bar{y} = p^2(\sinh 4u-4u);
\end{equation*}
and the surface of a paraboloid of revolution is $S = 2\pi\bar{y}l$.%
\index{Center of gravity}

\item[7.] The moment of inertia of the same arc about its terminal
ordinate is $I = \mu\left[xl(x - 2\bar{x}) +
\frac{1}{64}p^3N\right]$, where $\mu$ is the mass of unit length,
and
\begin{equation*}
N = u - \frac{1}{4}\sinh 2u
      - \frac{1}{4}\sinh 4u + \frac{1}{12}\sinh 6u.
\end{equation*}\index{Moment of inertia}

\item[8.] The centre of gravity of the arc of a catenary measured
from the lowest point is given by
\begin{equation*}
4l\bar{y} = c^2(\sinh 2u + 2u),\
 l\bar{x} = c^2(u\sinh u - \cosh u+1),
\end{equation*}
in which $u = \frac{x}{c}$; and the moment of inertia of this arc
about its terminal abscissa is
\begin{equation*}
I = \mu c^3\left(\frac{1}{12}\sinh 3u + \frac{3}{4}\sinh u -
                 u\cosh u\right).
\end{equation*}

\item[9.] Applications to the vibrations of bars are given in Rayleigh,
Theory of Sound, Vol.~I, art.~170; to the torsion of prisms in Love,
Elasticity, pp.~166--74; to the flow of heat and electricity in
Byerly, Fourier Series, pp.~75--81; to wave motion in fluids in
Rayleigh, Vol.~I, Appendix, p.~477, and in Bassett, Hydrodynamics,
arts.~120, 384; to the theory of potential in Byerly p.~135, and in
Maxwell, Electricity, arts.~172--4\index{Maxwell's Electricity}; to
Non-Euclidian geometry and many other subjects in
G�nther\index{Gunther's Die Lehre, etc.}, Hyperbelfunktionen,
Chaps.~V and VI. Several numerical examples are
worked out in Laisant, Essai sur les fonctions hyperboliques.%
\index{Bassett's Hydrodynamics}\index{Byerly's Fourier Series, etc.}%
\index{Fourier series}\index{Laisant's Essai, etc.}%
\index{Love's elasticity}\index{Potential theory}%
\index{Vibrations of bars}
\end{enumerate}
\index{Applications|)}\index{Hyperbolic functions!applictions of|)}%
\index{Physical problems|)}

\chapter{Explanation of Tables.}\index{Complex numbers!Tables}%
\index{Tables}

In Table I the numerical values of the hyperbolic functions $\sinh
u, \cosh u, \tanh u$ are tabulated for values of $u$ increasing from
0 to 4 at intervals of .02. When $u$ exceeds 4, Table IV may be
used.

Table II gives hyperbolic functions of complex arguments, in which
\begin{equation*}
\cosh (x \pm iy) = a \pm ib,\ \sinh (x \pm iy) = c \pm id,
\end{equation*}
and the values of $a, b, c, d$ are tabulated for values of $x$ and
of $y$ ranging separately from 0 to 1.5 at intervals of .1. When
interpolation is necessary it may be performed in three
stages.\index{Interpolation} For example, to find $\cosh (.82 +
1.34i)$: First find $\cosh (.82 + 1.3i)$, by keeping $y$ at 1.3 and
interpolating between the entries under $x =.8$ and $x =.9$; next
find $\cosh (.82 + 1.4i)$, by keeping $y$ at 1.4 and interpolating
between the entries under $x =.8$ and $x =.9$, as before; then by
interpolation between $\cosh (.82 + 1.3i)$ and $\cosh (.82 + 1.4i)$
find $\cosh(.82 + 1.34i)$, in which $x$ is kept at .82. The table is
available for all values of $y$, however great, by means of the
formulas on page~\pageref{periodicity of hyperbolic functions}:
\begin{equation*}
\sinh (x + 2i\pi) = \sinh x,\
   \cosh (x + 2i\pi) = \cosh x, \text{ etc.}
\end{equation*}
It does not apply when $x$ is greater than 1.5, but this case seldom
occurs in practice. This table can also be used as a complex table
of circular functions, for
\begin{equation*}
\cos (y \pm ix) = a \mp ib,\ \sin (y \pm ix) = d \pm ic;
\end{equation*}
and, moreover, the exponential function is given by
\begin{equation*}
\exp (\pm x \pm iy) = a \pm c \pm i(b \pm d),
\end{equation*}
in which the signs of $c$ and $d$ are to be taken the same as the
sign of $x$, and the sign of $i$ on the right is to be the product
of the signs of $x$ and of $i$ on the left.\index{Periodicity}

Table III gives the values of $v = \gd u$, and of the gudermanian
angle $\theta = \dfrac{180^\circ v}{\pi}$, as $u$ changes from 0 to
1 at intervals of .02, from 1 to 2 at intervals of .05, and from 2
to 4 at intervals of .1.

In Table IV are given, the values of $\gd u$, $\log \sinh u$, $\log
\cosh u$, as $u$ increases from 4 to 6 at intervals of .1, from 6 to
7 at intervals of .2, and from 7 to 9 at intervals of .5.

In the rare cases in which more extensive tables are necessary,
reference may be made to the tables\footnote{%
Gudermann in Crelle's Journal, vols. 6--9, 1831--2 (published
separately under the title Theorie der hyperbolischen Functionen,
Berlin, 1833). Glaisher in Cambridge Phil.\ Trans., vol.\ 13, 1881.
Geipel and Kilgour's Electrical Handbook.}\index{Geipel and
Kilgour's Electrical Handbook}\index{Glaisher's exponential tables}
of Gudermann, Glaisher, and Geipel and Kilgour. In the first the
Gudermanian angle (written $k$) is taken as the independent
variable, and increases from 0 to 100 grades at intervals of .01,
the corresponding value of $u$ (written $Lk$) being tabulated. In
the usual case, in which the table is entered with the value of $u$,
it gives by interpolation the value of the gudermanian angle, whose
circular functions would then give the hyperbolic functions of
$u$.\index{Interpolation} When $u$ is large, this angle is so nearly
right that interpolation is not reliable. To remedy this
inconvenience Gudermann's second table gives directly $\log\sinh u$,
$\log\cosh u$, $\log\tanh u$, to nine figures, for values of $u$
varying by .001 from 2 to 5, and by .01 from 5 to
12.\index{Gudermanian!function}

Glaisher has tabulated the values of $e^x$ and $e^{-x}$, to nine
significant figures, as $x$ varies by .001 from 0 to .1, by .01 from
0 to 2, by .1 from 0 to 10, and by 1 from 0 to 500. From these the
values of $\cosh x$, $\sinh x$ are easily obtained.

Geipel and Kilgour's handbook gives the values of $\cosh x$, $\sinh
x$, to seven figures, as $x$ varies by .01 from 0 to 4.

There are also extensive tables by Forti, Gronau, Vassal, Callet,
and Ho�el; and there are four-place tables in Byerly's Fourier
Series, and in Wheeler's Trigonometry.%
\index{Byerly's Fourier Series, etc.}\index{Callet's Tables}%
\index{Forti's Tavoli e teoria}\index{Gronau's!Tafeln}%
\index{Vassal's Tables}\index{Wheeler's Trigonometry}

In the following tables a dash over a final digit indicates that the
number has been increased.

\newpage
\markright{TABLES}

\addcontentsline{lot}{table}{Table I.---Hyperbolic Functions}
\index{Hyperbolic functions!tables of|(}
\begin{center}
\textsc{Table I.---Hyperbolic Functions.}\label{Table1p1} \\
\medskip\scriptsize
\begin{tabular}{r|r|r|r||r|r|r|r}
\hline

\multicolumn{1}{c|}{$u$}
         &$   \sinh u.   $&$   \cosh u.   $&$   \tanh u.  $ &
\multicolumn{1}{c|}{$u$}
         &$   \sinh u.   $&$   \cosh u.   $&$   \tanh u.  $
\\
\hline
&&&&&&& \\
$  .00 $&$  .0000       $&$ 1.0000       $&$  .0000       $ &
$ 1.00 $&$ 1.1752       $&$ 1.543\bar{1} $&$  .7616       $
\\
$   02 $&$   0200       $&$ 1.0002       $&$   0200       $ &
$ 1.02 $&$ 1.206\bar{3} $&$ 1.566\bar{9} $&$   769\bar{9} $
\\
$   04 $&$   0400       $&$ 1.0008       $&$   040\bar{0} $ &
$ 1.04 $&$ 1.237\bar{9} $&$ 1.5913       $&$   777\bar{9} $
\\
$   06 $&$   0600       $&$ 1.0018       $&$   0599       $ &
$ 1.06 $&$ 1.270\bar{0} $&$ 1.6164       $&$   785\bar{7} $
\\
$   08 $&$   080\bar{1} $&$ 1.0032       $&$   0798       $ &
$ 1.08 $&$ 1.3025       $&$ 1.6421       $&$   793\bar{2} $
\\
&&&&&&& \\
$  .10 $&$  .100\bar{2} $&$ 1.0050       $&$  .099\bar{7} $ &
$ 1.10 $&$ 1.3356       $&$ 1.6685       $&$  .8005       $
\\
$   12 $&$   120\bar{3} $&$ 1.0072       $&$   1194       $ &
$ 1.12 $&$ 1.369\bar{3} $&$ 1.695\bar{6} $&$   807\bar{6} $
\\
$   14 $&$   140\bar{5} $&$ 1.0098       $&$   139\bar{1} $ &
$ 1.14 $&$ 1.403\bar{5} $&$ 1.7233       $&$   8144       $
\\
$   16 $&$   160\bar{7} $&$ 1.0128       $&$   1586       $ &
$ 1.16 $&$ 1.4382       $&$ 1.7517       $&$   8210       $
\\
$   18 $&$   181\bar{0} $&$ 1.0162       $&$   178\bar{1} $ &
$ 1.18 $&$ 1.4735       $&$ 1.7808       $&$   827\bar{5} $
\\
&&&&&&& \\
$  .20 $&$  .2013       $&$ 1.020\bar{1} $&$  .197\bar{4} $ &
$ 1.20 $&$ 1.509\bar{5} $&$ 1.810\bar{7} $&$  .833\bar{7} $
\\
$   22 $&$   221\bar{8} $&$ 1.024\bar{3} $&$   2165       $ &
$ 1.22 $&$ 1.546\bar{0} $&$ 1.8412       $&$   839\bar{7} $
\\
$   24 $&$   2423       $&$ 1.0289       $&$   235\bar{5} $ &
$ 1.24 $&$ 1.5831       $&$ 1.872\bar{5} $&$   845\bar{5} $
\\
$   26 $&$   2629       $&$ 1.034\bar{0} $&$   254\bar{3} $ &
$ 1.26 $&$ 1.620\bar{9} $&$ 1.9045       $&$   851\bar{1} $
\\
$   28 $&$   283\bar{7} $&$ 1.0395       $&$   2729       $ &
$ 1.28 $&$ 1.6593       $&$ 1.9373       $&$   856\bar{5} $
\\
&&&&&&& \\
$  .30 $&$  .3045       $&$ 1.0453       $&$  .2913       $ &
$ 1.30 $&$ 1.6984       $&$ 1.9709       $&$  .8617       $
\\
$   32 $&$   325\bar{5} $&$ 1.0516       $&$   3095       $ &
$ 1.32 $&$ 1.7381       $&$ 2.005\bar{3} $&$   8668       $
\\
$   34 $&$   3466       $&$ 1.058\bar{4} $&$   327\bar{5} $ &
$ 1.34 $&$ 1.778\bar{6} $&$ 2.0404       $&$   871\bar{7} $
\\
$   36 $&$   3678       $&$ 1.0655       $&$   3452       $ &
$ 1.36 $&$ 1.819\bar{8} $&$ 2.0764       $&$   876\bar{4} $
\\
$   38 $&$   3892       $&$ 1.0731       $&$   3627       $ &
$ 1.38 $&$ 1.861\bar{7} $&$ 2.1132       $&$   881\bar{0} $
\\
&&&&&&& \\
$  .40 $&$  .410\bar{8} $&$ 1.081\bar{1} $&$  .3799       $ &
$ 1.40 $&$ 1.9043       $&$ 2.150\bar{9} $&$  .8854       $
\\
$   42 $&$   432\bar{5} $&$ 1.0895       $&$   3969       $ &
$ 1.42 $&$ 1.9477       $&$ 2.1894       $&$   889\bar{6} $
\\
$   44 $&$   4543       $&$ 1.098\bar{4} $&$   4136       $ &
$ 1.44 $&$ 1.991\bar{9} $&$ 2.2288       $&$   893\bar{7} $
\\
$   46 $&$   476\bar{4} $&$ 1.107\bar{7} $&$   430\bar{1} $ &
$ 1.46 $&$ 2.036\bar{9} $&$ 2.269\bar{1} $&$   897\bar{7} $
\\
$   48 $&$   4986       $&$ 1.1174       $&$   4462       $ &
$ 1.48 $&$ 2.082\bar{7} $&$ 2.310\bar{3} $&$   901\bar{5} $
\\
&&&&&&& \\
$  .50 $&$  .521\bar{1} $&$ 1.1276       $&$  .4621       $ &
$ 1.50 $&$ 2.129\bar{3} $&$ 2.3524       $&$  .9051       $
\\
$   52 $&$   543\bar{8} $&$ 1.138\bar{3} $&$   4777       $ &
$ 1.52 $&$ 2.176\bar{8} $&$ 2.395\bar{5} $&$   908\bar{7} $
\\
$   54 $&$   5666       $&$ 1.149\bar{4} $&$   493\bar{0} $ &
$ 1.54 $&$ 2.2251       $&$ 2.439\bar{5} $&$   9121       $
\\
$   56 $&$   5897       $&$ 1.1609       $&$   508\bar{0} $ &
$ 1.56 $&$ 2.2743       $&$ 2.484\bar{5} $&$   9154       $
\\
$   58 $&$   613\bar{1} $&$ 1.173\bar{0} $&$   522\bar{7} $ &
$ 1.58 $&$ 2.324\bar{5} $&$ 2.530\bar{5} $&$   9186       $
\\
&&&&&&& \\
$  .60 $&$  .636\bar{7} $&$ 1.185\bar{5} $&$  .5370       $ &
$ 1.60 $&$ 2.375\bar{6} $&$ 2.577\bar{5} $&$  .921\bar{7} $
\\
$   62 $&$   660\bar{5} $&$ 1.1984       $&$   5511       $ &
$ 1.62 $&$ 2.427\bar{6} $&$ 2.625\bar{5} $&$   9246       $
\\
$   64 $&$   684\bar{6} $&$ 1.211\bar{9} $&$   564\bar{9} $ &
$ 1.64 $&$ 2.480\bar{6} $&$ 2.674\bar{6} $&$   927\bar{5} $
\\
$   66 $&$   709\bar{0} $&$ 1.2258       $&$   578\bar{4} $ &
$ 1.66 $&$ 2.534\bar{6} $&$ 2.7247       $&$   9302       $
\\
$   68 $&$   7336       $&$ 1.2402       $&$   5915       $ &
$ 1.68 $&$ 2.5896       $&$ 2.776\bar{0} $&$   932\bar{9} $
\\
&&&&&&& \\
$  .70 $&$  .758\bar{6} $&$ 1.255\bar{2} $&$  .604\bar{4} $ &
$ 1.70 $&$ 2.6456       $&$ 2.8283       $&$  .9354       $
\\
$   72 $&$   7838       $&$ 1.270\bar{6} $&$   6169       $ &
$ 1.72 $&$ 2.7027       $&$ 2.881\bar{8} $&$   937\bar{9} $
\\
$   74 $&$   8094       $&$ 1.2865       $&$   6291       $ &
$ 1.74 $&$ 2.7609       $&$ 2.9364       $&$   9402       $
\\
$   76 $&$   8353       $&$ 1.303\bar{0} $&$   641\bar{1} $ &
$ 1.76 $&$ 2.820\bar{2} $&$ 2.9922       $&$   9425       $
\\
$   78 $&$   8615       $&$ 1.3199       $&$   6527       $ &
$ 1.78 $&$ 2.8806       $&$ 3.0492       $&$   944\bar{7} $
\\
&&&&&&& \\
$  .80 $&$  .8881       $&$ 1.3374       $&$  .6640       $ &
$ 1.80 $&$ 2.942\bar{2} $&$ 3.107\bar{5} $&$  .9468       $
\\
$   82 $&$   9150       $&$ 1.355\bar{5} $&$   675\bar{1} $ &
$ 1.82 $&$ 3.0049       $&$ 3.1669       $&$   9488       $
\\
$   84 $&$   9423       $&$ 1.3740       $&$   6858       $ &
$ 1.84 $&$ 3.068\bar{9} $&$ 3.227\bar{7} $&$   950\bar{8} $
\\
$   86 $&$   970\bar{0} $&$ 1.393\bar{2} $&$   696\bar{3} $ &
$ 1.86 $&$ 3.1340       $&$ 3.2897       $&$   952\bar{7} $
\\
$   88 $&$   998\bar{1} $&$ 1.4128       $&$   7064       $ &
$ 1.88 $&$ 3.200\bar{5} $&$ 3.3530       $&$   954\bar{5} $
\\
&&&&&&& \\
$  .90 $&$ 1.0265       $&$ 1.433\bar{1} $&$  .716\bar{3} $ &
$ 1.90 $&$ 3.268\bar{2} $&$ 3.4177       $&$  .9562       $
\\
$   92 $&$ 1.0554       $&$ 1.4539       $&$   725\bar{9} $ &
$ 1.92 $&$ 3.337\bar{2} $&$ 3.483\bar{8} $&$   9579       $
\\
$   94 $&$ 1.084\bar{7} $&$ 1.4753       $&$   7352       $ &
$ 1.94 $&$ 3.4075       $&$ 3.5512       $&$   9595       $
\\
$   96 $&$ 1.1144       $&$ 1.497\bar{3} $&$   744\bar{3} $ &
$ 1.96 $&$ 3.4792       $&$ 3.620\bar{1} $&$   961\bar{1} $
\\
$   98 $&$ 1.144\bar{6} $&$ 1.519\bar{9} $&$   753\bar{1} $ &
$ 1.98 $&$ 3.5523       $&$ 3.6904       $&$   962\bar{6} $
\\
&&&&&&& \\
\hline
\end{tabular} \end{center} \normalsize

\newpage
\begin{center}
\textsc{Table I.---Hyperbolic Functions} (\emph{continued})%
\label{Table1p2} \\
\medskip\scriptsize
\begin{tabular}{r|r|r|r||r|r|r|r}
\hline \multicolumn{1}{c|}{$u$}
         &$   \sinh u.   $&$   \cosh u.   $&$   \tanh u.  $ &
\multicolumn{1}{c|}{$u$}
         &$   \sinh u.   $&$   \cosh u.   $&$   \tanh u.  $
\\
\hline
&&&&&&& \\
$ 2.00 $&$  3.626\bar{9} $&$  3.762\bar{2} $&$  .9640        $ &
$ 3.00 $&$ 10.017\bar{9} $&$ 10.067\bar{7} $&$  .99505       $
\\
$ 2.02 $&$  3.7028       $&$  3.835\bar{5} $&$   9654        $ &
$ 3.02 $&$ 10.2212       $&$ 10.2700       $&$   99524       $
\\
$ 2.04 $&$  3.780\bar{3} $&$  3.9103       $&$   9667        $ &
$ 3.04 $&$ 10.4287       $&$ 10.4765       $&$   99543       $
\\
$ 2.06 $&$  3.859\bar{3} $&$  3.9867       $&$   9680        $ &
$ 3.06 $&$ 10.6403       $&$ 10.6872       $&$   99561       $
\\
$ 2.08 $&$  3.9398       $&$  4.0647       $&$   969\bar{3}  $ &
$ 3.08 $&$ 10.8562       $&$ 10.902\bar{2} $&$   99578       $
\\
&&&&&&& \\
$ 2.10 $&$  4.021\bar{9} $&$  4.1443       $&$  .970\bar{5}  $ &
$ 3.10 $&$ 11.076\bar{5} $&$ 11.1215       $&$  .99594       $
\\
$ 2.12 $&$  4.1055       $&$  4.225\bar{6} $&$   971\bar{6}  $ &
$ 3.12 $&$ 11.3011       $&$ 11.345\bar{3} $&$   99610       $
\\
$ 2.14 $&$  4.190\bar{9} $&$  4.3085       $&$   972\bar{7}  $ &
$ 3.14 $&$ 11.530\bar{3} $&$ 11.573\bar{6} $&$   99626       $
\\
$ 2.16 $&$  4.2779       $&$  4.3932       $&$   9737        $ &
$ 3.16 $&$ 11.764\bar{1} $&$ 11.8065       $&$   99640       $
\\
$ 2.18 $&$  4.3666       $&$  4.479\bar{7} $&$   974\bar{8}  $ &
$ 3.18 $&$ 12.002\bar{6} $&$ 12.044\bar{2} $&$   99654       $
\\
&&&&&&& \\
$ 2.20 $&$  4.4571       $&$  4.5679       $&$  .9757        $ &
$ 3.20 $&$ 12.245\bar{9} $&$ 12.2866       $&$  .99668       $
\\
$ 2.22 $&$  4.549\bar{4} $&$  4.658\bar{0} $&$   976\bar{7}  $ &
$ 3.22 $&$ 12.494\bar{1} $&$ 12.5340       $&$   99681       $
\\
$ 2.24 $&$  4.6434       $&$  4.749\bar{9} $&$   977\bar{6}  $ &
$ 3.24 $&$ 12.747\bar{3} $&$ 12.7864       $&$   99693       $
\\
$ 2.26 $&$  4.739\bar{4} $&$  4.8437       $&$   978\bar{5}  $ &
$ 3.26 $&$ 13.005\bar{6} $&$ 13.044\bar{0} $&$   99705       $
\\
$ 2.28 $&$  4.837\bar{2} $&$  4.939\bar{5} $&$   979\bar{3}  $ &
$ 3.28 $&$ 13.269\bar{1} $&$ 13.3067       $&$   99717       $
\\
&&&&&&& \\
$ 2.30 $&$  4.937\bar{0} $&$  5.0372       $&$  .980\bar{1}  $ &
$ 3.30 $&$ 13.537\bar{9} $&$ 13.574\bar{8} $&$  .99728       $
\\
$ 2.32 $&$  5.0387       $&$  5.137\bar{0} $&$   980\bar{9}  $ &
$ 3.32 $&$ 13.812\bar{1} $&$ 13.848\bar{3} $&$   99738       $
\\
$ 2.34 $&$  5.1425       $&$  5.238\bar{8} $&$   9816        $ &
$ 3.34 $&$ 14.0918       $&$ 14.127\bar{3} $&$   99749       $
\\
$ 2.36 $&$  5.248\bar{3} $&$  5.342\bar{7} $&$   9823        $ &
$ 3.36 $&$ 14.3772       $&$ 14.412\bar{0} $&$   99758       $
\\
$ 2.38 $&$  5.356\bar{2} $&$  5.4487       $&$   9830        $ &
$ 3.38 $&$ 14.668\bar{4} $&$ 14.7024       $&$   99768       $
\\
&&&&&&& \\
$ 2.40 $&$  5.4662       $&$  5.5569       $&$  .983\bar{7}  $ &
$ 3.40 $&$ 14.965\bar{4} $&$ 14.9987       $&$  .99777       $
\\
$ 2.42 $&$  5.5785       $&$  5.667\bar{4} $&$   9843        $ &
$ 3.42 $&$ 15.268\bar{4} $&$ 15.301\bar{1} $&$   99786       $
\\
$ 2.44 $&$  5.6929       $&$  5.7801       $&$   9849        $ &
$ 3.44 $&$ 15.5774       $&$ 15.6095       $&$   99794       $
\\
$ 2.46 $&$  5.809\bar{7} $&$  5.8951       $&$   9855        $ &
$ 3.46 $&$ 15.892\bar{8} $&$ 15.9242       $&$   99802       $
\\
$ 2.48 $&$  5.928\bar{8} $&$  6.0125       $&$   986\bar{1}  $ &
$ 3.48 $&$ 16.2144       $&$ 16.245\bar{3} $&$   99810       $
\\
&&&&&&& \\
$ 2.50 $&$  6.0502       $&$  6.132\bar{3} $&$  .9866        $ &
$ 3.50 $&$ 16.5426       $&$ 16.5728       $&$  .99817       $
\\
$ 2.52 $&$  6.174\bar{1} $&$  6.2545       $&$   9871        $ &
$ 3.52 $&$ 16.8774       $&$ 16.9070       $&$   99824       $
\\
$ 2.54 $&$  6.3004       $&$  6.379\bar{3} $&$   9876        $ &
$ 3.54 $&$ 17.219\bar{0} $&$ 17.248\bar{0} $&$   99831       $
\\
$ 2.56 $&$  6.429\bar{3} $&$  6.506\bar{6} $&$   9881        $ &
$ 3.56 $&$ 17.567\bar{4} $&$ 17.5958       $&$   99831       $
\\
$ 2.58 $&$  6.560\bar{7} $&$  6.6364       $&$   988\bar{6}  $ &
$ 3.58 $&$ 17.9228       $&$ 17.9507       $&$   99844       $
\\
&&&&&&& \\
$ 2.60 $&$  6.6947       $&$  6.7690       $&$  .9890        $ &
$ 3.60 $&$ 18.2854       $&$ 18.312\bar{8} $&$  .99850       $
\\
$ 2.62 $&$  6.831\bar{5} $&$  6.904\bar{3} $&$   989\bar{5}  $ &
$ 3.62 $&$ 18.655\bar{4} $&$ 18.682\bar{2} $&$   99856       $
\\
$ 2.64 $&$  6.9709       $&$  7.042\bar{3} $&$   989\bar{9}  $ &
$ 3.64 $&$ 19.032\bar{8} $&$ 19.0590       $&$   99862       $
\\
$ 2.66 $&$  7.113\bar{2} $&$  7.183\bar{2} $&$   990\bar{3}  $ &
$ 3.66 $&$ 19.4178       $&$ 19.4435       $&$   99867       $
\\
$ 2.68 $&$  7.258\bar{3} $&$  7.3268       $&$   9906        $ &
$ 3.68 $&$ 19.810\bar{6} $&$ 19.8358       $&$   99872       $
\\
&&&&&&& \\
$ 2.70 $&$  7.406\bar{3} $&$  7.473\bar{5} $&$  .9910        $ &
$ 3.70 $&$ 20.211\bar{3} $&$ 20.2360       $&$  .99877       $
\\
$ 2.72 $&$  7.5572       $&$  7.623\bar{1} $&$   991\bar{4}  $ &
$ 3.72 $&$ 20.620\bar{1} $&$ 20.6443       $&$   99882       $
\\
$ 2.74 $&$  7.7112       $&$  7.775\bar{8} $&$   991\bar{7}  $ &
$ 3.74 $&$ 21.0371       $&$ 21.060\bar{9} $&$   99887       $
\\
$ 2.76 $&$  7.868\bar{3} $&$  7.931\bar{6} $&$   9920        $ &
$ 3.76 $&$ 21.462\bar{6} $&$ 21.485\bar{9} $&$   99891       $
\\
$ 2.78 $&$  8.028\bar{5} $&$  8.0905       $&$   9923        $ &
$ 3.78 $&$ 21.8966       $&$ 21.9194       $&$   9989\bar{6} $
\\
&&&&&&& \\
$ 2.80 $&$  8.1919       $&$  8.2527       $&$  .9926        $ &
$ 3.80 $&$ 22.3394       $&$ 22.361\bar{8} $&$  .9990\bar{0} $
\\
$ 2.82 $&$  8.3586       $&$  8.4182       $&$   9929        $ &
$ 3.82 $&$ 22.7911       $&$ 22.813\bar{1} $&$   9990\bar{4} $
\\
$ 2.84 $&$  8.528\bar{7} $&$  8.587\bar{1} $&$   993\bar{2}  $ &
$ 3.84 $&$ 23.252\bar{0} $&$ 23.273\bar{5} $&$   99907       $
\\
$ 2.86 $&$  8.7021       $&$  8.759\bar{4} $&$   993\bar{5}  $ &
$ 3.86 $&$ 23.7221       $&$ 23.7432       $&$   99911       $
\\
$ 2.88 $&$  8.879\bar{1} $&$  8.9352       $&$   9937        $ &
$ 3.88 $&$ 24.2018       $&$ 24.2224       $&$   9991\bar{5} $
\\
&&&&&&& \\
$ 2.90 $&$  9.059\bar{6} $&$  9.114\bar{6} $&$  .994\bar{0}  $ &
$ 3.90 $&$ 24.6911       $&$ 24.7113       $&$  .99918       $
\\
$ 2.92 $&$  9.243\bar{7} $&$  9.2976       $&$   994\bar{2}  $ &
$ 3.92 $&$ 25.1903       $&$ 25.2101       $&$   99921       $
\\
$ 2.94 $&$  9.431\bar{5} $&$  9.484\bar{4} $&$   9944        $ &
$ 3.94 $&$ 25.699\bar{6} $&$ 25.7190       $&$   99924       $
\\
$ 2.96 $&$  9.623\bar{1} $&$  9.674\bar{9} $&$   994\bar{7}  $ &
$ 3.96 $&$ 26.2191       $&$ 26.238\bar{2} $&$   99927       $
\\
$ 2.98 $&$  9.8185       $&$  9.8693       $&$   994\bar{9}  $ &
$ 3.98 $&$ 26.749\bar{2} $&$ 26.767\bar{9} $&$   99930       $
\\
&&&&&&& \\
\hline
\end{tabular} \end{center} \normalsize

\newpage
\begin{center}
\addcontentsline{lot}{table}{Table II.---Values of $\cosh(x + iy)$
and $\sinh (x + iy)$}
\textsc{Table II.---Values of $\cosh(x + iy)$ and $\sinh(x + iy)$.}%
\index{Complex numbers!Tables}
\\
\medskip \footnotesize
\begin{tabular}{r| rc| cr| rr| rr}
\hline
  & \multicolumn{4}{|c|}{$ x = 0 $} & \multicolumn{4}{|c}{$ x = .1 $}
\\
\cline{2-9}
  $y$ & \multicolumn{1}{c}{$a$}&\multicolumn{1}{c|}{$b$}
      & \multicolumn{1}{c}{$c$}&\multicolumn{1}{c|}{$d$}
      & \multicolumn{1}{c}{$a$}&\multicolumn{1}{c|}{$b$}
      & \multicolumn{1}{c}{$c$}&\multicolumn{1}{c}{$d$}
\\
\hline
&&&&&&&& \\
  0 &$ 1.0000       $&$ 0000        $&$ 0000         $&$ .0000       $
    &$ 1.0050       $&$ .00000      $&$ .1001\bar{7} $&$ .0000       $
\\
 .1 &$ 0.9950       $& ''            & ''             &$  0998       $
    &$ 1.000\bar{0} $&$  01000      $&$  09967       $&$  1003       $
\\
 .2 &$ 0.980\bar{1} $& ''            & ''             &$  198\bar{7} $
    &$ 0.9850       $&$  0199\bar{0}$&$  09817       $&$  199\bar{7} $
\\
 .3 &$ 0.9553       $& ''            & ''             &$  2955       $
    &$ 0.9601       $&$  02960      $&$  0957\bar{0} $&$  297\bar{0} $
\\
&&&&&&&& \\
 .4 &$  .921\bar{1} $& ''            & ''             &$ .3894       $
    &$  .925\bar{7} $&$ .03901      $&$ .09226       $&$ .3914       $
\\
 .5 &$   8776       $& ''            & ''             &$  4794       $
    &$   882\bar{0} $&$  04802      $&$  0879\bar{1} $&$  4818       $
\\
 .6 &$   8253       $& ''            & ''             &$  5646       $
    &$   829\bar{5} $&$  05656      $&$  08267       $&$  567\bar{5} $
\\
 .7 &$   7648       $& ''            & ''             &$  6442       $
    &$   768\bar{7} $&$  06453      $&$  07661       $&$  6474       $
\\
&&&&&&&& \\
 .8 &$  .6967       $& ''            & ''             &$ .717\bar{4} $
    &$  .700\bar{2} $&$ .0718\bar{6}$&$ .0697\bar{9} $&$ .7800       $
\\
 .9 &$   6216       $& ''            & ''             &$  7833       $
    &$   624\bar{7} $&$  0784\bar{7}$&$  0622\bar{7} $&$  7872       $
\\

1.0 &$   5403       $& ''            & ''             &$  841\bar{5} $
    &$   5430       $&$  08429      $&$  05412       $&$  845\bar{7} $
\\
1.1 &$   4536       $& ''            & ''             &$  8912       $
    &$   455\bar{9} $&$  08927      $&$  04544       $&$  895\bar{7} $
\\
&&&&&&&& \\
1.2 &$  .362\bar{4} $& ''            & ''             &$ .9320       $
    &$  .364\bar{2} $&$ .09336      $&$ .0363\bar{0} $&$0.936\bar{7} $
\\
1.3 &$   2675       $& ''            & ''             &$  963\bar{6} $
    &$   268\bar{8} $&$  0965\bar{2}$&$  0268\bar{0} $&$0.968\bar{4} $
\\
1.4 &$   170\bar{0} $& ''            & ''             &$  9854       $
    &$   1708       $&$  09871      $&$  0170\bar{3} $&$0.990\bar{4} $
\\
1.5 &$   0707       $& ''            & ''             &$  997\bar{5} $
    &$   0711       $&$  0999\bar{2}$&$  0070\bar{9} $&$1.002\bar{5} $
\\
&&&&&&&& \\
$\tfrac{1}{2}\pi $
    &$   0000       $& ''            & ''             &$1.0000       $
    &$   0000       $&$  1001\bar{7}$&$  00000       $&$1.0050       $
\\
&&&&&&&& \\
\hline
\end{tabular} \\

\bigskip
\begin{tabular}{r| rr| rr| rr| rr}
\hline
  & \multicolumn{4}{|c|}{$ x = .2 $} & \multicolumn{4}{|c}{$ x = .3 $}
\\
\cline{2-9}
  $y$ & \multicolumn{1}{c}{$a$}&\multicolumn{1}{c|}{$b$}
      & \multicolumn{1}{c}{$c$}&\multicolumn{1}{c|}{$d$}
      & \multicolumn{1}{c}{$a$}&\multicolumn{1}{c|}{$b$}
      & \multicolumn{1}{c}{$c$}&\multicolumn{1}{c}{$d$}
\\
\hline
&&&&&&&& \\
  0 &$ 1.020\bar{1}  $&$ .0000       $&$ .2013       $&$ .0000       $
    &$ 1.0453        $&$ .0000       $&$ .3045       $&$ .0000       $
\\
 .1 &$ 1.015\bar{0}  $&$  0201       $&$  2003       $&$  1018       $
    &$ 1.040\bar{1}  $&$  0304       $&$  303\bar{0} $&$  1044       $
\\
 .2 &$ 0.9997        $&$  0400       $&$  1973       $&$  202\bar{7} $
    &$ 1.024\bar{5}  $&$  0605       $&$  298\bar{5} $&$  207\bar{7} $
\\
 .3 &$ 0.9745        $&$  0595       $&$  1923       $&$  3014       $
    &$   9987        $&$  090\bar{0} $&$  2909       $&$  3089       $
\\
&&&&&&&& \\
 .4 &$  .9395        $&$ .0784       $&$ .1854       $&$ .3972       $
    &$  .9628        $&$ .1186       $&$ .280\bar{5} $&$ .407\bar{1} $
\\
 .5 &$   895\bar{2}  $&$  0965       $&$  176\bar{7} $&$  4890       $
    &$   917\bar{4}  $&$  146\bar{0} $&$  267\bar{2} $&$  501\bar{2} $
\\
 .6 &$   8419        $&$  113\bar{7} $&$  166\bar{2} $&$  576\bar{0} $
    &$   8687        $&$  1719       $&$  2513       $&$  590\bar{3} $
\\
 .7 &$   780\bar{2}  $&$  1297       $&$  154\bar{0} $&$  6571       $
    &$   7995        $&$  196\bar{2} $&$  2329       $&$  6734       $
\\
&&&&&&&& \\
 .8 &$  .710\bar{7}  $&$ .1444       $&$ .140\bar{3} $&$ .731\bar{8} $
    &$  .728\bar{3}  $&$ .2184       $&$ .212\bar{2} $&$ .7498       $
\\
 .9 &$   634\bar{1}  $&$  1577       $&$  125\bar{2} $&$  7990       $
    &$   649\bar{8}  $&$  2385       $&$  189\bar{3} $&$  8188       $
\\
1.0 &$   5511        $&$  1694       $&$  108\bar{8} $&$  858\bar{4} $
    &$   5648        $&$  2562       $&$  1645       $&$  8796       $
\\
1.1 &$   4627        $&$  179\bar{5} $&$  0913       $&$  909\bar{1} $
    &$   474\bar{2}  $&$  2714       $&$  1381       $&$  9316       $
\\
&&&&&&&& \\
1.2 &$  .3696        $&$ .187\bar{7} $&$ .073\bar{0} $&$0.9507       $
    &$  .378\bar{8}  $&$ .2838       $&$ .1103       $&$0.974\bar{3} $
\\
1.3 &$   272\bar{9}  $&$  1940       $&$  053\bar{9} $&$0.982\bar{9} $
    &$   2796        $&$  2934       $&$  081\bar{5} $&$1.0072       $
\\
1.4 &$   173\bar{4}  $&$  1984       $&$  0342       $&$1.0052       $
    &$   177\bar{7}  $&$  3001       $&$  051\bar{8} $&$1.0301       $
\\
1.5 &$   072\bar{2}  $&$  2008       $&$  0142       $&$1.0175       $
    &$   0739        $&$  303\bar{8} $&$  0215       $&$1.042\bar{7} $
\\
&&&&&&&& \\
$\tfrac{1}{2}\pi $
    &$   0000        $&$  2013       $&$  0000       $&$1.020\bar{1} $
    &$   0000        $&$  3045       $&$  0000       $&$1.0453       $
\\
&&&&&&&& \\
\hline
\end{tabular} \normalsize

\newpage
\textsc{Table II.---Values of $\cosh(x + iy)$ and $\sinh(x + iy)$.}
(\emph{continued}) \\
\footnotesize
\bigskip
\begin{tabular}{r| rr| rr| rr| rr}
\hline
  & \multicolumn{4}{|c|}{$ x = .4 $} & \multicolumn{4}{|c}{$ x = .5 $}
\\
\cline{2-9}
  $y$ & \multicolumn{1}{c}{$a$}&\multicolumn{1}{c|}{$b$}
      & \multicolumn{1}{c}{$c$}&\multicolumn{1}{c|}{$d$}
      & \multicolumn{1}{c}{$a$}&\multicolumn{1}{c|}{$b$}
      & \multicolumn{1}{c}{$c$}&\multicolumn{1}{c}{$d$}
\\
\hline
&&&&&&& \\
  0 &$ 1.081\bar{1} $&$ .0000       $&$ .410\bar{8} $&$ .0000      $
    &$ 1.1276       $&$ .0000       $&$ .521\bar{1} $&$ .0000      $
\\
 .1 &$ 1.0756       $&$  0410       $&$  408\bar{7} $&$  1079      $
    &$ 1.122\bar{0} $&$  0520       $&$  518\bar{5} $&$  1126      $
\\
 .2 &$ 1.0595       $&$  0816       $&$  402\bar{6} $&$  214\bar{8}$
    &$ 1.1051       $&$  1025       $&$  5107       $&$  2240      $
\\
 .3 &$ 1.032\bar{8} $&$  121\bar{4} $&$  3924       $&$  319\bar{5}$
    &$ 1.077\bar{3} $&$  154\bar{0} $&$  4978       $&$  3332      $
\\
&&&&&&& \\
 .4 &$  .9957       $&$ .160\bar{0} $&$ .3783       $&$ .421\bar{0}$
    &$ 1.0386       $&$ .2029       $&$ .480\bar{0} $&$ .4391      $
\\
 .5 &$   9487       $&$  1969       $&$  360\bar{5} $&$  518\bar{3}$
    &$ 0.989\bar{6} $&$  2498       $&$  4573       $&$  5406      $
\\
 .6 &$   8922       $&$  2319       $&$  3390       $&$  6104      $
    &$ 0.9306       $&$  2942       $&$  430\bar{1} $&$  6367      $
\\
 .7 &$   8268       $&$  2646       $&$  314\bar{2} $&$  6964      $
    &$ 0.8624       $&$  335\bar{7} $&$  398\bar{6} $&$  7264      $
\\
&&&&&&& \\
 .8 &$  .753\bar{2} $&$ .2947       $&$ .286\bar{2} $&$ .7755      $
    &$  .7856       $&$ .3738       $&$ .363\bar{1} $&$0.8089      $
\\
 .9 &$   672\bar{0} $&$  3218       $&$  2553       $&$  8468      $
    &$   7009       $&$  408\bar{2} $&$  3239       $&$0.8833      $
\\
1.0 &$   5841       $&$  3456       $&$  2219       $&$  909\bar{7}$
    &$   609\bar{3} $&$  438\bar{5} $&$  2815       $&$0.948\bar{9}$
\\
1.1 &$   4904       $&$  366\bar{1} $&$  1863       $&$  963\bar{5}$
    &$   511\bar{5} $&$  4644       $&$  236\bar{4} $&$1.005\bar{0}$
\\
&&&&&&& \\
1.2 &$  .3917       $&$ .328\bar{9} $&$ .1488       $&$1.0076      $
    &$  .4056       $&$ .485\bar{7} $&$ .1888       $&$1.051\bar{0}$
\\
1.3 &$   289\bar{2} $&$  395\bar{8} $&$  109\bar{9} $&$1.041\bar{7}$
    &$   3016       $&$  5021       $&$  139\bar{4} $&$1.0865      $
\\
1.4 &$   183\bar{8} $&$  404\bar{8} $&$  0698       $&$1.0653      $
    &$   191\bar{7} $&$  5135       $&$  088\bar{6} $&$1.1163      $
\\
1.5 &$   076\bar{5} $&$  4097       $&$  029\bar{1} $&$1.078\bar{4}$
    &$   079\bar{8} $&$  519\bar{8} $&$  036\bar{9} $&$1.124\bar{8}$
\\
&&&&&&& \\
$ \tfrac{1}{2}\pi$
    &$   0000       $&$  410\bar{8} $&$  0000       $&$1.081\bar{1}$
    &$   0000       $&$  521\bar{1} $&$  0000       $&$1.1276      $
\\
&&&&&&&& \\
\hline
\end{tabular}

\bigskip
\begin{tabular}{r| rr| rr| rr| rr}
\hline
  & \multicolumn{4}{|c|}{$ x = .6 $} & \multicolumn{4}{|c}{$ x = .7 $}
\\
\cline{2-9}
  $y$ & \multicolumn{1}{c}{$a$}&\multicolumn{1}{c|}{$b$}
      & \multicolumn{1}{c}{$c$}&\multicolumn{1}{c|}{$d$}
      & \multicolumn{1}{c}{$a$}&\multicolumn{1}{c|}{$b$}
      & \multicolumn{1}{c}{$c$}&\multicolumn{1}{c}{$d$}
\\
\hline
&&&&&&& \\
  0 &$ 1.185\bar{5} $&$ .0000       $&$ .636\bar{7} $&$ .0000      $
    &$ 1.2552       $&$ .0000       $&$ .758\bar{6} $&$ .0000      $
\\
 .1 &$ 1.1795       $&$  063\bar{6} $&$  633\bar{5} $&$  1183      $
    &$ 1.248\bar{9} $&$  0757       $&$  754\bar{8} $&$  1253      $
\\
 .2 &$ 1.161\bar{8} $&$  126\bar{5} $&$  624\bar{0} $&$  2355      $
    &$ 1.2301       $&$  1542       $&$  743\bar{5} $&$  249\bar{4}$
\\
 .3 &$ 1.132\bar{5} $&$  1881       $&$  6082       $&$  3503      $
    &$ 1.1991       $&$  224\bar{2} $&$  7247       $&$  3709      $
\\
&&&&&&& \\
 .4 &$ 1.0918       $&$ .2479       $&$ .5864       $&$ .461\bar{7}$
    &$ 1.156\bar{1} $&$ .2954       $&$ .6987       $&$ .488\bar{8}$
 \\
 .5 &$ 1.0403       $&$  3052       $&$  5587       $&$  5684      $
    &$ 1.1015       $&$  363\bar{7} $&$  6657       $&$  601\bar{8}$
 \\
 .6 &$ 0.9784       $&$  395\bar{5} $&$  525\bar{5} $&$  669\bar{4}$
    &$ 1.0359       $&$  4253       $&$  626\bar{1} $&$  7087      $
 \\
 .7 &$ 0.906\bar{7} $&$  4101       $&$  4869       $&$  763\bar{7}$
    &$ 0.960\bar{0} $&$  488\bar{7} $&$  580\bar{2} $&$  8086      $
\\
&&&&&&& \\
 .8 &$  .8259       $&$ .4567       $&$ .443\bar{6} $&$0.8504      $
    &$  .874\bar{5} $&$ .544\bar{2} $&$ .5285       $&$0.9004      $
\\
 .9 &$   736\bar{9} $&$  4987       $&$  3957       $&$0.9286      $
    &$   7802       $&$  5942       $&$  4715       $&$0.9832      $
\\
1.0 &$   6405       $&$  5357       $&$  344\bar{0} $&$0.9975      $
    &$   678\bar{2} $&$  6383       $&$  409\bar{9} $&$1.056\bar{2}$
\\
1.1 &$   5377       $&$  567\bar{4} $&$  288\bar{8} $&$1.056\bar{5}$
    &$   5693       $&$  6760       $&$  344\bar{1} $&$1.1186      $
\\
&&&&&&& \\
1.2 &$  .429\bar{6} $&$ .593\bar{4} $&$ .230\bar{7} $&$1.104\bar{9}$
    &$  .4548       $&$ .7070       $&$ .274\bar{9} $&$1.169\bar{9}$
\\
1.3 &$   3171       $&$  613\bar{5} $&$  1703       $&$1.1422      $
    &$   335\bar{8} $&$  7309       $&$  2029       $&$1.2094      $
\\
1.4 &$   201\bar{5} $&$  627\bar{4} $&$  1082       $&$1.1682      $
    &$   2133       $&$  7475       $&$  1289       $&$1.2369      $
\\
1.5 &$   083\bar{9} $&$  635\bar{1} $&$  0450       $&$1.182\bar{5}$
    &$   088\bar{8} $&$  756\bar{7} $&$  053\bar{7} $&$1.2520      $
\\
&&&&&&& \\
$ \tfrac{1}{2}\pi$
    &$   0000       $&$  636\bar{7} $&$  0000       $&$1.185\bar{5}$
    &$   0000       $&$  7586       $&$  0000       $&$1.2552      $
\\
&&&&&&&& \\
\hline
\end{tabular} \normalsize

\newpage
\textsc{Table II.---Values of $\cosh(x + iy)$ and $\sinh(x + iy)$.}
(\emph{continued}) \\
\footnotesize
\bigskip
\begin{tabular}{r| rr| rr| rr| rr}
\hline
  & \multicolumn{4}{|c|}{$ x = .8 $} & \multicolumn{4}{|c}{$ x = .9 $}
\\
\cline{2-9}
  $y$ & \multicolumn{1}{c}{$a$}&\multicolumn{1}{c|}{$b$}
      & \multicolumn{1}{c}{$c$}&\multicolumn{1}{c|}{$d$}
      & \multicolumn{1}{c}{$a$}&\multicolumn{1}{c|}{$b$}
      & \multicolumn{1}{c}{$c$}&\multicolumn{1}{c}{$d$}
\\
\hline
&&&&&&& \\
 0  &$ 1.3374      $&$ .0000       $&$  .8881      $&$ .0000      $
    &$ 1.433\bar{1}$&$ .0000       $&$ 1.0265      $&$ .0000      $
\\
 .1 &$ 1.330\bar{8}$&$  088\bar{7} $&$   883\bar{7}$&$  1335      $
    &$ 1.4259      $&$  102\bar{5} $&$ 1.021\bar{4}$&$  143\bar{1}$
\\
 .2 &$ 1.3108      $&$  1764       $&$   8704      $&$  2657      $
    &$ 1.4045      $&$  2039       $&$ 1.006\bar{1}$&$  2847      $
\\
 .3 &$ 1.2776      $&$  262\bar{5} $&$   8484      $&$  3952      $
    &$ 1.3691      $&$  303\bar{4} $&$ 0.980\bar{7}$&$  4235      $
\\
&&&&&&&& \\
 .4 &$ 1.231\bar{9}$&$ .3458       $&$  .8180      $&$ .5208      $
    &$ 1.320\bar{0}$&$ .3997       $&$  .945\bar{5}$&$ .558\bar{1}$
\\
 .5 &$ 1.173\bar{7}$&$  425\bar{8} $&$   779\bar{4}$&$  641\bar{2}$
    &$ 1.257\bar{7}$&$  4921       $&$   9008      $&$  687\bar{1}$
\\
 .6 &$ 1.1038      $&$  501\bar{5} $&$   733\bar{0}$&$  755\bar{2}$
    &$ 1.182\bar{8}$&$  5796       $&$   8472      $&$  809\bar{2}$
\\
 .7 &$ 1.0229      $&$  5721       $&$   679\bar{3}$&$  861\bar{6}$
    &$ 1.096\bar{1}$&$  661\bar{3} $&$   7851      $&$  9232      $
\\
&&&&&&&& \\
 .8 &$  .931\bar{8}$&$ .637\bar{1} $&$  .618\bar{8}$&$0.9595      $
    &$  .9984      $&$ .736\bar{4} $&$  .715\bar{2}$&$1.0280      $
\\
 .9 &$   831\bar{4}$&$  695\bar{7} $&$   552\bar{1}$&$1.0476      $
    &$   8908      $&$  804\bar{1} $&$   638\bar{1}$&$1.1226      $
\\
1.0 &$   7226      $&$  7472       $&$   4798      $&$1.1254      $
    &$   7743      $&$  8638       $&$   5546      $&$1.205\bar{9}$
\\
1.1 &$   606\bar{7}$&$  791\bar{5} $&$   4028      $&$1.1919      $
    &$   6500      $&$  9148       $&$   4656      $&$1.277\bar{2}$
\\
&&&&&&&& \\
1.2 &$  .4846      $&$ .827\bar{8} $&$  .3218      $&$1.2465      $
    &$  .519\bar{3}$&$0.956\bar{8} $&$  .372\bar{0}$&$1.335\bar{7}$
\\
1.3 &$   357\bar{8}$&$  8557       $&$   237\bar{6}$&$1.288\bar{7}$
    &$   383\bar{4}$&$0.9891       $&$   274\bar{6}$&$1.380\bar{9}$
\\
1.4 &$   2273      $&$  875\bar{2} $&$   151\bar{0}$&$1.3180      $
    &$   2436      $&$1.0124       $&$   1745      $&$1.4122      $
\\
1.5 &$   0946      $&$  885\bar{9} $&$   0628      $&$1.334\bar{1}$
    &$   101\bar{4}$&$1.0239       $&$   0726      $&$1.429\bar{5}$
\\
&&&&&&&& \\
$\tfrac{1}{2} \pi$
    &$   0000      $&$ .8881       $&$   0000      $&$1.3374      $
    &$   0000      $&$1.0265       $&$   0000      $&$1.433\bar{1}$
\\
&&&&&&&& \\
\hline
\end{tabular}

\bigskip
\begin{tabular}{r| rr| rr| rr| rr}
\hline
  & \multicolumn{4}{|c|}{$ x = 1.0 $} & \multicolumn{4}{|c}{$ x = 1.1 $}
\\
\cline{2-9}
  $y$ & \multicolumn{1}{c}{$a$}&\multicolumn{1}{c|}{$b$}
      & \multicolumn{1}{c}{$c$}&\multicolumn{1}{c|}{$d$}
      & \multicolumn{1}{c}{$a$}&\multicolumn{1}{c|}{$b$}
      & \multicolumn{1}{c}{$c$}&\multicolumn{1}{c}{$d$}
\\
\hline
&&&&&&& \\
 0  &$1.543\bar{1}  $&$ .0000      $&$1.1752       $&$ .0000      $
    &$1.6685        $&$ .0000      $&$1.3356       $&$ .0000      $
\\
 .1 &$1.535\bar{4}  $&$  1173      $&$1.1693       $&$  154\bar{1}$
    &$1.660\bar{2}  $&$  1333      $&$1.329\bar{0} $&$  1666      $
\\
 .2 &$1.5123        $&$  2335      $&$1.1518       $&$  306\bar{6}$
    &$1.635\bar{3}  $&$  2654      $&$1.3090       $&$  331\bar{5}$
\\
 .3 &$1.474\bar{2}  $&$  347\bar{3}$&$1.1227       $&$  4560      $
    &$1.594\bar{0}  $&$  3946      $&$1.276\bar{0} $&$  493\bar{1}$
\\
&&&&&&&& \\
 .4 &$1.421\bar{3}  $&$  457\bar{6}$&$1.0824       $&$ .6009      $
    &$1.5368        $&$  5201      $&$1.2302       $&$0.649\bar{8}$
\\
 .5 &$1.354\bar{2}  $&$  5634      $&$1.031\bar{4} $&$  739\bar{8}$
    &$1.464\bar{3}  $&$  6403      $&$1.1721       $&$0.7999      $
\\
 .6 &$1.273\bar{6}  $&$  663\bar{6}$&$0.9699       $&$  871\bar{8}$
    &$1.377\bar{1}  $&$  754\bar{2}$&$1.102\bar{4} $&$0.9421      $
\\
 .7 &$1.1802        $&$  757\bar{1}$&$0.8988       $&$  994\bar{1}$
    &$1.276\bar{2}  $&$  8604      $&$1.021\bar{6} $&$1.074\bar{9}$
\\
&&&&&&&& \\
 .8 &$1.075\bar{1}  $&$0.8430      $&$ .818\bar{8} $&$1.1069      $
    &$1.162\bar{5}  $&$0.9581      $&$ .930\bar{6} $&$1.1969      $
\\
 .9 &$0.9592        $&$0.920\bar{6}$&$  7305       $&$1.2087      $
    &$1.037\bar{2}  $&$1.0462      $&$  8302       $&$1.3070      $
\\
1.0 &$0.8337        $&$0.9889      $&$  635\bar{0} $&$1.298\bar{5}$
    &$0.9015        $&$1.1239      $&$  721\bar{7} $&$1.4040      $
\\
1.1 &$0.6999        $&$1.0473      $&$  533\bar{1} $&$1.375\bar{2}$
    &$0.7568        $&$1.1903      $&$  6058       $&$1.487\bar{0}$
\\
&&&&&&&& \\
1.2 &$ .559\bar{2} $&$1.0953       $&$ .4258       $&$1.4382      $
    &$ .6046       $&$1.244\bar{9} $&$ .484\bar{0} $&$1.5551      $
\\
1.3 &$  5128       $&$1.132\bar{4} $&$  314\bar{4} $&$1.486\bar{8}$
    &$  4463       $&$1.287\bar{0} $&$  357\bar{5} $&$1.6077      $
\\
1.4 &$  262\bar{3} $&$1.158\bar{1} $&$  199\bar{8} $&$1.5213      $
    &$  2836       $&$1.3162       $&$  2270       $&$1.6442      $
\\
1.5 &$  109\bar{2} $&$1.172\bar{3} $&$  0831       $&$1.5392      $
    &$  1180       $&$1.332\bar{3} $&$  094\bar{5} $&$1.6643      $
\\
&&&&&&&& \\
$\tfrac{1}{2} \pi$
    &$  0000       $&$1.1752       $&$  0000       $&$1.543\bar{1}$
    &$ .0000       $&$1.3356       $&$ .0000       $&$1.6685      $
\\
&&&&&&&& \\
\hline
\end{tabular} \normalsize

\newpage
\textsc{Table II.---Values of $\cosh(x + iy)$ and $\sinh(x + iy)$.}
(\emph{continued}) \\
\footnotesize
\bigskip
\begin{tabular}{r| rr| rr| rr| rr}
\hline
  & \multicolumn{4}{|c|}{$ x = 1.2 $} & \multicolumn{4}{|c}{$ x = 1.3 $}
\\
\cline{2-9}
  $y$ & \multicolumn{1}{c}{$a$}&\multicolumn{1}{c|}{$b$}
      & \multicolumn{1}{c}{$c$}&\multicolumn{1}{c|}{$d$}
      & \multicolumn{1}{c}{$a$}&\multicolumn{1}{c|}{$b$}
      & \multicolumn{1}{c}{$c$}&\multicolumn{1}{c}{$d$}
\\
\hline
&&&&&&& \\
 0  &$1.810\bar{7} $&$ .0000      $&$1.509\bar{5}$&$ .0000      $
    &$1.9709       $&$  0000      $&$1.698\bar{4}$&$ .0000      $
\\
 .1 &$1.8016       $&$  150\bar{7}$&$1.5019      $&$  180\bar{8}$
    &$1.961\bar{1} $&$  169\bar{6}$&$1.689\bar{9}$&$  196\bar{8}$
\\
 .2 &$1.774\bar{6} $&$  299\bar{9}$&$1.479\bar{4}$&$  359\bar{8}$
    &$1.9316       $&$  3374      $&$1.6645      $&$  3916      $
\\
 .3 &$1.729\bar{8} $&$  446\bar{1}$&$1.4420      $&$  535\bar{1}$
    &$1.882\bar{9} $&$  5019      $&$1.6225      $&$  5824      $
\\
&&&&&&& \\
 .4 &$1.6677       $&$ .5878      $&$1.3903      $&$0.7051      $
    &$1.8153       $&$ .661\bar{4}$&$1.5643      $&$0.7675      $
\\
 .5 &$1.5890       $&$  723\bar{7}$&$1.324\bar{7}$&$0.868\bar{1}$
    &$1.7296       $&$  8142      $&$1.490\bar{5}$&$0.9449      $
\\
 .6 &$1.4944       $&$  8523      $&$1.2458      $&$1.022\bar{4}$
    &$1.626\bar{7} $&$  959\bar{0}$&$1.4017      $&$1.1131      $
\\
 .7 &$1.384\bar{9} $&$  9724      $&$1.154\bar{5}$&$1.166\bar{5}$
    &$1.5074       $&$1.0941      $&$1.299\bar{0}$&$1.2697      $
\\
&&&&&&& \\
 .8 &$1.261\bar{5}$&$1.0828       $&$1.051\bar{7}$&$1.298\bar{9}$
    &$1.3731      $&$1.2183       $&$1.183\bar{3}$&$1.413\bar{9}$
\\
 .9 &$1.1255      $&$1.182\bar{4} $&$0.938\bar{3}$&$1.4183      $
    &$1.2251      $&$1.330\bar{4} $&$1.0557      $&$1.543\bar{9}$
\\
1.0 &$0.9783      $&$1.270\bar{2} $&$0.815\bar{6}$&$1.5236      $
    &$1.064\bar{9}$&$1.4291       $&$0.9176      $&$1.658\bar{5}$
\\
1.1 &$0.8213      $&$1.3452       $&$0.684\bar{7}$&$1.613\bar{7}$
    &$0.8940      $&$1.5136       $&$0.770\bar{4}$&$1.756\bar{5}$
\\
&&&&&&& \\
1.2 &$ .6561      $&$1.406\bar{9} $&$0.547\bar{0}$&$1.6876      $
    &$ .714\bar{2}$&$1.583\bar{0} $&$0.6154      $&$1.837\bar{0}$
\\
1.3 &$  484\bar{4}$&$1.4544       $&$0.403\bar{8}$&$1.744\bar{7}$
    &$  5272      $&$1.636\bar{5} $&$0.4543      $&$1.899\bar{1}$
\\
1.4 &$  307\bar{8}$&$1.487\bar{5} $&$0.256\bar{6}$&$1.7843      $
    &$  3350      $&$1.673\bar{7} $&$0.288\bar{7}$&$1.9422      $
\\
1.5 &$  128\bar{1}$&$1.505\bar{7} $&$0.106\bar{8}$&$1.8061      $
    &$  1394      $&$1.6941       $&$0.1201      $&$1.966\bar{0}$
\\
&&&&&&& \\
$\frac{1}{2}\pi$
    &$  0000      $&$1.509\bar{5} $&$  0000      $&$1.810\bar{7}$
    &$  0000      $&$1.698\bar{4} $&$  0000      $&$1.9709      $
\\
&&&&&&&& \\
\hline
\end{tabular}

\bigskip
\begin{tabular}{r| rr| rr| rr| rr}
\hline
  & \multicolumn{4}{|c|}{$ x = 1.4 $} & \multicolumn{4}{|c}{$ x = 1.5 $}
\\
\cline{2-9}
  $y$ & \multicolumn{1}{c}{$a$}&\multicolumn{1}{c|}{$b$}
      & \multicolumn{1}{c}{$c$}&\multicolumn{1}{c|}{$d$}
      & \multicolumn{1}{c}{$a$}&\multicolumn{1}{c|}{$b$}
      & \multicolumn{1}{c}{$c$}&\multicolumn{1}{c}{$d$}
\\
\hline
&&&&&&& \\
 0  &$2.150\bar{9}$&$ .0000      $&$1.9043      $&$ .0000      $
    &$2.3524$      &$ .0000      $&$2.129\bar{3}$&$ .0000      $
\\
 .1 &$2.1401      $&$  1901      $&$1.8948      $&$  2147      $
    &$2.3413      $&$  2126      $&$2.118\bar{7}$&$  2348      $
\\
 .2 &$2.1080      $&$  3783      $&$1.8663      $&$  4273      $
    &$2.3055      $&$  4230      $&$2.0868      $&$  4674      $
\\
 .3 &$2.0548      $&$  562\bar{8}$&$1.8192      $&$  6356      $
    &$2.2473      $&$  6292      $&$2.034\bar{2}$&$  6951      $
\\
&&&&&&& \\
 .4 &$1.9811      $&$0.741\bar{6}$&$1.7540      $&$0.8376      $
    &$2.1667      $&$0.829\bar{2}$&$1.961\bar{2}$&$0.916\bar{1}$
\\
 .5 &$1.887\bar{6}$&$0.913\bar{0}$&$1.671\bar{2}$&$1.031\bar{2}$
    &$2.0644      $&$1.0208      $&$1.8686      $&$1.1278      $
\\
 .6 &$1.7752      $&$1.075\bar{3}$&$1.5713      $&$1.2145      $
    &$1.9415      $&$1.2023      $&$1.757\bar{4}$&$1.328\bar{3}$
\\
 .7 &$1.6451      $&$1.228\bar{8}$&$1.4565      $&$1.3856      $
    &$1.7992      $&$1.3717      $&$1.628\bar{6}$&$1.515\bar{5}$
\\
&&&&&&& \\
 .8 &$1.4985      $&$1.3661      $&$1.326\bar{8}$&$1.543\bar{0}$
    &$1.6389      $&$1.527\bar{5}$&$1.483\bar{5}$&$1.6875      $
\\
 .9 &$1.3370      $&$1.4917      $&$1.183\bar{8}$&$1.6849      $
    &$1.462\bar{3}$&$1.6679      $&$1.323\bar{6}$&$1.842\bar{7}$
\\
1.0 &$1.162\bar{2}$&$1.6024      $&$1.0289      $&$1.8099      $
    &$1.2710      $&$1.7917      $&$1.150\bar{5}$&$1.979\bar{5}$
\\
1.1 &$0.9756      $&$1.6971      $&$0.8638      $&$1.9168      $
    &$1.067\bar{1}$&$1.8976      $&$0.965\bar{9}$&$2.096\bar{5}$
\\
&&&&&&& \\
1.2 &$ .7794      $&$1.774\bar{9}$&$ .6900      $&$2.0047      $
    &$ .8524      $&$1.984\bar{6}$&$ .771\bar{6}$&$2.1925      $
\\
1.3 &$  5754      $&$1.8349      $&$  5094      $&$2.0725      $
    &$  629\bar{3}$&$2.051\bar{7}$&$  569\bar{6}$&$2.266\bar{7}$
\\
1.4 &$  365\bar{6}$&$1.876\bar{6}$&$  323\bar{7}$&$2.1196      $
    &$  3998      $&$2.0983      $&$  3619      $&$2.318\bar{2}$
\\
1.5 &$  152\bar{2}$&$1.8996      $&$  1347      $&$2.1455      $
    &$  1664      $&$2.1239      $&$  1506      $&$2.3465      $
\\
&&&&&&& \\
$\tfrac{1}{2}\pi$
    &$ .0000      $&$1.9043      $&$  0000      $&$2.150\bar{9}$
    &$ .0000      $&$2.129\bar{3}$&$ .0000      $&$2.3524      $
\\
&&&&&&&& \\
\hline
\end{tabular} \normalsize

\newpage
\textsc{Table III.} \\
\addcontentsline{lot}{table}{Table III.---Values of $\gd u$ and
$\theta^\circ$}
\scriptsize \medskip
\begin{tabular}{r|rr||r|rr||r|rr}
\hline
\multicolumn{1}{c|}{\rule[-5pt]{0pt}{12pt}$u$}
  &\multicolumn{1}{c}{$\gd u$}
  &\multicolumn{1}{c||}{$\theta^\circ$}
  &\multicolumn{1}{c|}{$u$}&\multicolumn{1}{c}{$\gd u$}
  &\multicolumn{1}{c||}{$\theta^\circ$}
  &\multicolumn{1}{c|}{$u$}&\multicolumn{1}{c}{$\gd u$}
  &\multicolumn{1}{c}{$\theta^\circ$}
\\
\hline
   &  &\multicolumn{1}{c||}{$\circ$}
&  &  &\multicolumn{1}{c||}{$\circ$}
&  &  &\multicolumn{1}{c}{$\circ$}
\\
    00 &$  .0000       $&$  0.000 $
&  .60 &$  .5669       $&$ 32.483 $
& 1.50 &$ 1.1317       $&$ 64.843 $
\\
   .02 &$   020\bar{0} $&$  1.146 $
&  .62 &$   583\bar{7} $&$ 33.444 $
& 1.55 &$ 1.152\bar{5} $&$ 66.034 $
\\
   .04 &$   040\bar{0} $&$  2.291 $
&  .64 &$   600\bar{3} $&$ 34.395 $
& 1.60 &$ 1.172\bar{4} $&$ 67.171 $
\\
   .06 &$   060\bar{0} $&$  3.436 $
&  .66 &$   6167       $&$ 35.336 $
& 1.65 &$ 1.1913       $&$ 68.257 $
\\
   .08 &$   0799       $&$  4.579 $
&  .68 &$   6329       $&$ 36.265 $
& 1.70 &$ 1.2094       $&$ 69.294 $
\\
&&&&&&&\\
   .10 &$  .0998       $&$  5.720 $
&  .70 &$  .6489       $&$ 37.183 $
& 1.75 &$ 1.226\bar{7} $&$ 70.284 $
\\
   .12 &$   1197       $&$  6.859 $
&  .72 &$   6648       $&$ 38.091 $
& 1.80 &$ 1.243\bar{2} $&$ 71.228 $
\\
   .14 &$   1395       $&$  7.995 $
&  .74 &$   6804       $&$ 38.987 $
& 1.85 &$ 1.258\bar{9} $&$ 72.128 $
\\
   .16 &$   1593       $&$  9.128 $
&  .76 &$   6958       $&$ 39.872 $
& 1.90 &$ 1.273\bar{9} $&$ 72.987 $
\\
   .18 &$   1790       $&$ 10.258 $
&  .78 &$   7111       $&$ 40.746 $
& 1.95 &$ 1.2881       $&$ 73.805 $
\\
  &  &  &  &  &  & \hrulefill &  &
\\
   .20 &$  .198\bar{7} $&$ 11.384 $
&  .80 &$  .7261       $&$ 41.608 $
& 2.00 &$ 1.3017       $&$ 74.584 $
\\
   .22 &$   218\bar{3} $&$ 12.505 $
&  .82 &$   7410       $&$ 42.460 $
& 2.10 &$ 1.3271       $&$ 76.037 $
\\
   .24 &$   2377       $&$ 13.621 $
&  .84 &$   755\bar{7} $&$ 43.299 $
& 2.20 &$ 1.350\bar{1} $&$ 77.354 $
\\
   .26 &$   2571       $&$ 14.732 $
&  .86 &$   770\bar{2} $&$ 44.128 $
& 2.30 &$ 1.371\bar{0} $&$ 78.549 $
\\
   .28 &$   2764       $&$ 15.837 $
&  .88 &$   7844       $&$ 44.944 $
& 2.40 &$ 1.389\bar{9} $&$ 79.633 $
\\
&&&&&&&\\
   .30 &$  .2956       $&$ 16.937 $
&  .90 &$  .798\bar{5} $&$ 45.750 $
& 2.50 &$ 1.407\bar{0} $&$ 80.615 $
\\
   .32 &$   314\bar{7} $&$ 18.030 $
&  .92 &$   8123       $&$ 46.544 $
& 2.60 &$ 1.422\bar{7} $&$ 81.513 $
\\
   .34 &$   3336       $&$ 19.116 $
&  .94 &$   826\bar{0} $&$ 47.326 $
& 2.70 &$ 1.436\bar{6} $&$ 82.310 $
\\
   .36 &$   352\bar{5} $&$ 20.195 $
&  .96 &$   8394       $&$ 48.097 $
& 2.80 &$ 1.4493       $&$ 83.040 $
\\
   .38 &$   371\bar{2} $&$ 21.267 $
&  .98 &$   8528       $&$ 48.857 $
& 2.90 &$ 1.460\bar{9} $&$ 83.707 $
\\
  &  &  & \hrulefill &  &  &  &
\\
   .40 &$  .3897       $&$ 22.331 $
& 1.00 &$  .865\bar{8} $&$ 49.605 $
& 3.00 &$ 1.4713       $&$ 84.301 $
\\
   .42 &$   408\bar{2} $&$ 23.386 $
& 1.05 &$   897\bar{6} $&$ 51.428 $
& 3.10 &$ 1.4808       $&$ 84.841 $
\\
   .44 &$   4264       $&$ 24.434 $
& 1.10 &$   9281       $&$ 53.178 $
& 3.20 &$ 1.4894       $&$ 85.336 $
\\
   .46 &$   444\bar{6} $&$ 25.473 $
& 1.15 &$   957\bar{5} $&$ 54.860 $
& 3.30 &$ 1.497\bar{1} $&$ 80.715 $
\\
   .48 &$   462\bar{6} $&$ 26.503 $
& 1.20 &$   985\bar{7} $&$ 56.476 $
& 3.40 &$ 1.504\bar{1} $&$ 86.177 $
\\
&&&&&&&\\
   .50 &$  .4804       $&$ 27.524 $
& 1.25 &$ 1.0127       $&$ 58.026 $
& 3.50 &$ 1.5104       $&$ 86.541 $
\\
   .52 &$   4980       $&$ 28.535 $
& 1.30 &$ 1.038\bar{7} $&$ 59.511 $
& 3.60 &$ 1.516\bar{2} $&$ 86.870 $
\\
   .54 &$   5155       $&$ 29.537 $
& 1.35 &$ 1.063\bar{5} $&$ 60.933 $
& 3.70 &$ 1.5214       $&$ 87.168 $
\\
   .56 &$   5328       $&$ 30.529 $
& 1.40 &$ 1.087\bar{3} $&$ 62.295 $
& 3.80 &$ 1.526\bar{1} $&$ 87.437 $
\\
   .58 &$   550\bar{0} $&$ 31.511 $
& 1.45 &$ 1.110\bar{0} $&$ 63.598 $
& 3.90 &$ 1.5303       $&$ 87.681 $
\\
&&&&&&&& \\
\hline
\end{tabular} \\ \normalsize

\bigskip
\textsc{Table IV.} \\
\addcontentsline{lot}{table}{Table IV.---Values of $\gd u, \log\sinh
u, \log\cosh u$}
\medskip \scriptsize
\begin{tabular}{r|r|r|r||r|r|r|r}
\hline \multicolumn{1}{c|}{\rule[-5pt]{0pt}{12pt}$u$}
  &\multicolumn{1}{c|}{$\gd u$}
  &\multicolumn{1}{c|}{$\log\sinh u$}&\multicolumn{1}{c||}{$\log\cosh u$}
  &\multicolumn{1}{c|}{$u$}&\multicolumn{1}{c|}{$\gd u$}
  &\multicolumn{1}{c|}{$\log\sinh u$}&\multicolumn{1}{c}{$\log\cosh u$}
\\
\hline
&&&&&&&\\
  4.0 &$ 1.534\bar{2} $&$ 1.4360        $&$ 1.4363        $
& 5.5 &$ 1.5626       $&$ 2.08758       $&$ 2.0876\bar{0} $
\\
  4.1 &$ 1.537\bar{7} $&$ 1.4795        $&$ 1.4797        $
& 5.6 &$ 1.5634       $&$ 2.13101       $&$ 2.1310\bar{3} $
\\
  4.2 &$ 1.5408       $&$ 1.5229        $&$ 1.5231        $
& 5.7 &$ 1.5641       $&$ 2.17444       $&$ 2.17445       $
\\
  4.3 &$ 1.543\bar{7} $&$ 1.5664        $&$ 1.5665        $
& 5.8 &$ 1.5648       $&$ 2.21787       $&$ 2.21788       $
\\
  4.4 &$ 1.5462       $&$ 1.6098        $&$ 1.6099        $
& 5.9 &$ 1.5653       $&$ 2.36130       $&$ 2.26131       $
\\
  &  &  &  & \hrulefill &  &  &
\\
  4.5 &$ 1.548\bar{6} $&$ 1.6532        $&$ 1.6533        $
& 6.0 &$ 1.5658       $&$ 2.30473       $&$ 2.3047\bar{4} $
\\
  4.6 &$ 1.550\bar{7} $&$ 1.6967        $&$ 1.6968        $
& 6.2 &$ 1.5667       $&$ 2.39159       $&$ 2.3916\bar{0} $
\\
  4.7 &$ 1.5526       $&$ 1.7401        $&$ 1.7402        $
& 6.4 &$ 1.567\bar{5} $&$ 2.47845       $&$ 2.47846       $
\\
  4.8 &$ 1.5543       $&$ 1.7836        $&$ 1.7836        $
& 6.6 &$ 1.568\bar{1} $&$ 2.56531       $&$ 2.56531       $
\\
  4.9 &$ 1.5559       $&$ 1.8270        $&$ 1.8270        $
& 6.8 &$ 1.568\bar{6} $&$ 2.65217       $&$ 2.65217       $
\\
  &  &  &  & \hrulefill &  &  &
\\
  5.0 &$ 1.5573       $&$ 1.8704        $&$ 1.870\bar{5}  $
& 7.0 &$ 1.569\bar{0} $&$ 2.73903       $&$ 2.73903       $
\\
  5.1 &$ 1.5586       $&$ 1.913\bar{9}  $&$ 1.913\bar{9}  $
& 7.5 &$ 1.569\bar{7} $&$ 2.9561\bar{8} $&$ 3.9561\bar{8} $
\\
  5.2 &$ 1.559\bar{8} $&$ 1.957\bar{3}  $&$ 1.9573        $
& 8.0 &$ 1.570\bar{1} $&$ 3.1733\bar{3} $&$ 3.1733\bar{3} $
\\
  5.3 &$ 1.5608       $&$ 2.0007        $&$ 2.0007        $
& 8.5 &$ 1.570\bar{4} $&$ 3.39047       $&$ 3.39047       $
\\
  5.4 &$ 1.561\bar{8} $&$ 2.044\bar{2}  $&$ 2.044\bar{2}  $
& 9.0 &$ 1.5705       $&$ 3.60762       $&$ 3.60762       $
\\
      &                &                 &                &
\multicolumn{1}{c|}{$\infty $}
      &$ 1.570\bar{8} $&\multicolumn{1}{c|}{$\infty $}
                                         &\multicolumn{1}{c}{$\infty $}
\\
&&&&&&& \\
\hline
\end{tabular} \index{Hyperbolic functions!tables of|)} \normalsize
\end{center}

\chapter{Appendix.}

\section{Historical and Bibliographical.}

What is probably the earliest suggestion of the analogy between the
sector of the circle and that of the hyperbola is found in Newton's
Principia (Bk.~2, prop.~8 et seq.) in connection with the solution
of a dynamical problem.\index{Newton, reference to} On the
analytical side, the first hint of the modified sine and cosine is
seen in Roger Cotes' Harmonica Mensurarum (1722), where he suggests
the possibility of modifying the expression for the area of the
prolate spheroid so as to give that of the oblate one, by a certain
use of the operator $\sqrt{-1}$.\index{Cotes, reference to} The
actual inventor of the hyperbolic trigonometry was Vincenzo Riccati,
S.J.\ (Opuscula ad res Phys.\ et Math.\ pertinens, Bononi\ae{},
1757).\index{Riccati's place in the history} He adopted the notation
$\mathrm{Sh.}\phi$, $\mathrm{Ch.}\phi$ for the hyperbolic functions,
and $\mathrm{Sc.}\phi$, $\mathrm{Cc.}\phi$ for the circular ones. He
proved the addition theorem geometrically and derived a construction
for the solution of a cubic equation. Soon after, Daviet de Foncenex
showed how to interchange circular and hyperbolic functions by the
use of $\sqrt{-1}$, and gave the analogue of De Moivre's theorem,
the work resting more on analogy, however, than on clear definition
(Reflex.\ sur les quant.\ imag., Miscel.\ Turin Soc.,
Tom.~1).\index{Foncenex, reference to} Johann Heinrich Lambert
systematized the subject, and gave the serial developments and the
exponential expressions. He adopted the notation $\sinh u$, etc.,
and introduced the transcendent angle, now called the gudermanian,
using it in computation and in the construction of tables (l.~c.\
page 30).\index{Lambert's!place in the history} The important place
occupied by Gudermann in the history of the subject is indicated on
page~\pageref{gudermanian}.\index{Gudermanian!function}

The analogy of the circular and hyperbolic trigonometry naturally
played a considerable part in the controversy regarding the doctrine
of imaginaries, which occupied so much attention in the eighteenth
century, and which gave birth to the modern theory of functions of
the complex variable. In the growth of the general complex theory,
the importance of the ``singly periodic functions'' became still
clearer, and was gradually developed by such writers as Ferroni
(Magnit. expon.\ log.\ et trig., Florence, 1782)%
\index{Ferroni, reference to}; Dirksen (Organon der tran.\ Anal.,
Berlin, 1845)\index{Dirksen's Organon}; Schellbach (Die einfach.\
period.\ funkt., Crelle, 1854)\index{Schellback, reference to}; Ohm
(Versuch eines volk.\ conseq.\ Syst.\ der Math., N�rnberg,
1855)\index{Ohm, reference to}; Ho�el (Theor.\ des quant.\ complex,
Paris, 1870).\index{Ho�el's notation, etc.} Many other writers have
helped in systematizing and tabulating these functions, and in
adapting them to a variety of applications. The following works may
be especially mentioned: Gronau (Tafeln, 1862, Theor.\ und Anwend.,
1865)\index{Gronau's!Tafeln}\index{Gronau's!Theor.\ und Anwend.};
Forti (Tavoli e teoria, 1870)\index{Forti's Tavoli e teoria};
Laisant (Essai, 1874)\index{Laisant's Essai, etc.}; Gunther (Die
Lehre ..., 1881)\index{Gunther's Die Lehre, etc.}. The last-named
work contains a very full history and bibliography with numerous
applications. Professor A.~G.\ Greenhill, in various places in his
writings, has shown the importance of both the direct and inverse
hyperbolic functions, and has done much to popularize their use (see
Diff.\ and Int.\ Calc., 1891).\index{Greenhill's!Calculus} The
following articles on fundamental conceptions should be noticed:
Macfarlane, On the definitions of the trigonometric functions
(Papers on Space Analysis, N.~Y., 1894)\index{Macfarlane on
definitions}; Haskell, On the introduction of the notion of
hyperbolic functions (Bull.\ N.~Y.\ M.\ Soc., 1895).\index{Haskell
on fundamental notions} Attention has been called in Arts.\ 30 and
37 to the work of Arthur E.\ Kennelly in applying the hyperbolic
complex theory to the plane vectors which present themselves in the
theory of alternating currents; and his chart has been described on
page~\pageref{periodicity of hyperbolic functions} as a useful
substitute for a numerical complex table (Proc. A.~I.~E.~E., 1895).
It may be worth mentioning in this connection that the present
writer's complex table in Art.\ 39 is believed to be the only one of
its kind for any function of the general argument $x+iy$.

\medskip
\section{Exponential Expressions as Definitions.}%
\index{Exponential expressions}

For those who wish to start with the exponential expressions as the
definitions of $\sinh u$ and $\cosh u$, as indicated on
page~\pageref{def hyper as exp}, it is here proposed to show how
these definitions can be easily brought into direct geometrical
relation with the hyperbolic sector in the form $\frac{x}{a}=\cosh
\frac{S}{K}$, $\frac{y}{b} = \sinh \frac{S}{K}$, by making use of
the identity $\cosh^2 u - \sinh^2 u = 1$, and the differential
relations $d \cosh u = \sinh u\, du$, $d \sinh u = \cosh u\, du$,
which are themselves immediate consequences of those exponential
definitions. Let $OA$, the initial radius of the hyperbolic sector,
be taken as axis of $x$, and its conjugate radius $OB$ as axis of
$y$; let $OA = a$, $OB = b$, angle $AOB = \omega$, and area of
triangle $AOB = K$, then $K = \frac{1}{2}ab \sin \omega$. Let the
coordinates of a point $P$ on the hyperbola be $x$ and $y$, then
$\frac{x^2}{a^2} - \frac{y^2}{b^2} = 1$. Comparison of this equation
with the identity $\cosh^2 u - \sinh^2 u = 1$ permits the two
assumptions $\frac{x}{a} = \cosh u$ and $\frac{y}{b} = \sinh u$,
wherein $u$ is a single auxiliary variable; and it now remains to
give a geometrical interpretation to $u$, and to prove that $u =
\frac{S}{K}$, wherein $S$ is the area of the sector $OAP$. Let the
coordinates of a second point $Q$ be $x + \Delta x$ and $y + \Delta
y$, then the area of the triangle $POQ$ is, by analytic geometry,
$\frac{1}{2}(x \Delta y - y \Delta x) \sin \omega$. Now the sector
$POQ$ bears to the triangle $POQ$ a ratio whose limit is unity,
hence the differential of the sector $S$ may be written $dS =
\frac{1}{2}(x dy - y dx) \sin \omega = \frac{1}{2} ab \sin \omega
(\cosh^2 u-\sinh^2 u) du = K du$. By integration $S = Ku$, hence $u=
\frac{S}{K}$, the sectorial measure (p.~\pageref{sectoral
measures}); this establishes the fundamental geometrical relations
$\frac{x}{a}=\cosh \frac{S}{K}, \frac{y}{b} = \sinh \frac{S}{K}$.

\addcontentsline{toc}{chapter}{Index}
\printindex

\newpage
\chapter{PROJECT GUTENBERG "SMALL PRINT"}
\small
\pagenumbering{gobble}
\begin{verbatim}





End of the Project Gutenberg EBook Hyperbolic Functions, by James McMahon

*** END OF THIS PROJECT GUTENBERG EBOOK HYPERBOLIC FUNCTIONS ***

***** This file should be named 13692-t.tex or 13692-t.zip *****
This and all associated files of various formats will be found in:
        http://www.gutenberg.net/1/3/6/9/13692/

Produced by David Starner, Joshua Hutchinson, John Hagerson, 
and the Project Gutenberg On-line Distributed Proofreading Team.

Updated editions will replace the previous one--the old editions
will be renamed.

Creating the works from public domain print editions means that no
one owns a United States copyright in these works, so the Foundation
(and you!) can copy and distribute it in the United States without
permission and without paying copyright royalties.  Special rules,
set forth in the General Terms of Use part of this license, apply to
copying and distributing Project Gutenberg-tm electronic works to
protect the PROJECT GUTENBERG-tm concept and trademark.  Project
Gutenberg is a registered trademark, and may not be used if you
charge for the eBooks, unless you receive specific permission.  If you
do not charge anything for copies of this eBook, complying with the
rules is very easy.  You may use this eBook for nearly any purpose
such as creation of derivative works, reports, performances and
research.  They may be modified and printed and given away--you may do
practically ANYTHING with public domain eBooks.  Redistribution is
subject to the trademark license, especially commercial
redistribution.



*** START: FULL LICENSE ***

THE FULL PROJECT GUTENBERG LICENSE
PLEASE READ THIS BEFORE YOU DISTRIBUTE OR USE THIS WORK

To protect the Project Gutenberg-tm mission of promoting the free
distribution of electronic works, by using or distributing this work
(or any other work associated in any way with the phrase "Project
Gutenberg"), you agree to comply with all the terms of the Full Project
Gutenberg-tm License (available with this file or online at
http://gutenberg.net/license).


Section 1.  General Terms of Use and Redistributing Project Gutenberg-tm
electronic works

1.A.  By reading or using any part of this Project Gutenberg-tm
electronic work, you indicate that you have read, understand, agree to
and accept all the terms of this license and intellectual property
(trademark/copyright) agreement.  If you do not agree to abide by all
the terms of this agreement, you must cease using and return or destroy
all copies of Project Gutenberg-tm electronic works in your possession.
If you paid a fee for obtaining a copy of or access to a Project
Gutenberg-tm electronic work and you do not agree to be bound by the
terms of this agreement, you may obtain a refund from the person or
entity to whom you paid the fee as set forth in paragraph 1.E.8.

1.B.  "Project Gutenberg" is a registered trademark.  It may only be
used on or associated in any way with an electronic work by people who
agree to be bound by the terms of this agreement.  There are a few
things that you can do with most Project Gutenberg-tm electronic works
even without complying with the full terms of this agreement.  See
paragraph 1.C below.  There are a lot of things you can do with Project
Gutenberg-tm electronic works if you follow the terms of this agreement
and help preserve free future access to Project Gutenberg-tm electronic
works.  See paragraph 1.E below.

1.C.  The Project Gutenberg Literary Archive Foundation ("the Foundation"
or PGLAF), owns a compilation copyright in the collection of Project
Gutenberg-tm electronic works.  Nearly all the individual works in the
collection are in the public domain in the United States.  If an
individual work is in the public domain in the United States and you are
located in the United States, we do not claim a right to prevent you from
copying, distributing, performing, displaying or creating derivative
works based on the work as long as all references to Project Gutenberg
are removed.  Of course, we hope that you will support the Project
Gutenberg-tm mission of promoting free access to electronic works by
freely sharing Project Gutenberg-tm works in compliance with the terms of
this agreement for keeping the Project Gutenberg-tm name associated with
the work.  You can easily comply with the terms of this agreement by
keeping this work in the same format with its attached full Project
Gutenberg-tm License when you share it without charge with others.

1.D.  The copyright laws of the place where you are located also govern
what you can do with this work.  Copyright laws in most countries are in
a constant state of change.  If you are outside the United States, check
the laws of your country in addition to the terms of this agreement
before downloading, copying, displaying, performing, distributing or
creating derivative works based on this work or any other Project
Gutenberg-tm work.  The Foundation makes no representations concerning
the copyright status of any work in any country outside the United
States.

1.E.  Unless you have removed all references to Project Gutenberg:

1.E.1.  The following sentence, with active links to, or other immediate
access to, the full Project Gutenberg-tm License must appear prominently
whenever any copy of a Project Gutenberg-tm work (any work on which the
phrase "Project Gutenberg" appears, or with which the phrase "Project
Gutenberg" is associated) is accessed, displayed, performed, viewed,
copied or distributed:

This eBook is for the use of anyone anywhere at no cost and with
almost no restrictions whatsoever.  You may copy it, give it away or
re-use it under the terms of the Project Gutenberg License included
with this eBook or online at www.gutenberg.net

1.E.2.  If an individual Project Gutenberg-tm electronic work is derived
from the public domain (does not contain a notice indicating that it is
posted with permission of the copyright holder), the work can be copied
and distributed to anyone in the United States without paying any fees
or charges.  If you are redistributing or providing access to a work
with the phrase "Project Gutenberg" associated with or appearing on the
work, you must comply either with the requirements of paragraphs 1.E.1
through 1.E.7 or obtain permission for the use of the work and the
Project Gutenberg-tm trademark as set forth in paragraphs 1.E.8 or
1.E.9.

1.E.3.  If an individual Project Gutenberg-tm electronic work is posted
with the permission of the copyright holder, your use and distribution
must comply with both paragraphs 1.E.1 through 1.E.7 and any additional
terms imposed by the copyright holder.  Additional terms will be linked
to the Project Gutenberg-tm License for all works posted with the
permission of the copyright holder found at the beginning of this work.

1.E.4.  Do not unlink or detach or remove the full Project Gutenberg-tm
License terms from this work, or any files containing a part of this
work or any other work associated with Project Gutenberg-tm.

1.E.5.  Do not copy, display, perform, distribute or redistribute this
electronic work, or any part of this electronic work, without
prominently displaying the sentence set forth in paragraph 1.E.1 with
active links or immediate access to the full terms of the Project
Gutenberg-tm License.

1.E.6.  You may convert to and distribute this work in any binary,
compressed, marked up, nonproprietary or proprietary form, including any
word processing or hypertext form.  However, if you provide access to or
distribute copies of a Project Gutenberg-tm work in a format other than
"Plain Vanilla ASCII" or other format used in the official version
posted on the official Project Gutenberg-tm web site (www.gutenberg.net),
you must, at no additional cost, fee or expense to the user, provide a
copy, a means of exporting a copy, or a means of obtaining a copy upon
request, of the work in its original "Plain Vanilla ASCII" or other
form.  Any alternate format must include the full Project Gutenberg-tm
License as specified in paragraph 1.E.1.

1.E.7.  Do not charge a fee for access to, viewing, displaying,
performing, copying or distributing any Project Gutenberg-tm works
unless you comply with paragraph 1.E.8 or 1.E.9.

1.E.8.  You may charge a reasonable fee for copies of or providing
access to or distributing Project Gutenberg-tm electronic works provided
that

- You pay a royalty fee of 20% of the gross profits you derive from
     the use of Project Gutenberg-tm works calculated using the method
     you already use to calculate your applicable taxes.  The fee is
     owed to the owner of the Project Gutenberg-tm trademark, but he
     has agreed to donate royalties under this paragraph to the
     Project Gutenberg Literary Archive Foundation.  Royalty payments
     must be paid within 60 days following each date on which you
     prepare (or are legally required to prepare) your periodic tax
     returns.  Royalty payments should be clearly marked as such and
     sent to the Project Gutenberg Literary Archive Foundation at the
     address specified in Section 4, "Information about donations to
     the Project Gutenberg Literary Archive Foundation."

- You provide a full refund of any money paid by a user who notifies
     you in writing (or by e-mail) within 30 days of receipt that s/he
     does not agree to the terms of the full Project Gutenberg-tm
     License.  You must require such a user to return or
     destroy all copies of the works possessed in a physical medium
     and discontinue all use of and all access to other copies of
     Project Gutenberg-tm works.

- You provide, in accordance with paragraph 1.F.3, a full refund of any
     money paid for a work or a replacement copy, if a defect in the
     electronic work is discovered and reported to you within 90 days
     of receipt of the work.

- You comply with all other terms of this agreement for free
     distribution of Project Gutenberg-tm works.

1.E.9.  If you wish to charge a fee or distribute a Project Gutenberg-tm
electronic work or group of works on different terms than are set
forth in this agreement, you must obtain permission in writing from
both the Project Gutenberg Literary Archive Foundation and Michael
Hart, the owner of the Project Gutenberg-tm trademark.  Contact the
Foundation as set forth in Section 3 below.

1.F.

1.F.1.  Project Gutenberg volunteers and employees expend considerable
effort to identify, do copyright research on, transcribe and proofread
public domain works in creating the Project Gutenberg-tm
collection.  Despite these efforts, Project Gutenberg-tm electronic
works, and the medium on which they may be stored, may contain
"Defects," such as, but not limited to, incomplete, inaccurate or
corrupt data, transcription errors, a copyright or other intellectual
property infringement, a defective or damaged disk or other medium, a
computer virus, or computer codes that damage or cannot be read by
your equipment.

1.F.2.  LIMITED WARRANTY, DISCLAIMER OF DAMAGES - Except for the "Right
of Replacement or Refund" described in paragraph 1.F.3, the Project
Gutenberg Literary Archive Foundation, the owner of the Project
Gutenberg-tm trademark, and any other party distributing a Project
Gutenberg-tm electronic work under this agreement, disclaim all
liability to you for damages, costs and expenses, including legal
fees.  YOU AGREE THAT YOU HAVE NO REMEDIES FOR NEGLIGENCE, STRICT
LIABILITY, BREACH OF WARRANTY OR BREACH OF CONTRACT EXCEPT THOSE
PROVIDED IN PARAGRAPH F3.  YOU AGREE THAT THE FOUNDATION, THE
TRADEMARK OWNER, AND ANY DISTRIBUTOR UNDER THIS AGREEMENT WILL NOT BE
LIABLE TO YOU FOR ACTUAL, DIRECT, INDIRECT, CONSEQUENTIAL, PUNITIVE OR
INCIDENTAL DAMAGES EVEN IF YOU GIVE NOTICE OF THE POSSIBILITY OF SUCH
DAMAGE.

1.F.3.  LIMITED RIGHT OF REPLACEMENT OR REFUND - If you discover a
defect in this electronic work within 90 days of receiving it, you can
receive a refund of the money (if any) you paid for it by sending a
written explanation to the person you received the work from.  If you
received the work on a physical medium, you must return the medium with
your written explanation.  The person or entity that provided you with
the defective work may elect to provide a replacement copy in lieu of a
refund.  If you received the work electronically, the person or entity
providing it to you may choose to give you a second opportunity to
receive the work electronically in lieu of a refund.  If the second copy
is also defective, you may demand a refund in writing without further
opportunities to fix the problem.

1.F.4.  Except for the limited right of replacement or refund set forth
in paragraph 1.F.3, this work is provided to you 'AS-IS', WITH NO OTHER
WARRANTIES OF ANY KIND, EXPRESS OR IMPLIED, INCLUDING BUT NOT LIMITED TO
WARRANTIES OF MERCHANTIBILITY OR FITNESS FOR ANY PURPOSE.

1.F.5.  Some states do not allow disclaimers of certain implied
warranties or the exclusion or limitation of certain types of damages.
If any disclaimer or limitation set forth in this agreement violates the
law of the state applicable to this agreement, the agreement shall be
interpreted to make the maximum disclaimer or limitation permitted by
the applicable state law.  The invalidity or unenforceability of any
provision of this agreement shall not void the remaining provisions.

1.F.6.  INDEMNITY - You agree to indemnify and hold the Foundation, the
trademark owner, any agent or employee of the Foundation, anyone
providing copies of Project Gutenberg-tm electronic works in accordance
with this agreement, and any volunteers associated with the production,
promotion and distribution of Project Gutenberg-tm electronic works,
harmless from all liability, costs and expenses, including legal fees,
that arise directly or indirectly from any of the following which you do
or cause to occur: (a) distribution of this or any Project Gutenberg-tm
work, (b) alteration, modification, or additions or deletions to any
Project Gutenberg-tm work, and (c) any Defect you cause.


Section  2.  Information about the Mission of Project Gutenberg-tm

Project Gutenberg-tm is synonymous with the free distribution of
electronic works in formats readable by the widest variety of computers
including obsolete, old, middle-aged and new computers.  It exists
because of the efforts of hundreds of volunteers and donations from
people in all walks of life.

Volunteers and financial support to provide volunteers with the
assistance they need, is critical to reaching Project Gutenberg-tm's
goals and ensuring that the Project Gutenberg-tm collection will
remain freely available for generations to come.  In 2001, the Project
Gutenberg Literary Archive Foundation was created to provide a secure
and permanent future for Project Gutenberg-tm and future generations.
To learn more about the Project Gutenberg Literary Archive Foundation
and how your efforts and donations can help, see Sections 3 and 4
and the Foundation web page at http://www.pglaf.org.


Section 3.  Information about the Project Gutenberg Literary Archive
Foundation

The Project Gutenberg Literary Archive Foundation is a non profit
501(c)(3) educational corporation organized under the laws of the
state of Mississippi and granted tax exempt status by the Internal
Revenue Service.  The Foundation's EIN or federal tax identification
number is 64-6221541.  Its 501(c)(3) letter is posted at
http://pglaf.org/fundraising.  Contributions to the Project Gutenberg
Literary Archive Foundation are tax deductible to the full extent
permitted by U.S. federal laws and your state's laws.

The Foundation's principal office is located at 4557 Melan Dr. S.
Fairbanks, AK, 99712., but its volunteers and employees are scattered
throughout numerous locations.  Its business office is located at
809 North 1500 West, Salt Lake City, UT 84116, (801) 596-1887, email
business@pglaf.org.  Email contact links and up to date contact
information can be found at the Foundation's web site and official
page at http://pglaf.org

For additional contact information:
     Dr. Gregory B. Newby
     Chief Executive and Director
     gbnewby@pglaf.org

Section 4.  Information about Donations to the Project Gutenberg
Literary Archive Foundation

Project Gutenberg-tm depends upon and cannot survive without wide
spread public support and donations to carry out its mission of
increasing the number of public domain and licensed works that can be
freely distributed in machine readable form accessible by the widest
array of equipment including outdated equipment.  Many small donations
($1 to $5,000) are particularly important to maintaining tax exempt
status with the IRS.

The Foundation is committed to complying with the laws regulating
charities and charitable donations in all 50 states of the United
States.  Compliance requirements are not uniform and it takes a
considerable effort, much paperwork and many fees to meet and keep up
with these requirements.  We do not solicit donations in locations
where we have not received written confirmation of compliance.  To
SEND DONATIONS or determine the status of compliance for any
particular state visit http://pglaf.org

While we cannot and do not solicit contributions from states where we
have not met the solicitation requirements, we know of no prohibition
against accepting unsolicited donations from donors in such states who
approach us with offers to donate.

International donations are gratefully accepted, but we cannot make
any statements concerning tax treatment of donations received from
outside the United States.  U.S. laws alone swamp our small staff.

Please check the Project Gutenberg Web pages for current donation
methods and addresses.  Donations are accepted in a number of other
ways including including checks, online payments and credit card
donations.  To donate, please visit: http://pglaf.org/donate


Section 5.  General Information About Project Gutenberg-tm electronic
works.

Professor Michael S. Hart is the originator of the Project Gutenberg-tm
concept of a library of electronic works that could be freely shared
with anyone.  For thirty years, he produced and distributed Project
Gutenberg-tm eBooks with only a loose network of volunteer support.

Project Gutenberg-tm eBooks are often created from several printed
editions, all of which are confirmed as Public Domain in the U.S.
unless a copyright notice is included.  Thus, we do not necessarily
keep eBooks in compliance with any particular paper edition.

Most people start at our Web site which has the main PG search facility:

     http://www.gutenberg.net

This Web site includes information about Project Gutenberg-tm,
including how to make donations to the Project Gutenberg Literary
Archive Foundation, how to help produce our new eBooks, and how to
subscribe to our email newsletter to hear about new eBooks.

*** END: FULL LICENSE ***

\end{verbatim}
\normalsize


\end{document}

